\section{Mixed Multiband Path Analysis}
\label{sec:wmalgorithms}


In this section, we discuss the influence of ~\emph{Multiband} on ~\emph{Multihop Path} in mesh network. 
Accodirng to these analysis, we develop two algorithms for ~\emph{Channel Assignment} in multi-band multi-radio scenario.

% PEN part 
\subsection{Path Interference Induced on the Network}
\label{subsec:PEN}

In WhiteMesh networks, multihop paths can be intermixed with WiFi 
for more spacial reuse and white space bands with less hops.  
To deal with the trade-off, we consider
analyze the band choices reduce the number of hops along a path and the 
aggregate level of interference that hop-by-hop path choices have
on the network (i.e., Path Interference induced on the Network).

Mesh nodes closer to the gateway generally achieve
greater levels of throughput at sufficiently high offered loads. 
To combat such starvation effects, we treat each flow with equal priority in the network when
assigning channels. In the worst case, all nodes along a particular path have equal 
time shares for contending links (i.e., intra-path interference).
We start the channel assignment assuming that $h$ mesh nodes are demanding
traffic from each hop of an $h$-hop path to the gateway. If each link along the 
path uses orthogonal channels, then each link could be active simultaneously,
otherwise they will complete with each other. 
We note each node along the path had traffic demand $T_d$, obviously the bottleneck 
link along the path would be the one closest to the gateway, and then next. 
Thus, the total traffic along the path $h \cdot T_d$ must be less than the 
bottleneck link's capacity $\delta$ estimated from the measurements. In such a scenario, the $h$-hop mesh node 
would achieve the minimum served demand, which we define as the network efficiency. 
In general, the active time per link for an $h$-hop mesh node can be represented 
by $1,\frac{h-1}{h},\frac{h-2}{h}\cdots \frac{1}{h}$. The summation of all active 
times for each mesh node along the path is considered the intra-path network cost.

Considering only intra-path interference, using lower carrier frequencies allows a
reduction in hop count and increase in the network efficiency of each mesh node along
the $h$-hop path. However, a lower carrier frequency will induce greater interference
to other paths to the gateway (i.e., inter-path interference). 
Fig.~\ref{fig:interferencerange} depicts such an example where
links in different bands are represented by circles for 450 MHz, rectangles for
2.4 GHz, and triangles for the nodes which can choose between the two.
Nodes $A$ and $C$ could be connected through two 2.4-GHz links or a single 450-MHz link.
With 2.4 GHz, the interfering distance will be less than using 450 MHz. For example, only 
link $D,E$ will suffer from interference, whereas $H,I$ would not. However, with 450 MHz,
link $A,C$ would interfere with links $F,G$, $M,L$, and $K,J$. At each time unit, the number of
links interfering with the active links along a path would be the inter-path network cost.

When an $h$-hop flow is transmitted to a destination node, it prevents 
activity on a number of links in the same frequency via the protocol model. 
The active time on a single link can be noted as 
$\frac{T}{\gamma_h}$. 
An interfering link from the conflict matrix $F$ counts as $I_h$ per unit time
and contributes to the network cost in terms of:
$\frac{hT}{\gamma_1}\cdot I_1 + \frac{(h-1)T}{\gamma_2}\cdot I_2 \cdots \frac{T}{\gamma_h}\cdot I_h$.
Then, the traffic transmitted in a unit of network cost for the $h$-hop node is:
\begin{equation}
\label{eq:originpen}
E_{\eta}=\frac{T}{\sum_{i \in h}\frac{(h-i+1)\cdot T}{\gamma_i}\cdot I_i }
\end{equation}
Using network efficiency, the equation simplifies to:
\begin{equation}
\label{eq:pen}
E_{\eta}=\frac{\gamma}{\sum_{i \in h} (h-i+1)\cdot I_i}
\end{equation}

The network efficiency is the amount of traffic that could be 
offered on a path per unit time. With multiple channels from the same band,
$I_i$ will not change due to the common communication range. With multiple
bands, $I_i$ depends on the band choice due to the communication range diversity.  
This network efficiency jointly considers hop count and interference. We define
the Path Interference induced on the Network (PIN) as the denominator of Eq.~\ref{eq:pen},
which represents the sum of all interfering links in the network by a given path. 
PIN is used to quantify the current state of channel for channel assignment
across WiFi and white space bands.
To determine when the lower carrier frequency will be better than two or more hops at a
higher carrier frequency, we consider the average interference $\bar{I}$ of a given path
at the higher frequency.  The problem could be formulated as:
\begin{equation}
\label{eq:benefit}
\frac{\gamma}{\frac{h(h-1)}{2}\cdot \bar{I}+I_x} \geq \frac{\gamma}{\frac{h(h+1)}{2}\cdot \bar{I}}
\end{equation}

Here, from Eq.~ref{eq:benefit} when $I_x \leq 2\cdot h\bar{I}$, the performance of a lower-frequency link  
is better than two higher-frequency hops for the same destination node. $I_x$ is also a parameter of hop count 
in Eq.~\ref{eq:pen}. When the hop count is lower which closer to the gateway node, the threshold 
would be more strict since the interference would have a greater effect on the performance.





The discussion in \~ref{subsec:PEN} provide the methodology to improve channel assignment. But the difficulty of channel assignment is that before the process has been done, it could no be evaluated to tell which is better.
To approach the solution, we propose two local search based heuristic algorithms to adapt the multiband scenario. 


\subsection{Growing Spanning Tree Algorithm}
In a mesh network, gateway nodes always building in the most busy location ~\cite{robinson2008adding, he2008optimizing}.
As the service tree rooted at a gateway grows, the links closer to the gateway, the more interference will happen.
And in the edge of the network, it is less populated in which cases reduce hop count through lower frequency may bring more benefit. 
The main idea behind the ~\emph{Growing Spanning Tree} Algorithm is 
to find the link has least interference on the network for each node in a greedy manner each step. The hop count for gateway nodes themselves are 0.
We first initialize the mesh nodes ranking with the distance to all the gateway nodes. In the ranking order of the mesh nodes, the 1 hop links from gateway nodes ranking with the ~\emph{Path Interference}. Then select the lowest interferred link for this node and update the assignment information for next steps. 
Iterate these steps to assign channels for all mesh nodes. This process is phase 1 of our algorithm which is similar to but not exactly the same as the breath first search channel assignment. 

In phase 2 of the algorithm, we sort the mesh nodes with their hop count to gateway nodes. 
The algorithm traverses all the nodes whose hop count are less than the current node. If there are radio slots for the less hop nodes, it is possible to re-connect the mesh node to reduce the hop count. We rank all possible option with their path interference, then choose the lowest one re-connect the mesh node. If there exist new link has the same path interference, we count the number of nodes has connected to the gateway nodes, select the gateway has less node connected. Phase 2 process will be interated till no changes in the network.

The ~\emph{Growing Spanning Tree} Algorithm is described in ~\ref{algorithms:gsp}.


% Need to talk about how to improve the bottle neck links,
%FIXME talk about BFS-CA 





\begin{algorithm}
    \small
\caption{Multiband Growing Spanning Tree Algorithm}
\label{algorithms:gsp}
\begin{algorithmic}[1]
\REQUIRE  ~~\\
	 $M$: The set of all mesh nodes\\
	 $G$: The set of gateway nodes\\
	 $C$: Communication graph of potential links among all nodes\\
	 $I$: Interference matrix of all potential links \\
	 $B$: Available frequency bands
\ENSURE ~~\\    
$CA$: Channel Assignment of the Network\\
\STATE Initial $S_{current}=G$, $N_{served}=\emptyset$, $N_{unserved}=M$,$I_{active}=\emptyset$
\STATE Rank mesh nodes according to their distance to gateway nodes
\WHILE {$N_{served}=M$}
\FORALL {$s \in S_{current}$}
	\STATE Find one-hop nodes in $S_{Next}$
	\STATE Sort $S_Next$ according to distance from gateway nodes, shorter distance first
	\FORALL {$l \in S_{Next}$}
		\STATE Calculate one-hop path interference of link $s\rightarrow l$
		\STATE Sort the links, choose the one has the least path interference
		\STATE Assign(s,l) with the least interference link
		\STATE Update $N_{served},N_{unserved}$
		\STATE Update $I_{active}$ from $I$
	\ENDFOR
	\STATE $S_{current}=S_{Next}$
\ENDFOR
\ENDWHILE
% Round 2, resolve bottle neck problem
% After above, all node has its shortest path, but exist bottle neck, at least have a connection
% If could add band, add band to the high occupacied channel
\STATE Sort mesh nodes with their hop counts to gateway nodes $N_{sorted}$
\WHILE {Change of Channel Assignment Exist} 
\FORALL {$s \in N_{sorted}$}
	\STATE Tranverse all the 1 hop arrived nodes have less hop count than node $s$ 
	\STATE Check if these nodes have radio slots for node $s$
	\STATE Sort path through possible nodes with the path interference
	\STATE Choose a new path if it has less interference than the previous one
	\STATE If more than one path has the same interference, choose the gateway node has least leaves nodes 
\ENDFOR
\ENDWHILE

Output $Channel Assignment$ as Solution
\end{algorithmic}
\end{algorithm}
      

%FIXME need more explaination of the tree growing algorithm
%The algorithm use the average $\bar{I}$ and average hop count to approach the channel assignment
%In ~\cite{robinson2008adding}, Robinson talked about the bottle neck of a network is the links neighbor to the gateway nodes.





\subsection{Best Path Selection Algorithm}
\label{subsec:step}

Based on the previous path efficiency analyze, the network efficiency is related to each link's interference and the distance to gateway nodes. To find a path for each mesh node, which could be converge to a shortest weight path detect
We define weight for each link and improve Dijikstra's algorithm with ~\emph{PEN} weight to find the best path for each mesh node ~\cite{golden1976shortest}.
To run Dijikstra's algorithm, we define two parameter of each link between two nodes. First is the existing interference of the link $I_w$, we mark the interference of bands as multiple links. 
The second is the ~\emph{Load Weight} of a link $l_w$, which is the number of path chosen this link. In Dijikstra's algorithm, the weight is calculated as denominator of ~\emph{PEN}, since the numerator is the same among different bands.
This parameter used to adjust the ~\emph{PEN} with bottle neck links.
The weight of Dijikstra's algorithm is related to hop order $h_i$ according to the definition of ~\emph{PEN}, the weight is calculated as $I_w\times h_i \times \l_w$.
We iterately find the best path of each node and update the parameters in the graph.





\begin{algorithm}
    \small
\caption{Sink To End Path Algorithm}
\label{algorithms:step}
\begin{algorithmic}[1]
\REQUIRE  ~~\\
	$M$: The set of all mesh nodes\\
	$G$: The set of gateway nodes\\
	$C$: Communication graph of potential links among all nodes\\
	$I$: Interference matrix of all potential links \\
	$B$: Bands amount
\ENSURE ~~\\    
$CA$: Channel Assignment from Gateway nodes to Mesh nodes\\
\WHILE {$notAllnodesVisited(M)$}
\STATE Intialize $CA,I_w,l_w$\\
		\STATE Run Dijistra's Algorithm with $C,I,B$ to all the Gateways
		\STATE Compare the $PEN$ to all Gateway node
		\STATE Choose the best one adding to $CA$
		\STATE Update $I_w,l_w$
\STATE Calculate $\triangle cost$ for all valid operations\\
\STATE Apply $swap$ with largest positive $\triangle cost$
\ENDWHILE 

Output $CA$ as locally optimal solution\\
\end{algorithmic}
\end{algorithm}


