\section{Mixed Multiband Multihop Path and Solutions}
\label{sec:algorithms}


In this section, we discuss the influence of ~\emph{Multiband} on ~\emph{Multihop Path} in mesh network. 
Accodirng to these analysis, we develop two algorithms for ~\emph{Multiband Channel Assignment}.

% Talk about the network efficiency for multiband multihop mixed hop

In ~\emph{Multiband Multiradio Network}, 
a multihop path could have higher frequency band combination with less interference range or a set of lower frequency band with less hop count.
A key issue of multihop path in such network is to answer which combination is better.
We focus our work on ~\emph{Channel Assignment} dealing with more interference factors rather than routing protocol which would be more concern on delay. Other architecture also has such problem such as wireless sensor network.

To discuss this problem, we pick up a multihop path from mesh network and analyze its performance with worst case hypothesis. In mesh network, such a path would have a bottle neck in the link closet to gateway.
When a mesh network was built with gateway placement, constructor should considered load-aware demand of mesh nodes and mesh node population. 
Generally the nodes close to gateway should have more traffic demand and gateway itself should have the most connectivity population. 
We treat each node equally binding with fairness, otherwise mesh nodes close to gateway could be served more traffic and show a high goodput of the network.
For analyze, we assume all the node in the path equally share the time of the link next to a gateway. It is also the worst case for getting a larger goodput.


First, we introduce the ~\emph{Intra-Path} traffic. When we have a multihop path, in worst case all the nodes on the path have only one $h$ hop path arrived at a gateway node. The path is made of links from one node to another.
Each node has traffic $T$, nomatter uplink or downlink since both of them occupy link capacity in the same way. And the total traffic on the path $\sum T$ is less than the bottle neck link capacity $C$. 

We define the minimum transmission rate on a path as ~\emph{Network Efficiency}. 
With the fairness restriction, the last node in the path has the minimum transmission rate.
Then the acitve time in a time unit of each link can be represented as $1,\frac{h-1}{h},\frac{h-2}{h}\cdots \frac{1}{h}$. 
The unit time of each link in the path is counted as total cost time of network.
%\begin{equation}
%\label{eq:intrapath}
%\begin{split}
%E_{Intra-Path}=\frac{Path\ Active\ Time}{Network\ Time}\\
%E_{Intra-Path}=\frac{1}{2}+\frac{1}{2\cdot h}
%\end{split}
%\end{equation}


%As hop count increase, the ~\emph{Intra-Path} will decrease till the lower bound $\frac{1}{2}$. With routing protocol which is out of this work, the delay increase too.
Without considering ~\emph{Inter-Path} interference which represent interference with links out of the path, 
an intuition of using lower band is to reduce the hop count
 to increase the minimum time utility rate which is the active time of the last link over the total active time of the path. 
However, at the same time, the interference range increase too. An example shown in ~\ref{fig:networkefficiency}, 
the picture shows links in different bands, let's say 2.4GHz and 900MHz, as a sketch map, does not represent the real distance.
Node $A,C$ could be connected through two 2.4GHz links or a single 900MHz link; with 2.4GHz links, only link $D,E$ will be interferenced; however, with 900MHz $A,C$ link, link $F,G;M,L;K,J$ will be interferenced. 


\begin{figure}
%\vspace{-0.0in}
\centering
\includegraphics[width=74mm]{figures/networkefficiency}
\vspace{-0.1in}
\caption{Path Network Efficiency Introduction, Solid Wire notes 2.4GHz link, Dashed line notes 900MHz}
\label{fig:networkefficiency}
%\vspace{-0.0in}
a\end{figure}

To quantization this ~\emph{Inter-Path Interference}, 
the unit time of these links are counted as ~\emph{Network Time}. 
When a $h$ hop path transmitting traffic $T$ for the destination node, it stops activity on a number of links in the same band. 
In a multihop path, when the traffic arrived at the last destination node, all the previous links are serving for these traffic.
The active time on a single link can be noted as 
$\frac{T}{c_h}$. We keep in the worst case when the last node in the path got traffic $T$, the other node also be served traffic $T$.
With interference counts $I_h$ from the conflict matrix:
the ~\emph{Network Time} counted as 
$\frac{hT}{c_1}\cdot I_1 + \frac{(h-1)T}{c_2}\cdot I_2 \cdots \frac{T}{c_h}\cdot I_h$, the ~\emph{Path Efficiency over Network} is defined the traffic over the ~\emph{Network Time} and could be represented as:



\begin{equation}
\label{eq:originpen}
E_{PEN}=\frac{T}{\sum_{i \leq h}\frac{i\cdot T}{c_h}\cdot I_i }
\end{equation}

With protocol model, if link exist, then they have the same capacity $c_1=c_2 \cdots =c_h=c$. 
To avoid $0$ value in the denominator, we add a $1$ to adjust the denominator which does not change the parameter characteristics. 
The \emph{Path Efficiency over Network}could be represented as:


\begin{equation}
\label{eq:pen}
E_{PEN}=(\frac{c}{1+\sum_{i \leq h} i\cdot I_i}
\end{equation}
 

The meaning of the ~\emph{Network Efficiency} is that in a unit time, the traffic could be loaded by this path. In multichannel scenario, all the channel will have the same communication range, this parameter equals to the conflic graph in many multichannel works which try to minimize the interference~\cite{jain2005impact}. Since we count only one channel not all possible links, it also could be seen as an extention of a single link ~\emph{Link Load} defined in ~\cite{raniwala2004centralized}.

The ~\emph{Path Efficiency over Network} connect hop counts and interference. 
Then we discuss when a lower ~\emph{White Space Band} is better to be used in a path.
In a path, we use an average interference count $\bar{I}$ replace each interference count with assumption the links in the path all in one higher freq band. Then a ~\emph{White Space Band} is used to replace two links in the path as a single link with interference count $X$ represent one of the factor $i\cdot I_i$. The problem could be formulated as:

 
\begin{equation}
\label{eq:benefit}
\frac{c}{1+\frac{h(h-1)}{2}\cdot \bar{I}+X} \geq \frac{c}{1+\frac{h(h+1)}{2}\cdot \bar{I}}
\end{equation}

From the inequation, when $X \leq 2\cdot h\bar{I}$ a lower band could be better. $X$ is also a function of hop order in the path, generally the path order lower, the threshold would be more strict; otherwise it could be loose. It matches the intuition the hop order is small, it close to the gateway, it should be more crowd for this link.
It helps to tell the ranking of set of links and a path where we can start to resolve channel assignment problem.












Then ~\emph{Tree Generated Algorithm} and ~\emph{Heuristic Efficiency Improvement Algorithm} are introduced  

We propose two local search based algorithms with $Band Assign$ operation to adapt the multiband scenario. The gateway placement problem could be formulated as an integer program (IP), however, the solution based on IP has following disadvantages: (i) can not be solved exactly in polynomial time,
   (ii) has an unbounded integrality gap
   (iii) IP is not suitable for online computation ~\cite{robinson2008adding}.

   To approach the maximizing capacity, we switch to local search algorithms.
   ~\emph{Local Search Algorithms} optimize one of the two major components of our capacity calculation: the size of the routes in $R$ or the the impact of contention in $I$ on mesh nodes.

   % Analyze the complexity of the algorithms, need to know about the two proposed algorithms
In the worst case, without proper algorithms to approach the optimal placment, running the brute force algorithm to find the optimal placement is almost an impossible mission. The combination of placement in a $n$ nodes network is $2^n$, with the capacity of calculation for each node, the complexity is $O(n2^n)$. It is a nightmare even for a powerful server. That is why we have to approach the optimal placement verse finding the solution by traversing all possible options.

   Previous work J, Robinson proposed two ~\emph{Local search algorithms} to optimize the capacity for single band mesh network deployment ~\cite{robinson2008adding}. 
   We therefore develop the ~\emph{Local Search Algorithms} to adpot multiband scenario.

   \subsection{Minimizeing Hop Count}
   \label{subsec:minhop_intro}

   % Foreach all mesh nodes
   The available gateway locations are $W$, which is a specific subset of all mesh node locations. $G$ represent the set of installed gateway with multiple radios work in different bands locations throughout the execution of the algorithm, meaning if there is a gateway node work in band $k$ in the location, $G[i,k]=1$.
   We start to add network capacity from an designed deployment single mesh placement, since there are tons of methods for a single band mesh deployment ~\cite{akyildiz2005wireless}.


   We perform ~\emph{add(), assign(), open() and, close()}, to output a deployment of mesh network. 
   To terminate the process, we require that each step lowers the cost by at least $c(S)/p(n,\varepsilon)$, $S$ is the deployment for this step, $p(n,\varepsilon)$ is a chosen polynomial in $n$ and $1/ \varepsilon$.


   $add(s)$ installs a multiband mesh node at potential location $s$, $assign(s,T)$ install a gateway at node $s$ and discover the coverage of different band, assign the band to different mesh nodes according to the demands and fairshare of the nodes. $open(s,T)$ installs a gateway at location $s$ and removes all the gateway nodes in set $T$, and $close(s,T)$ removes the gateway at location $s$ and installs gateway nodes at all nodes in set $T$.

   As found in ~\cite{robinson2008adding} all possible combinations for set $T$ can not be evaluated in polynomial time, we inherited the methodology of solving a knapsack problem for the set $T$ discovering. $T$ is found as the set of items to put in the knapsack which has an arbitrary capacity.
   The details of the operation is described as:

   % FIXME need to change the type as 

%FIXME should change the order of assign and add(s)

   % Add a node
   \begin{itemize}
   \item add(s) for all non-gateway nodes $s$, evaluate the cost to open a gateway at $s \in W$. This cost evaluation requires solving a transhipment problem to find optimal routing matrix $R$ for the set of all installed gateways in $G \cap \{s\}$
   % Connect to all gateway nodes to see if can remove them according
   \item assign(s,T) Assign the bands to the nodes in coverage according to the demands and contention. Distribute the capacity of a multiband gateway node to mesh nodes. 
    We first assign the lower band for remote mesh nodes and then the higher band for the close nodes. If we have more bandwidth than required for the first step, then we assign the remaining higher bandwidth to close nodes and then the lower bands for all nodes in coverage since the nodes close to the gateway may suffer more contention. The process of assign helps to reduce the hop count/contention of the network.
   The operation adopt the propagation characteristics of different making the problem scale of transship and knapsack become smaller.
   % Try to keep the demand and remove relay node
   \item open(s,T) Install gateway at location $s \in W$ and remove gateways in set $T \subseteq G-\{s\}$, transfer the traffic on $T$ to teh gateway nodes $s$. Gateway $s$ could be a node with some unused capacity.
   \item close(s,T) Remove $s \in G$ and install a set of gateways $T \subseteq W-\{s\}$. Then reassign routes destined to $s$ to gateways in $T$ without any effect on mesh nodes served by other gateways. $T$ should be gateway nodes have unused capacity.
   \end{itemize}
  % calculate the capacity


	  \begin{algorithm}
          \small
	  \caption{Multiband MinHopCount Algorithm}
	  \label{algorithms: Minihop}
	  \begin{algorithmic}[1]
	  \REQUIRE  ~~\\
		 $M$: The set of all mesh nodes\\
		 $u_{init}[i]$: Initialize values of capacities
\ENSURE ~~\\    
$G$: Installed Gateway Locations \\
Valid Here means satisfies budget \\
Start with arbitrary, valid solution G
\FORALL {$s \in M$} 
\WHILE {$\sum_{i=1}^N u_{prev}[i]-u_{cur}[i] \ge \phi$}
\WHILE {$\triangle cost \ge c(G)/p(n,\epsilon)$}
\STATE Find valid add(s)
\STATE Assign(s,T)
\STATE Find valid open(s,T) \\
where $T$ is solution to knapsack problem with knapsack size of u[s]
\STATE  Find valid close(s,T) \\
where $T$ is minimal covering kanpsack with knapsack size of u[s] \\
Calculate $\triangle cost$ for all valid operations\\
Apply operation to $G$ with best $\triangle cost$

\ENDWHILE
\\Output G as locally optimal solution\\
Calculate capacities $\hat{u}[i]$ of placement G\\
Update $u_{cur}[i]$ to new lower bound if $\hat{u}[i] < u_{prev}[i]$ \\
\ENDWHILE 
\ENDFOR
Output G as Solution
\end{algorithmic}
\end{algorithm}
      


	Hop count is a first-order approximation of the capacity. The relation between the capacity and the hop count is the decrease of the hop count will reduce the contention and increase the capacity in \ref{eq:contention}.
			The advantage of hop count as the cost function is that it preserves the triangle inequality, which provable the upbound of network capacity.
The local search operations are not able to know a priori of the placement, we use lower bound estimates for the gateway capacities $u[i]$ and update the capacity every successful interation.
The process will end when the current sum of the lower bound capacity estimate $u_{cur}[i]$ does not decrease by more than user-chosen parameter $\phi$ from the previouse estimate, $u_{prev}[i]$.
In the process, subject to the lower capacity bound, we capture the optimal placement. The run time of the ~\emph{Multiband MinHopCount Algorithm} is polynomial in $\frac{1}{\epsilon}$ and $\frac{1}{\phi}$.

			% Calculate the cost

			% Update the deployment


\subsection{MinContention}
\label{subsec:mincontention}

The second approaching, Multiband MinContention, finds the gateway placement that minimizes the average contention in the network.
The main idea of the Multiband MinContention algorithm is to install $k$ gateways to minimize the average contention on the mesh nodes, related to the links contend with each node and how often these links are used in routes.
We can not count the total contention on gateways, since the full gateway placement is not known in advance.
The problem could be mapped to a $k$-median problem ~\cite{robinson2008adding}.
In this work, we introduce the k-median problem and propose a ~\emph{Multiband MinContention} local search algorithm. We find the placement with the lowest average contention region size.

\subsubsection{k-median}
The k-Median Problme is a variant of the facility location problem where there are only a fixed number k of facilities that can be opened. The objective is to minimize the cost of connecting all clients to a facility. 
There is a local search algorithm for the uncapacitated k-median problem with a locality gap of $3+2/p$ ~\cite{arya2004local},
where the locality gap is the maximum difference between the worst local optimum and the global optimum and the parameter $p$ controls the number of gateways the algorithm considers for simultaneous swapping. 
This locality gap results in an approximation ratio of $3+\epsilon$. The algorithm repeatedxly swap $p$ open gateways for $p$ unopen gateways iuntil no swaps can improve the solution. Larger $p$ can improve the accuracy but result in exponential increase running time.

For multiband scenario, to employ the local search algorithm, designer need to balance the contention among different bands. It happens that one band may reach the best contention placement but other bands are not in optimization status. The main difference from single band MinContention algorithm is that when we choose an additional gateway node, we try to assign the same number of one hop links for different band. To implement the equal assignment, we prefer to use higher frequency band since there click is smaller than lower frequency band. A similar process is designed to start from assigning higher band first and then lower band with interation.

\subsubsection{Swap-Function}
~\emph{ Swap-based Local Search}: The Multiband MinContention algorithm is summarized in ~\ref{algorithms:Mincontention}. THe cost of a placement is the sum of the active link weights, which are each sassigned to be the total number of mesh nodes in contention range of the link. 


	  \begin{algorithm}
          \small
	  \caption{Multiband MinContention Algorithm}
	  \label{algorithms:Mincontention}
	  \begin{algorithmic}[1]
	  \REQUIRE  ~~\\
		 $M$: The set of all mesh nodes\\
		 $u_{init}[i]$: Initialize values of capacities
\ENSURE ~~\\    
$G$: Installed Gateway Locations \\
Valid Here means satisfies budget \\
Start with arbitrary, valid solution G
\WHILE {$\triangle cost \ge C(G)/p(n,\epsilon)$}
\STATE Find all valid swap(S,T)\\
where S is set of p gateways to open\\
and T is set of p gateways to close
\STATE Assign(S) \\
Assign equal number of one hop mesh node for different band
\STATE Calculate $\triangle cost$ for all valid operations\\
\STATE Apply $swap$ with largest positive $\triangle cost$
\ENDWHILE 

Output G as locally optimal solution\\
\end{algorithmic}
\end{algorithm}

\subsubsection{triangle}
~\emph{Triangle Inequality for Contention} A simple example is a three nodes mesh network. The contention caused by link AB is less than the sum of the contention of links AC and BC. This characteristic is useful to find the minimum contention links.


bitcpf: ~\ref{subsec:minhop_intro}
	  \ref{algorithms: Minihop}

