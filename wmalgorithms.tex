\section{Path Analysis with Diverse Propagation}
\label{sec:wmalgorithms}


In this section, we discuss the influence of diverse propagation
characteristics of the wide range of carrier frequencies of
% NEWClaim fix
white space and WiFi bands. We then introduce our measurement 
driven heuristic algorithm for channel assignment in 
WhiteMesh networks.

% PEN part 
\subsection{Path Interference Induced on the Network}
\label{subsec:PEN}

In WhiteMesh networks, multihop paths can be intermixed with WiFi 
for more spacial reuse and white space bands with less hops.  
To deal with the trade-off, we consider
analyze the band choices reduce the number of hops along a path and the 
aggregate level of interference that hop-by-hop path choices have
on the network (i.e., Path Interference induced on the Network).

Mesh nodes closer to the gateway generally achieve
greater levels of throughput at sufficiently high offered loads. 
To combat such starvation effects, we treat each flow with equal priority in the network when
assigning channels. In the worst case, all nodes along a particular path have equal 
time shares for contending links (i.e., intra-path interference).
We start the channel assignment assuming that $h$ mesh nodes are demanding
traffic from each hop of an $h$-hop path to the gateway. If each link along the 
path uses orthogonal channels, then each link could be active simultaneously,
otherwise they will complete with each other. 
We note each node along the path had traffic demand $T_d$, obviously the bottleneck 
link along the path would be the one closest to the gateway, and then next. 
Thus, the total traffic along the path $h \cdot T_d$ must be less than the 
bottleneck link's capacity $\delta$ estimated from the measurements. In such a scenario, the $h$-hop mesh node 
would achieve the minimum served demand, which we define as the network efficiency. 
In general, the active time per link for an $h$-hop mesh node can be represented 
by $1,\frac{h-1}{h},\frac{h-2}{h}\cdots \frac{1}{h}$. The summation of all active 
times for each mesh node along the path is considered the intra-path network cost.

Considering only intra-path interference, using lower carrier frequencies allows a
reduction in hop count and increase in the network efficiency of each mesh node along
the $h$-hop path. However, a lower carrier frequency will induce greater interference
to other paths to the gateway (i.e., inter-path interference). 
Fig.~\ref{fig:interferencerange} depicts such an example where
links in different bands are represented by circles for 450 MHz, rectangles for
2.4 GHz, and triangles for the nodes which can choose between the two.
Nodes $A$ and $C$ could be connected through two 2.4-GHz links or a single 450-MHz link.
With 2.4 GHz, the interfering distance will be less than using 450 MHz. For example, only 
link $D,E$ will suffer from interference, whereas $H,I$ would not. However, with 450 MHz,
link $A,C$ would interfere with links $F,G$, $M,L$, and $K,J$. At each time unit, the number of
links interfering with the active links along a path would be the inter-path network cost.

When an $h$-hop flow is transmitted to a destination node, it prevents 
activity on a number of links in the same frequency via the protocol model. 
The active time on a single link can be noted as 
$\frac{T}{\gamma_h}$. 
An interfering link from the conflict matrix $F$ counts as $I_h$ per unit time
and contributes to the network cost in terms of:
$\frac{hT}{\gamma_1}\cdot I_1 + \frac{(h-1)T}{\gamma_2}\cdot I_2 \cdots \frac{T}{\gamma_h}\cdot I_h$.
Then, the traffic transmitted in a unit of network cost for the $h$-hop node is:
\begin{equation}
\label{eq:originpen}
E_{\eta}=\frac{T}{\sum_{i \in h}\frac{(h-i+1)\cdot T}{\gamma_i}\cdot I_i }
\end{equation}
Using network efficiency, the equation simplifies to:
\begin{equation}
\label{eq:pen}
E_{\eta}=\frac{\gamma}{\sum_{i \in h} (h-i+1)\cdot I_i}
\end{equation}

The network efficiency is the amount of traffic that could be 
offered on a path per unit time. With multiple channels from the same band,
$I_i$ will not change due to the common communication range. With multiple
bands, $I_i$ depends on the band choice due to the communication range diversity.  
This network efficiency jointly considers hop count and interference. We define
the Path Interference induced on the Network (PIN) as the denominator of Eq.~\ref{eq:pen},
which represents the sum of all interfering links in the network by a given path. 
PIN is used to quantify the current state of channel for channel assignment
across WiFi and white space bands.
To determine when the lower carrier frequency will be better than two or more hops at a
higher carrier frequency, we consider the average interference $\bar{I}$ of a given path
at the higher frequency.  The problem could be formulated as:
\begin{equation}
\label{eq:benefit}
\frac{\gamma}{\frac{h(h-1)}{2}\cdot \bar{I}+I_x} \geq \frac{\gamma}{\frac{h(h+1)}{2}\cdot \bar{I}}
\end{equation}

Here, from Eq.~ref{eq:benefit} when $I_x \leq 2\cdot h\bar{I}$, the performance of a lower-frequency link  
is better than two higher-frequency hops for the same destination node. $I_x$ is also a parameter of hop count 
in Eq.~\ref{eq:pen}. When the hop count is lower which closer to the gateway node, the threshold 
would be more strict since the interference would have a greater effect on the performance.




\subsection{Band-based Path Selection (BPS) Algorithm}
\label{subsec:BPS}

\begin{algorithm}[t]
    \small
\caption{Band-based Path Selection (BPS)}
\label{algorithms:bps}
\begin{algorithmic}[1]
\REQUIRE  ~~\\
	$M$: Set of mesh nodes\\
	$G$: Set of gateway nodes\\
	$C$: Communication graph of potential links among all nodes\\
	$I$: Interference matrix of all potential links \\
	$B$: Available frequency bands \\
	$\delta$: Measurements based Channel Capacity
\ENSURE ~~\\    
$CA$: Channel Assignment of the Network\\
\STATE Rank mesh nodes in Set $M$ according to physical distance from gateway nodes $G$
\STATE Initialize $S_{curr}=G$, $N_{srv}=\emptyset$, $N_{unsrv}=M$,$I_{active}=\emptyset$
\WHILE {$N_{srv}=!M$}
\STATE Select node with largest distance to gateway
\STATE Find the adjacency matrix across band combinations $A_c$
\FORALL{$A_{i}\in A_c$}
\STATE Find the shortest path $SP_i$ in mixed adjacency matrix A 
\FORALL{Link $l \in SP_i$, ordered from gateway to mesh node}
\STATE Find the least interfering path with measured $\delta \times E_n$
\STATE If equally-interfering links, choose higher frequency
\STATE Calculate the path interference of $SP_i$
\ENDFOR
\STATE Store the shortest path $SP_i$ as $SP$
\ENDFOR
\STATE Assign the path in the network\\
		\STATE Update $N_{srv},N_{unsrv}$
		\STATE Update $I_{active}$ from $I$
\ENDWHILE 

Output $CA$ as the locally-optimal solution\\
\end{algorithmic}
\end{algorithm}

We design a Band-based Path Selection (BPS) algorithm
(described in Alg.~\ref{algorithms:bps}) which first chooses the 
mesh node that has the largest physical distance from the gateway 
nodes to reduce the whole time cost of the network. When a path is constructed for
the mesh node with the greatest distance, all subsequent mesh nodes along
the path are also connected to the gateway. The intuition behind the
BPS algorithm is to improve the worst mesh node performance in a path.
In large-scale mesh networks, it is impractical to traverse all the paths with
different combination of bands from a mesh node to any gateway node since 
it is a NP-hard problem. However,
based on the discussion in Section~\ref{subsec:PEN}, if two paths have the same
number of used bands along those paths, then the path with the least hops
is likely to have the greatest performance and is chosen.  Similarly, if
two path have the same path interference, we choose the path which has
higher-frequency links for spatial reuse. Thus, the next step of the
algorithm is to find the shortest path across band combinations.

To run the algorithm, compared to the number of mesh nodes, the amount of channels $N_B$ in
different bands is small. The time complexity of calculating the combination
is $O(2^{N_B})$. Finding the shortest path in Dijkstra algorithm will
cost $O(N_E^2)$ according to~\cite{golden1976shortest}, where $N_E$ is the set of possible links in the
network, and as a result, the total complexity would be $O(N_E^2\cdot 2^{N_B})$.
The algorithm would then calculate the PIN of the candidate path and select the path
with the least interference channel induced on the network for the source mesh node.
After a path is assigned, the algorithm updates the network's channel assignment
with served nodes, activated links, and radio information. Then,
we iteratively assign channels for all the mesh nodes in the
network.

If all the nodes are connected to gateway nodes ($N_E={n \choose 2}$ which is $O(N_V^2)$), 
then the complexity of assigning a channel for a mesh node is $O(N_E^2\cdot2^{N_B})$. 
Then, the complexity of assigning a mesh node is $O(N_V^4\cdot2^{N_B})$.
To assign {\it all} the nodes in the network, the complexity would 
be $O(N_V^5\cdot2^{N_B})$. The complexity is polynomial time of
the number of traffic demands points (client group) for a wireless
network assignment.


