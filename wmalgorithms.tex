\section{Mixed Multiband Path Analysis}
\label{sec:wmalgorithms}


In this section, we discuss the influence of ~\emph{Multiband} on ~\emph{Multihop Path} in mesh network. 
Accodirng to these analysis, we develop two algorithms for ~\emph{Channel Assignment} in multi-band multi-radio scenario.

% PEN part 
% Talk about the network efficiency for multiband multihop mixed hop

In ~\emph{Multiband Multiradio Network}, 
a multihop path could have higher frequency band combination with less interference range or a set of lower frequency band with less hop count.
A key issue of multihop path in such network is to answer which combination is better.
We focus our work on ~\emph{Channel Assignment} dealing with more interference factors rather than routing protocol which would be more concern on delay. Other architecture also has such problem such as wireless sensor network.

To discuss this problem, we pick up a multihop path from mesh network and analyze its performance with worst case hypothesis. In mesh network, such a path would have a bottle neck in the link closet to gateway.
When a mesh network was built with gateway placement, constructor should considered load-aware demand of mesh nodes and mesh node population. 
Generally the nodes close to gateway should have more traffic demand and gateway itself should have the most connectivity population. 
We treat each node equally binding with fairness, otherwise mesh nodes close to gateway could be served more traffic and show a high goodput of the network.
For analyze, we assume all the node in the path equally share the time of the link next to a gateway. It is also the worst case for getting a larger goodput.


First, we introduce the ~\emph{Intra-Path} traffic. When we have a multihop path, in worst case all the nodes on the path have only one $h$ hop path arrived at a gateway node. The path is made of links from one node to another.
Each node has traffic $T$, nomatter uplink or downlink since both of them occupy link capacity in the same way. And the total traffic on the path $\sum T$ is less than the bottle neck link capacity $C$. 

We define the minimum transmission rate on a path as ~\emph{Network Efficiency}. 
With the fairness restriction, the last node in the path has the minimum transmission rate.
Then the acitve time in a time unit of each link can be represented as $1,\frac{h-1}{h},\frac{h-2}{h}\cdots \frac{1}{h}$. 
The unit time of each link in the path is counted as total cost time of network.
%\begin{equation}
%\label{eq:intrapath}
%\begin{split}
%E_{Intra-Path}=\frac{Path\ Active\ Time}{Network\ Time}\\
%E_{Intra-Path}=\frac{1}{2}+\frac{1}{2\cdot h}
%\end{split}
%\end{equation}


%As hop count increase, the ~\emph{Intra-Path} will decrease till the lower bound $\frac{1}{2}$. With routing protocol which is out of this work, the delay increase too.
Without considering ~\emph{Inter-Path} interference which represent interference with links out of the path, 
an intuition of using lower band is to reduce the hop count
 to increase the minimum time utility rate which is the active time of the last link over the total active time of the path. 
However, at the same time, the interference range increase too. An example shown in ~\ref{fig:networkefficiency}, 
the picture shows links in different bands, let's say 2.4GHz and 900MHz, as a sketch map, does not represent the real distance.
Node $A,C$ could be connected through two 2.4GHz links or a single 900MHz link; with 2.4GHz links, only link $D,E$ will be interferenced; however, with 900MHz $A,C$ link, link $F,G;M,L;K,J$ will be interferenced. 


\begin{figure}
%\vspace{-0.0in}
\centering
\includegraphics[width=74mm]{figures/networkefficiency}
\vspace{-0.1in}
\caption{Path Network Efficiency Introduction, Solid Wire notes 2.4GHz link, Dashed line notes 900MHz}
\label{fig:networkefficiency}
%\vspace{-0.0in}
a\end{figure}

To quantization this ~\emph{Inter-Path Interference}, 
the unit time of these links are counted as ~\emph{Network Time}. 
When a $h$ hop path transmitting traffic $T$ for the destination node, it stops activity on a number of links in the same band. 
In a multihop path, when the traffic arrived at the last destination node, all the previous links are serving for these traffic.
The active time on a single link can be noted as 
$\frac{T}{c_h}$. We keep in the worst case when the last node in the path got traffic $T$, the other node also be served traffic $T$.
With interference counts $I_h$ from the conflict matrix:
the ~\emph{Network Time} counted as 
$\frac{hT}{c_1}\cdot I_1 + \frac{(h-1)T}{c_2}\cdot I_2 \cdots \frac{T}{c_h}\cdot I_h$, the ~\emph{Path Efficiency over Network} is defined the traffic over the ~\emph{Network Time} and could be represented as:



\begin{equation}
\label{eq:originpen}
E_{PEN}=\frac{T}{\sum_{i \leq h}\frac{i\cdot T}{c_h}\cdot I_i }
\end{equation}

With protocol model, if link exist, then they have the same capacity $c_1=c_2 \cdots =c_h=c$. 
To avoid $0$ value in the denominator, we add a $1$ to adjust the denominator which does not change the parameter characteristics. 
The \emph{Path Efficiency over Network}could be represented as:


\begin{equation}
\label{eq:pen}
E_{PEN}=(\frac{c}{1+\sum_{i \leq h} i\cdot I_i}
\end{equation}
 

The meaning of the ~\emph{Network Efficiency} is that in a unit time, the traffic could be loaded by this path. In multichannel scenario, all the channel will have the same communication range, this parameter equals to the conflic graph in many multichannel works which try to minimize the interference~\cite{jain2005impact}. Since we count only one channel not all possible links, it also could be seen as an extention of a single link ~\emph{Link Load} defined in ~\cite{raniwala2004centralized}.

The ~\emph{Path Efficiency over Network} connect hop counts and interference. 
Then we discuss when a lower ~\emph{White Space Band} is better to be used in a path.
In a path, we use an average interference count $\bar{I}$ replace each interference count with assumption the links in the path all in one higher freq band. Then a ~\emph{White Space Band} is used to replace two links in the path as a single link with interference count $X$ represent one of the factor $i\cdot I_i$. The problem could be formulated as:

 
\begin{equation}
\label{eq:benefit}
\frac{c}{1+\frac{h(h-1)}{2}\cdot \bar{I}+X} \geq \frac{c}{1+\frac{h(h+1)}{2}\cdot \bar{I}}
\end{equation}

From the inequation, when $X \leq 2\cdot h\bar{I}$ a lower band could be better. $X$ is also a function of hop order in the path, generally the path order lower, the threshold would be more strict; otherwise it could be loose. It matches the intuition the hop order is small, it close to the gateway, it should be more crowd for this link.
It helps to tell the ranking of set of links and a path where we can start to resolve channel assignment problem.











The discussion in \~ref{subsec:PEN} provide the methodology to improve channel assignment. But the difficulty of channel assignment is that before the process has been done, it could no be evaluated to tell which is better.
To approach the solution, we propose two local search based heuristic algorithms to adapt the multiband scenario. 


\subsection{Growing Spanning Tree Algorithm}
In a mesh network, gateway nodes always building in the most busy location ~\cite{robinson2008adding, he2008optimizing}.
As the service tree rooted at a gateway grows, the links closer to the gateway, the more interference will happen.
And in the edge of the network, it is less populated in which cases reduce hop count through lower frequency may bring more benefit. 
The main idea behind the ~\emph{Growing Spanning Tree} Algorithm is 
to find the link has least interference on the network for each node in a greedy manner each step. The hop count for gateway nodes themselves are 0.
We first initialize the mesh nodes ranking with the distance to all the gateway nodes. In the ranking order of the mesh nodes, the 1 hop links from gateway nodes ranking with the ~\emph{Path Interference}. Then select the lowest interferred link for this node and update the assignment information for next steps. 
Iterate these steps to assign channels for all mesh nodes. This process is phase 1 of our algorithm which is similar to but not exactly the same as the breath first search channel assignment. 

In phase 2 of the algorithm, we sort the mesh nodes with their hop count to gateway nodes. 
The algorithm traverses all the nodes whose hop count are less than the current node. If there are radio slots for the less hop nodes, it is possible to re-connect the mesh node to reduce the hop count. We rank all possible option with their path interference, then choose the lowest one re-connect the mesh node. If there exist new link has the same path interference, we count the number of nodes has connected to the gateway nodes, select the gateway has less node connected. Phase 2 process will be interated till no changes in the network.

The ~\emph{Growing Spanning Tree} Algorithm is described in ~\ref{algorithms:gsp}.


% Need to talk about how to improve the bottle neck links,
%FIXME talk about BFS-CA 





\begin{algorithm}
    \small
\caption{Multiband Growing Spanning Tree Algorithm}
\label{algorithms:gsp}
\begin{algorithmic}[1]
\REQUIRE  ~~\\
	 $M$: The set of all mesh nodes\\
	 $G$: The set of gateway nodes\\
	 $C$: Communication graph of potential links among all nodes\\
	 $I$: Interference matrix of all potential links \\
	 $B$: Available frequency bands
\ENSURE ~~\\    
$CA$: Channel Assignment of the Network\\
\STATE Initial $S_{current}=G$, $N_{served}=\emptyset$, $N_{unserved}=M$,$I_{active}=\emptyset$
\STATE Rank mesh nodes according to their distance to gateway nodes
\WHILE {$N_{served}=M$}
\FORALL {$s \in S_{current}$}
	\STATE Find one-hop nodes in $S_{Next}$
	\STATE Sort $S_Next$ according to distance from gateway nodes, shorter distance first
	\FORALL {$l \in S_{Next}$}
		\STATE Calculate one-hop path interference of link $s\rightarrow l$
		\STATE Sort the links, choose the one has the least path interference
		\STATE Assign(s,l) with the least interference link
		\STATE Update $N_{served},N_{unserved}$
		\STATE Update $I_{active}$ from $I$
	\ENDFOR
	\STATE $S_{current}=S_{Next}$
\ENDFOR
\ENDWHILE
% Round 2, resolve bottle neck problem
% After above, all node has its shortest path, but exist bottle neck, at least have a connection
% If could add band, add band to the high occupacied channel
\STATE Sort mesh nodes with their hop counts to gateway nodes $N_{sorted}$
\WHILE {Change of Channel Assignment Exist} 
\FORALL {$s \in N_{sorted}$}
	\STATE Tranverse all the 1 hop arrived nodes have less hop count than node $s$ 
	\STATE Check if these nodes have radio slots for node $s$
	\STATE Sort path through possible nodes with the path interference
	\STATE Choose a new path if it has less interference than the previous one
	\STATE If more than one path has the same interference, choose the gateway node has least leaves nodes 
\ENDFOR
\ENDWHILE

Output $Channel Assignment$ as Solution
\end{algorithmic}
\end{algorithm}
      

%FIXME need more explaination of the tree growing algorithm
%The algorithm use the average $\bar{I}$ and average hop count to approach the channel assignment
%In ~\cite{robinson2008adding}, Robinson talked about the bottle neck of a network is the links neighbor to the gateway nodes.





\subsection{Best Path Selection Algorithm}
\label{subsec:step}

Based on the previous path efficiency analyze, the network efficiency is related to each link's interference and the distance to gateway nodes. To find a path for each mesh node, which could be converge to a shortest weight path detect
We define weight for each link and improve Dijikstra's algorithm with ~\emph{PEN} weight to find the best path for each mesh node ~\cite{golden1976shortest}.
To run Dijikstra's algorithm, we define two parameter of each link between two nodes. First is the existing interference of the link $I_w$, we mark the interference of bands as multiple links. 
The second is the ~\emph{Load Weight} of a link $l_w$, which is the number of path chosen this link. In Dijikstra's algorithm, the weight is calculated as denominator of ~\emph{PEN}, since the numerator is the same among different bands.
This parameter used to adjust the ~\emph{PEN} with bottle neck links.
The weight of Dijikstra's algorithm is related to hop order $h_i$ according to the definition of ~\emph{PEN}, the weight is calculated as $I_w\times h_i \times \l_w$.
We iterately find the best path of each node and update the parameters in the graph.





\begin{algorithm}
    \small
\caption{Sink To End Path Algorithm}
\label{algorithms:step}
\begin{algorithmic}[1]
\REQUIRE  ~~\\
	$M$: The set of all mesh nodes\\
	$G$: The set of gateway nodes\\
	$C$: Communication graph of potential links among all nodes\\
	$I$: Interference matrix of all potential links \\
	$B$: Bands amount
\ENSURE ~~\\    
$CA$: Channel Assignment from Gateway nodes to Mesh nodes\\
\WHILE {$notAllnodesVisited(M)$}
\STATE Intialize $CA,I_w,l_w$\\
		\STATE Run Dijistra's Algorithm with $C,I,B$ to all the Gateways
		\STATE Compare the $PEN$ to all Gateway node
		\STATE Choose the best one adding to $CA$
		\STATE Update $I_w,l_w$
\STATE Calculate $\triangle cost$ for all valid operations\\
\STATE Apply $swap$ with largest positive $\triangle cost$
\ENDWHILE 

Output $CA$ as locally optimal solution\\
\end{algorithmic}
\end{algorithm}


