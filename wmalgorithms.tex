\section{Path Analysis with Diverse Propagation}
\label{sec:wmalgorithms}


In this section, we discuss the influence of diverse propagation
characteristics of the wide range of carrier frequencies introduced
by white space and WiFi bands. We then introduce two heuristic
algorithms for channel assignment in WhiteMesh networks.
%According to the analysis, we develop two algorithms for ~\emph{Channel Assignment} in multi-band multi-radio scenario.

% PEN part 
\subsection{Path Interference Induced on the Network}
\label{subsec:PEN}

In WhiteMesh networks, multihop paths can be intermixed with WiFi 
for more spacial reuse and white space bands with less hops.  
To deal with the trade-off, we consider
analyze the band choices reduce the number of hops along a path and the 
aggregate level of interference that hop-by-hop path choices have
on the network (i.e., Path Interference induced on the Network).

Mesh nodes closer to the gateway generally achieve
greater levels of throughput at sufficiently high offered loads. 
To combat such starvation effects, we treat each flow with equal priority in the network when
assigning channels. In the worst case, all nodes along a particular path have equal 
time shares for contending links (i.e., intra-path interference).
We start the channel assignment assuming that $h$ mesh nodes are demanding
traffic from each hop of an $h$-hop path to the gateway. If each link along the 
path uses orthogonal channels, then each link could be active simultaneously,
otherwise they will complete with each other. 
We note each node along the path had traffic demand $T_d$, obviously the bottleneck 
link along the path would be the one closest to the gateway, and then next. 
Thus, the total traffic along the path $h \cdot T_d$ must be less than the 
bottleneck link's capacity $\delta$ estimated from the measurements. In such a scenario, the $h$-hop mesh node 
would achieve the minimum served demand, which we define as the network efficiency. 
In general, the active time per link for an $h$-hop mesh node can be represented 
by $1,\frac{h-1}{h},\frac{h-2}{h}\cdots \frac{1}{h}$. The summation of all active 
times for each mesh node along the path is considered the intra-path network cost.

Considering only intra-path interference, using lower carrier frequencies allows a
reduction in hop count and increase in the network efficiency of each mesh node along
the $h$-hop path. However, a lower carrier frequency will induce greater interference
to other paths to the gateway (i.e., inter-path interference). 
Fig.~\ref{fig:interferencerange} depicts such an example where
links in different bands are represented by circles for 450 MHz, rectangles for
2.4 GHz, and triangles for the nodes which can choose between the two.
Nodes $A$ and $C$ could be connected through two 2.4-GHz links or a single 450-MHz link.
With 2.4 GHz, the interfering distance will be less than using 450 MHz. For example, only 
link $D,E$ will suffer from interference, whereas $H,I$ would not. However, with 450 MHz,
link $A,C$ would interfere with links $F,G$, $M,L$, and $K,J$. At each time unit, the number of
links interfering with the active links along a path would be the inter-path network cost.

When an $h$-hop flow is transmitted to a destination node, it prevents 
activity on a number of links in the same frequency via the protocol model. 
The active time on a single link can be noted as 
$\frac{T}{\gamma_h}$. 
An interfering link from the conflict matrix $F$ counts as $I_h$ per unit time
and contributes to the network cost in terms of:
$\frac{hT}{\gamma_1}\cdot I_1 + \frac{(h-1)T}{\gamma_2}\cdot I_2 \cdots \frac{T}{\gamma_h}\cdot I_h$.
Then, the traffic transmitted in a unit of network cost for the $h$-hop node is:
\begin{equation}
\label{eq:originpen}
E_{\eta}=\frac{T}{\sum_{i \in h}\frac{(h-i+1)\cdot T}{\gamma_i}\cdot I_i }
\end{equation}
Using network efficiency, the equation simplifies to:
\begin{equation}
\label{eq:pen}
E_{\eta}=\frac{\gamma}{\sum_{i \in h} (h-i+1)\cdot I_i}
\end{equation}

The network efficiency is the amount of traffic that could be 
offered on a path per unit time. With multiple channels from the same band,
$I_i$ will not change due to the common communication range. With multiple
bands, $I_i$ depends on the band choice due to the communication range diversity.  
This network efficiency jointly considers hop count and interference. We define
the Path Interference induced on the Network (PIN) as the denominator of Eq.~\ref{eq:pen},
which represents the sum of all interfering links in the network by a given path. 
PIN is used to quantify the current state of channel for channel assignment
across WiFi and white space bands.
To determine when the lower carrier frequency will be better than two or more hops at a
higher carrier frequency, we consider the average interference $\bar{I}$ of a given path
at the higher frequency.  The problem could be formulated as:
\begin{equation}
\label{eq:benefit}
\frac{\gamma}{\frac{h(h-1)}{2}\cdot \bar{I}+I_x} \geq \frac{\gamma}{\frac{h(h+1)}{2}\cdot \bar{I}}
\end{equation}

Here, from Eq.~ref{eq:benefit} when $I_x \leq 2\cdot h\bar{I}$, the performance of a lower-frequency link  
is better than two higher-frequency hops for the same destination node. $I_x$ is also a parameter of hop count 
in Eq.~\ref{eq:pen}. When the hop count is lower which closer to the gateway node, the threshold 
would be more strict since the interference would have a greater effect on the performance.




%The discussion in subsection ~\ref{subsec:PEN} provide the methodology to balance hop counts and low frequency long distance links in channel assignment. But the difficulty of channel assignment is that before the process has been done, it could no be evaluated to tell which is better.
%To approach the solution, we propose two local search based heuristic algorithms to adapt the multiband scenario. 

\subsection{Growing Spanning Tree (GST) Algorithm}

In a mesh network, gateway nodes tend to be located at the points
of most dense demand~\cite{robinson2008adding, he2008optimizing}.
In the mesh topology, the closer a mesh node is to the gateway, 
the more interference it will likely have due to higher demand.
Conversely, edges of the network tend to have more sparse demand,
resulting in less interference. Based on this intuition, 
the Growing Spanning Tree (GST) algorithm (described in Alg.~~\ref{algorithms:gst}) 
assigns channels to have the least resulting interference on the network (PIN) in a 
greedy manner. To do so, we first initialize the mesh-node ranking 
with respect to the physical distance to all gateway nodes.
We then consider the one-hop nodes from the gateways (based upon
if any carrier frequency of the available bands $B$ is in 
communication range of the gateway) with least Path Interference
induced on the Network (PIN) for these available band. This 
least-interfering, one-hop node is chosen for channel assignment,
and the network is updated for the next step. We term this Phase~1
of the GST, and it resembles the Breadth First Search Channel
Assignment (BFS-CA)~\cite{ramachandran2006interference}.

In Phase~2 of the GST algorithm, we sort the mesh nodes according
to their hop count from the gateway nodes.  The algorithm then
traverses all the nodes whose hop count are less than the current node. 
If there are available radio slots for the mesh nodes of lower hop
count from the gateway, it is possible to reassign the mesh node 
to reduce the hop count.  We rank all possible options with their PIN.
We then choose the lowest one for reassignment of the mesh node. If 
there exists new links has the same PIN to two or more gateways, we 
consider the total number of nodes connected to each gateway, selecting
the gateway that has fewer connected mesh nodes. Phase~2 process will 
iterate until no changes in channel assignment occur or up to the total 
number of mesh nodes.

% Need to talk about how to improve the bottle neck links,
%FIXME talk about BFS-CA 

\begin{algorithm}
    \small
\caption{Growing Spanning Tree (GST)}
\label{algorithms:gst}
\begin{algorithmic}[1]
\REQUIRE  ~~\\
	 $M$: The set of mesh nodes\\
	 $G$: The set of gateway nodes\\
	 $C$: Communication graph of potential links among all nodes\\
	 $I$: Interference matrix of all potential links \\
	 $B$: Available frequency bands
\ENSURE ~~\\    
$CA$: Channel Assignment of the Network\\
\STATE Initialize $S_{current}=G$, $N_{served}=\emptyset$, $N_{unserved}=M$,$I_{active}=\emptyset$
\STATE Rank mesh nodes according to physical distance from gateway nodes
\WHILE {$N_{served}=!M$}
\FORALL {$s \in S_{current}$}
	\STATE Find one-hop nodes in $S_{Next}$
	\STATE Sort $S_Next$ according to distance from gateway nodes
	\FORALL {$l \in S_{Next}$}
		\STATE Calculate 1-hop path interference of link $s\rightarrow l$
		\STATE Sort the links according to path interference
		\STATE Assign(s,l) with the least interference link
		\STATE Update $N_{served},N_{unserved}$
		\STATE Update $I_{active}$ from $I$
	\ENDFOR
	\STATE $S_{current}=S_{Next}$
\ENDFOR
\ENDWHILE
\STATE Sort mesh nodes with their hop counts to gateway nodes $N_{sorted}$
\WHILE {Change of Channel Assignment Exists} 
\FORALL {$s \in N_{sorted}$}
	\STATE Traverse all 1-hop arrived nodes have less hop count than node $s$ 
	\STATE Check if these nodes have radio slots for node $s$
	\STATE Sort path through possible nodes with the path interference
	\STATE Choose a new path if it has less interference than the previous one
	\STATE If more than one path has the same interference, choose least-leaved gateway node
\ENDFOR
\ENDWHILE

Output $Channel Assignment$ as Solution
\end{algorithmic}
\end{algorithm}
      
% Talk a little bit about the tree growing and continue to the best path
The GST algorithm greedily assigns a single link to the network (Phase~1) 
and balances the gateway load in the adjustment process (Phase~2). 
The breadth first search from Phase~1 for a multiband network has a complexity 
of $O((N_B \cdot N_V)^2)$, where $N_V$ is the number of nodes $V$, $N_B$ is the number of bands, 
sorting of nodes would cost $O(N_B \cdot N_V log(N_B \cdot N_V))$. 
Hence assigning a node takes $O((N_B \cdot N_V)^2)$ time. When there are $N_V$ nodes, the complexity of an adjust interation is $O(N_B^2 \cdot N_V^3)$.
The total interation would be less than $N_V$ since we put an upbound their and in our simulation it does not touch even $\frac{N_V}{2}$. So the complexity of the method would be $O(N_B^2 \cdot N_V^4)$.

\subsection{Band-based Path Selection (BPS) Algorithm}
\label{subsec:step}

The GST algorithm starts from the gateway nodes to generate the channel assignment, in contrast, ~\emph{Band-based Path Selection} Algorithm starts from the mesh node who has the largest distance from the gateway nodes.
When a path is select for such a node, the relay nodes on the path are served. 
The main idea behind the ~\emph{Band-based Path Selection} Algorithm is to improve the worst mesh node performance in a path. 

The algorithm first sort the mesh nodes in order of their distance to any gateway nodes. Then we select the mesh node has the furthest distance to gateway nodes. In the network, it is impossible traverse all the path with different combination of bands from a mesh node to any gateway nodes. Based on the analysis in ~\ref{subsec:PEN}, if paths has the same bands combinations, a shortest path most of the time could have the best performance.
In the same path under a bands combination, we will choose the link in a channel has the least interference.
 In case two path has the same path interference, we choose the path who has more high frequency links for spacing re-use.
Thus, the next step of the algorithm is to find the shortest path in different bands combinations. Comparing to the number of mesh nodes, the amount of channels $N_B$ in different bands is small. The time complexity of calculation the combination is $O(2^{N_B})$. 
Then finding the shortest path in Dijkstra algorithm will cost $O(N_E^2)$ ~\cite{golden1976shortest}, $N_E$ is the links in the network. So the total would be $O(N_E^2\cdot 2^{N_B})$.
Then the algorithm calculate PIN of the candidate path and select the path bringing the least interference to the network for the starting mesh node.

After a path is assigned, the algorithm update the network assignment with served nodes, activated links, and nodes' radio information. Then we assign the next node till all the mesh nodes are connected in the network.
The ~\emph{Band-based Path Selection} Algorithm is described in ~\ref{algorithms:bps}.

\begin{algorithm}
    \small
\caption{Band-based Path Selection (BPS)}
\label{algorithms:bps}
\begin{algorithmic}[1]
\REQUIRE  ~~\\
	$M$: The set of mesh nodes\\
	$G$: The set of gateway nodes\\
	$C$: Communication graph of potential links among all nodes\\
	$I$: Interference matrix of all potential links \\
	$B$: Available frequency bands
\ENSURE ~~\\    
$CA$: Channel Assignment of the Network\\
\STATE Rank mesh nodes according to physical distance from gateway nodes
\STATE Initialize $S_{current}=G$, $N_{served}=\emptyset$, $N_{unserved}=M$,$I_{active}=\emptyset$
\WHILE {$N_{served}=!M$}
\STATE Select node with largest distance to gateway nodes
\STATE Find the Adjacency Matrix in different band combinations $A_c$
\FORALL{$A_{i}\in A_c$}
\STATE Find the shortest path $SP_i$ in the mixed adjacency matrix A 
\FORALL{Link $l \in SP_i$ in order from gateway node to mesh node}
\STATE Find the link that has less interference
\STATE If there are links have the same interference, choose higher frequency
\STATE Calculate the path interference of path $SP_i$
\ENDFOR
\STATE Store the shortest path $SP_i$ as $SP$
\ENDFOR
\STATE Assign the path in the Network\\
		\STATE Update $N_{served},N_{unserved}$
		\STATE Update $I_{active}$ from $I$
\ENDWHILE 

Output $CA$ as locally-optimal solution\\
\end{algorithmic}
\end{algorithm}

The complexity of the assign a node would be $O(N_E^2\cdot2^{N_B})$, if all the nodes could be connected, $N_E=C_n^2$ which is $O(N_V^2)$.
Then the complexity of assigning a node could be marked as $O(N_V^4\cdot2^{N_B})$.
 To assign all the node in the network, the complexity would be $O(N_V^5\cdot2^{N_B})$.


