\section{Path Analysis with Diverse Propagation}
\label{sec:wmalgorithms}


In this section, we discuss the influence of diverse propagation
characteristics of the wide range of carrier frequencies introduced
by white space and WiFi bands. We then introduce two heuristic
algorithms for channel assignment in WhiteMesh networks.
%According to the analysis, we develop two algorithms for ~\emph{Channel Assignment} in multi-band multi-radio scenario.

% PEN part 
% Talk about the network efficiency for multiband multihop mixed hop

In ~\emph{Multiband Multiradio Network}, 
a multihop path could have higher frequency band combination with less interference range or a set of lower frequency band with less hop count.
A key issue of multihop path in such network is to answer which combination is better.
We focus our work on ~\emph{Channel Assignment} dealing with more interference factors rather than routing protocol which would be more concern on delay. Other architecture also has such problem such as wireless sensor network.

To discuss this problem, we pick up a multihop path from mesh network and analyze its performance with worst case hypothesis. In mesh network, such a path would have a bottle neck in the link closet to gateway.
When a mesh network was built with gateway placement, constructor should considered load-aware demand of mesh nodes and mesh node population. 
Generally the nodes close to gateway should have more traffic demand and gateway itself should have the most connectivity population. 
We treat each node equally binding with fairness, otherwise mesh nodes close to gateway could be served more traffic and show a high goodput of the network.
For analyze, we assume all the node in the path equally share the time of the link next to a gateway. It is also the worst case for getting a larger goodput.


First, we introduce the ~\emph{Intra-Path} traffic. When we have a multihop path, in worst case all the nodes on the path have only one $h$ hop path arrived at a gateway node. The path is made of links from one node to another.
Each node has traffic $T$, nomatter uplink or downlink since both of them occupy link capacity in the same way. And the total traffic on the path $\sum T$ is less than the bottle neck link capacity $C$. 

We define the minimum transmission rate on a path as ~\emph{Network Efficiency}. 
With the fairness restriction, the last node in the path has the minimum transmission rate.
Then the acitve time in a time unit of each link can be represented as $1,\frac{h-1}{h},\frac{h-2}{h}\cdots \frac{1}{h}$. 
The unit time of each link in the path is counted as total cost time of network.
%\begin{equation}
%\label{eq:intrapath}
%\begin{split}
%E_{Intra-Path}=\frac{Path\ Active\ Time}{Network\ Time}\\
%E_{Intra-Path}=\frac{1}{2}+\frac{1}{2\cdot h}
%\end{split}
%\end{equation}


%As hop count increase, the ~\emph{Intra-Path} will decrease till the lower bound $\frac{1}{2}$. With routing protocol which is out of this work, the delay increase too.
Without considering ~\emph{Inter-Path} interference which represent interference with links out of the path, 
an intuition of using lower band is to reduce the hop count
 to increase the minimum time utility rate which is the active time of the last link over the total active time of the path. 
However, at the same time, the interference range increase too. An example shown in ~\ref{fig:networkefficiency}, 
the picture shows links in different bands, let's say 2.4GHz and 900MHz, as a sketch map, does not represent the real distance.
Node $A,C$ could be connected through two 2.4GHz links or a single 900MHz link; with 2.4GHz links, only link $D,E$ will be interferenced; however, with 900MHz $A,C$ link, link $F,G;M,L;K,J$ will be interferenced. 


\begin{figure}
%\vspace{-0.0in}
\centering
\includegraphics[width=74mm]{figures/networkefficiency}
\vspace{-0.1in}
\caption{Path Network Efficiency Introduction, Solid Wire notes 2.4GHz link, Dashed line notes 900MHz}
\label{fig:networkefficiency}
%\vspace{-0.0in}
a\end{figure}

To quantization this ~\emph{Inter-Path Interference}, 
the unit time of these links are counted as ~\emph{Network Time}. 
When a $h$ hop path transmitting traffic $T$ for the destination node, it stops activity on a number of links in the same band. 
In a multihop path, when the traffic arrived at the last destination node, all the previous links are serving for these traffic.
The active time on a single link can be noted as 
$\frac{T}{c_h}$. We keep in the worst case when the last node in the path got traffic $T$, the other node also be served traffic $T$.
With interference counts $I_h$ from the conflict matrix:
the ~\emph{Network Time} counted as 
$\frac{hT}{c_1}\cdot I_1 + \frac{(h-1)T}{c_2}\cdot I_2 \cdots \frac{T}{c_h}\cdot I_h$, the ~\emph{Path Efficiency over Network} is defined the traffic over the ~\emph{Network Time} and could be represented as:



\begin{equation}
\label{eq:originpen}
E_{PEN}=\frac{T}{\sum_{i \leq h}\frac{i\cdot T}{c_h}\cdot I_i }
\end{equation}

With protocol model, if link exist, then they have the same capacity $c_1=c_2 \cdots =c_h=c$. 
To avoid $0$ value in the denominator, we add a $1$ to adjust the denominator which does not change the parameter characteristics. 
The \emph{Path Efficiency over Network}could be represented as:


\begin{equation}
\label{eq:pen}
E_{PEN}=(\frac{c}{1+\sum_{i \leq h} i\cdot I_i}
\end{equation}
 

The meaning of the ~\emph{Network Efficiency} is that in a unit time, the traffic could be loaded by this path. In multichannel scenario, all the channel will have the same communication range, this parameter equals to the conflic graph in many multichannel works which try to minimize the interference~\cite{jain2005impact}. Since we count only one channel not all possible links, it also could be seen as an extention of a single link ~\emph{Link Load} defined in ~\cite{raniwala2004centralized}.

The ~\emph{Path Efficiency over Network} connect hop counts and interference. 
Then we discuss when a lower ~\emph{White Space Band} is better to be used in a path.
In a path, we use an average interference count $\bar{I}$ replace each interference count with assumption the links in the path all in one higher freq band. Then a ~\emph{White Space Band} is used to replace two links in the path as a single link with interference count $X$ represent one of the factor $i\cdot I_i$. The problem could be formulated as:

 
\begin{equation}
\label{eq:benefit}
\frac{c}{1+\frac{h(h-1)}{2}\cdot \bar{I}+X} \geq \frac{c}{1+\frac{h(h+1)}{2}\cdot \bar{I}}
\end{equation}

From the inequation, when $X \leq 2\cdot h\bar{I}$ a lower band could be better. $X$ is also a function of hop order in the path, generally the path order lower, the threshold would be more strict; otherwise it could be loose. It matches the intuition the hop order is small, it close to the gateway, it should be more crowd for this link.
It helps to tell the ranking of set of links and a path where we can start to resolve channel assignment problem.










%The discussion in subsection ~\ref{subsec:PEN} provide the methodology to balance hop counts and low frequency long distance links in channel assignment. But the difficulty of channel assignment is that before the process has been done, it could no be evaluated to tell which is better.
%To approach the solution, we propose two local search based heuristic algorithms to adapt the multiband scenario. 

\subsection{Growing Spanning Tree (GST) Algorithm}

In a mesh network, gateway nodes tend to be located at the points
of most dense demand~\cite{robinson2008adding, he2008optimizing}.
In the mesh topology, the closer a mesh node is to the gateway, 
the more interference it will likely have due to higher demand.
Conversely, edges of the network tend to have more sparse demand,
resulting in less interference. Based on this intuition, 
the Growing Spanning Tree (GST) algorithm (described in Alg.~~\ref{algorithms:gst}) 
assigns channels to have the least resulting interference on the network (PIN) in a 
greedy manner. To do so, we first initialize the mesh-node ranking 
with respect to the physical distance to all gateway nodes.
We then consider the one-hop nodes from the gateways (based upon
if any carrier frequency of the available bands $B$ is in 
communication range of the gateway) with least Path Interference
induced on the Network (PIN) for these available band. This 
least-interfering, one-hop node is chosen for channel assignment,
and the network is updated for the next step. We term this Phase~1
of the GST, and it resembles the Breadth First Search Channel
Assignment (BFS-CA)~\cite{ramachandran2006interference}.

In Phase~2 of the GST algorithm, we sort the mesh nodes according
to their hop count from the gateway nodes.  The algorithm then
traverses all the nodes whose hop count are less than the current node. 
If there are available radio slots for the mesh nodes of lower hop
count from the gateway, it is possible to reassign the mesh node 
to reduce the hop count.  We rank all possible options with their PIN.
We then choose the lowest one for reassignment of the mesh node. If 
there exists new links has the same PIN to two or more gateways, we 
consider the total number of nodes connected to each gateway, selecting
the gateway that has fewer connected mesh nodes. Phase~2 process will 
iterate until no changes in channel assignment occur or up to the total 
number of mesh nodes.

% Need to talk about how to improve the bottle neck links,
%FIXME talk about BFS-CA 

\begin{algorithm}
    \small
\caption{Growing Spanning Tree (GST)}
\label{algorithms:gst}
\begin{algorithmic}[1]
\REQUIRE  ~~\\
	 $M$: The set of mesh nodes\\
	 $G$: The set of gateway nodes\\
	 $C$: Communication graph of potential links among all nodes\\
	 $I$: Interference matrix of all potential links \\
	 $B$: Available frequency bands
\ENSURE ~~\\    
$CA$: Channel Assignment of the Network\\
\STATE Initialize $S_{current}=G$, $N_{served}=\emptyset$, $N_{unserved}=M$,$I_{active}=\emptyset$
\STATE Rank mesh nodes according to physical distance from gateway nodes
\WHILE {$N_{served}=!M$}
\FORALL {$s \in S_{current}$}
	\STATE Find one-hop nodes in $S_{Next}$
	\STATE Sort $S_Next$ according to distance from gateway nodes
	\FORALL {$l \in S_{Next}$}
		\STATE Calculate 1-hop path interference of link $s\rightarrow l$
		\STATE Sort the links according to path interference
		\STATE Assign(s,l) with the least interference link
		\STATE Update $N_{served},N_{unserved}$
		\STATE Update $I_{active}$ from $I$
	\ENDFOR
	\STATE $S_{current}=S_{Next}$
\ENDFOR
\ENDWHILE
\STATE Sort mesh nodes with their hop counts to gateway nodes $N_{sorted}$
\WHILE {Change of Channel Assignment Exists} 
\FORALL {$s \in N_{sorted}$}
	\STATE Traverse all 1-hop arrived nodes have less hop count than node $s$ 
	\STATE Check if these nodes have radio slots for node $s$
	\STATE Sort path through possible nodes with the path interference
	\STATE Choose a new path if it has less interference than the previous one
	\STATE If more than one path has the same interference, choose least-leaved gateway node
\ENDFOR
\ENDWHILE

Output $Channel Assignment$ as Solution
\end{algorithmic}
\end{algorithm}
      
% Talk a little bit about the tree growing and continue to the best path
The GST algorithm greedily assigns a single link to the network (Phase~1) 
and balances the gateway load in the adjustment process (Phase~2). 
The breadth first search from Phase~1 for a multiband network has a complexity 
of $O((N_B \cdot N_V)^2)$, where $N_V$ is the number of nodes $V$, $N_B$ is the number of bands, 
sorting of nodes would cost $O(N_B \cdot N_V log(N_B \cdot N_V))$. 
Hence assigning a node takes $O((N_B \cdot N_V)^2)$ time. When there are $N_V$ nodes, the complexity of an adjust interation is $O(N_B^2 \cdot N_V^3)$.
The total interation would be less than $N_V$ since we put an upbound their and in our simulation it does not touch even $\frac{N_V}{2}$. So the complexity of the method would be $O(N_B^2 \cdot N_V^4)$.

\subsection{Band-based Path Selection (BPS) Algorithm}
\label{subsec:step}

The GST algorithm starts from the gateway nodes to generate the channel assignment, in contrast, ~\emph{Band-based Path Selection} Algorithm starts from the mesh node who has the largest distance from the gateway nodes.
When a path is select for such a node, the relay nodes on the path are served. 
The main idea behind the ~\emph{Band-based Path Selection} Algorithm is to improve the worst mesh node performance in a path. 

The algorithm first sort the mesh nodes in order of their distance to any gateway nodes. Then we select the mesh node has the furthest distance to gateway nodes. In the network, it is impossible traverse all the path with different combination of bands from a mesh node to any gateway nodes. Based on the analysis in ~\ref{subsec:PEN}, if paths has the same bands combinations, a shortest path most of the time could have the best performance.
In the same path under a bands combination, we will choose the link in a channel has the least interference.
 In case two path has the same path interference, we choose the path who has more high frequency links for spacing re-use.
Thus, the next step of the algorithm is to find the shortest path in different bands combinations. Comparing to the number of mesh nodes, the amount of channels $N_B$ in different bands is small. The time complexity of calculation the combination is $O(2^{N_B})$. 
Then finding the shortest path in Dijkstra algorithm will cost $O(N_E^2)$ ~\cite{golden1976shortest}, $N_E$ is the links in the network. So the total would be $O(N_E^2\cdot 2^{N_B})$.
Then the algorithm calculate PIN of the candidate path and select the path bringing the least interference to the network for the starting mesh node.

After a path is assigned, the algorithm update the network assignment with served nodes, activated links, and nodes' radio information. Then we assign the next node till all the mesh nodes are connected in the network.
The ~\emph{Band-based Path Selection} Algorithm is described in ~\ref{algorithms:bps}.

\begin{algorithm}
    \small
\caption{Band-based Path Selection (BPS)}
\label{algorithms:bps}
\begin{algorithmic}[1]
\REQUIRE  ~~\\
	$M$: The set of mesh nodes\\
	$G$: The set of gateway nodes\\
	$C$: Communication graph of potential links among all nodes\\
	$I$: Interference matrix of all potential links \\
	$B$: Available frequency bands
\ENSURE ~~\\    
$CA$: Channel Assignment of the Network\\
\STATE Rank mesh nodes according to physical distance from gateway nodes
\STATE Initialize $S_{current}=G$, $N_{served}=\emptyset$, $N_{unserved}=M$,$I_{active}=\emptyset$
\WHILE {$N_{served}=!M$}
\STATE Select node with largest distance to gateway nodes
\STATE Find the Adjacency Matrix in different band combinations $A_c$
\FORALL{$A_{i}\in A_c$}
\STATE Find the shortest path $SP_i$ in the mixed adjacency matrix A 
\FORALL{Link $l \in SP_i$ in order from gateway node to mesh node}
\STATE Find the link that has less interference
\STATE If there are links have the same interference, choose higher frequency
\STATE Calculate the path interference of path $SP_i$
\ENDFOR
\STATE Store the shortest path $SP_i$ as $SP$
\ENDFOR
\STATE Assign the path in the Network\\
		\STATE Update $N_{served},N_{unserved}$
		\STATE Update $I_{active}$ from $I$
\ENDWHILE 

Output $CA$ as locally-optimal solution\\
\end{algorithmic}
\end{algorithm}

The complexity of the assign a node would be $O(N_E^2\cdot2^{N_B})$, if all the nodes could be connected, $N_E=C_n^2$ which is $O(N_V^2)$.
Then the complexity of assigning a node could be marked as $O(N_V^4\cdot2^{N_B})$.
 To assign all the node in the network, the complexity would be $O(N_V^5\cdot2^{N_B})$.


