\section{Problem Formulation}
\label{sec:problemformulation}

% Organization of the Sec
In this section, we formulate the problem of how to optimally 
use WiFi and white space bands in concert when deploying wireless 
mesh networks.  We first introduce the multiband mesh network 
system model and illustrate the challenges of such a WhiteMesh 
architecture in propagation and channel occupancy.
We then discuss the network performance in economic view 
to evaluate WhiteMesh networks and the corresponding goal of
both the optimization framework and the heuristic algorithm that 
we propose in the following section.  Finally, we present 
our integer linear programming model used to address the problem. 
 
\subsection{WhiteMesh Network Architecture}
\label{subsec:architecture}

Wireless propagation is the behavior of the signal loss characteristics 
when wireless signals are transmitted from a transmitter to receiver.
The strength of the receiving signal depends on both the line-of-sight
path (or lack thereof) and multiple other paths that are a result of 
reflection, diffraction, and scattering from obstacles in the 
environment~\cite{andersen1995propagation}. The widely-used Friis
equation characterizes the power of the received signal $P_r$ in terms 
of the power $P_t$ and gain $G_t$ of the transmitting signal, gain of 
the receiver $G_r$, wavelength $\lambda$ of the carrier frequency, 
distance $R$ from transmitter to receiver, and path loss exponent $n$ according 
to~\cite{friis}:
\begin{equation}
\label{eq:friis}
P_r=P_t+G_t+G_r+10n \log_{10}\left( \frac{\lambda}{4\pi R}\right)
\end{equation}
Here, the path loss exponent $n$ changes according to the
aforementioned environmental factors and ranges from 2 to 5 in typical
outdoor settings~\cite{rappaport}.

% Explain multiband vs multichannel Adding channel occupancy here, and also activity level
A frequency band is commonly defined as a group of channels which have
similar propagation characteristics with small frequency separation.
A common assumption in previous works that use many WiFi channels 
is that the propagation characteristics of one channel is similar 
to another, since the channel separation is relatively small 
(e.g., 5 MHz for the 2.4 GHz band).
Many works which rely on such an assumption have focused on the 
allocation of multiple WiFi channels with multiple radios in 
multihop wireless networks with channels in one band~\cite{doraghinejad2014channel}.
However, as FCC licensed the white space bands for communication,
the propagation variation has to be considered in wireless network. 

%We use a similar assumption for %channels within a frequency band, but consider the propagation 
%differences of a channel in one band (e.g., 450 MHz) as compared to 
%another (e.g., 2.4 GHz). FIXME modified
Moreover, the FCC has flexible rules all over the states because of the 
existing TV stations and devices working in white space bands~\cite{googledatabase}.
These existing channel occupancy has to be considered in the wireless communication with
white space frequency~\cite{fallah2010congestion}. To quantify the 
channel occupancy, we have the concept activity level from~\cite{cui2013leveraging}
as defined in Eq.~\ref{eq:actlevel}. The concept is defined based on a long-term measurement 
as the percentage of sensing samples ($S_\theta$) above an 
interference threshold ($\theta$) over the total samples ($S$) of each band ($b$) 
during a unit time. 
%The activity level ($A$) represents inter-network interference:
\begin{equation}
\label{eq:actlevel}
A=\frac{S_\theta}{S_a}
\end{equation}

Wireless mesh networks are a particular type of multihop wireless network
that are typically considered to have at least two
tiers~\cite{CRSK06}: {\it (i)} an access tier, where client traffic 
is aggregated to and from mesh nodes, and {\it (ii)} a multihop 
backhaul tier for connecting all mesh nodes to the Internet through 
gateway nodes. In this work, we focus on optimally allocating 
white space and WiFi bands on a finite set of radios per mesh node
along the backhaul tier, since we assume that client devices will use 
WiFi (due to the economies of scale) along access tier.  
In each of the WhiteMesh topologies studied in Section
~\ref{sec:experimentdesign}, a sufficient number of orthogonal WiFi 
channels remain for the access tier clients connection through additional 
radios co-located on the mesh nodes.

\begin{figure}
\vspace{-0.0in}
\centering
\includegraphics[width=84mm]{figures/interferencerange2}
\vspace{-0.1in}
\caption{Example WhiteMesh topology with different mesh-node shapes 
representing different frequency band choices per link.}
\label{fig:interferencerange}
\vspace{-0.2in}
\end{figure}

In this work, we consider the diverse propagation characteristics
for four frequency bands: 450 MHz, 800 MHz, 2.4 GHz, and 5.8 GHz.
We refer to the two former frequency bands as white space (WS) bands, whereas
the two latter frequency bands as WiFi bands.
% Make multiband challenges
Due to the broadcast nature of the wireless medium, greater levels of
propagation induce higher levels of interference. Also, the more allowed frequency
channels and low level signal activity of white space bands co-exist in sparse areas. 
Thus, in sparsely-populated
rural areas, the lower frequencies of the white space bands might be a
more appropriate choice for multihop paths to gateways having reduced hop
count with more capacity. However, as the population and demand scales up (e.g., for 
urban regions), the reduced spatial reuse and greater levels of interference 
of white space bands might detract from the overall channel assignment strategy. 
In such urban areas, select links of greater distance might be the most 
appropriate choice for white space bands, especially since the number of 
available channels is often inversely proportional to the population (due 
to the existence of greater TV channels).

Figure \ref{fig:interferencerange} depicts an example where mesh node $A$ 
could connect to mesh node $C$ through $B$ at 2.4 GHz, or directly connect 
to $C$ at 450 MHz. If 2.4 GHz were used, link $D,E$ might be able to reuse
2.4 GHz if they are out of the interference range. However, if link $A,C$
used 450 MHz, a lower hop count would result for the path, but lower levels
of spatial reuse also result (e.g., for link $D,E$). While the issues of 
propagation, interference, and spatial reuse are simple to understand,
the joint use of white space and WiFi bands to form optimal WhiteMesh 
topologies is challenging since the optimization is based on the knowledge
of prior channel assignment which is not available before the work has been
finished.

\subsection{Model and Problem Formulation}
\label{subsec:problem}

% Assumptions of the network
Multiband wireless network involve the propagation variation
and channel occupancy additional to the previous multichannel scenario. 
Thus, we improve the prior multi-channel models~\cite{tang2005interference,
doraghinejad2014channel} to adopt the new parameters. In 
these previous models, they fail to distinguish the communication 
range $D_c$, interference range $D_r$ and channel capacity variation 
across frequency bands. While these works would attempt to minimize 
the interference in a multihop network topology by an optimal channel 
assignment for a given set of radios, we hypothesize that
using radios with a greater diversity in propagation could yield overall network 
performance gains.
Therefore, for a given set of radios, we allow the channel
choices to come from multiple frequency bands (i.e., multiband channel 
assignment, which also includes multiple channels in the same band).
In this work, we assume all channels have the same bandwidth and 
adjust the channel capacity of each band through the measured 
channel occupancy parameter activity level~\ref{eq:actlevel}.
% FIXME Weds. Here
In practice, we could easily calculate the proportional ratio of channels in 
each band according to their bandwidth. We assume that the locations of 
mesh nodes and gateway nodes are given and all mesh nodes have radios with the same 
transmit power, channel bandwidth, and antenna gain.
Each mesh node operates with a classic protocol model~\cite{gupta2000capacity}. 

A mesh network could be represented by a unidirectional graph $G=(V,E)$, where
$V$ is the set of mesh nodes, and $E$ is the set of all possible physical links 
in the network. If the received signal (according to Eq.~\ref{eq:friis}) between 
two mesh nodes $i,j$ for a given frequency band (from the set of all bands $B$) 
is greater than a communication-range threshold, then a data link exists and 
belongs to the set $L$ with a fixed, non-zero capacity $\delta$ with a protocol model.  
Due to the frequency occupancy, the available channel capacity could be calculated 
through Eq.~\ref{eq:intercap}.
\begin{equation}
\label{eq:intercap}
\delta_r=\delta*(1-\bar{A})
\end{equation}
The capacity of a clean channel is denoted by $\delta$. Under the protocol model, the capacity 
of a channel with interference of existing signals $\delta_r$ could be represented as 
the remaining free time of the channel capacity according to the measured
activity level $A$. We employ the activity level based on multiple measurements
in the target area to represent the average inter-network interference.
Correspondingly, a connectivity graph $C$ is formed for each 
band in $B$ such that $C=(V,L,B)$.  If the received signal for a given band is 
above an interference-range threshold, then contention occurs between
nodes.  We extend the conflict matrix in~\cite{tang2005interference} related to
different interference per band according to $F=(E_{i,j},I_{Set},B)$, where $E_{i,j}$
represents the link and $I_{Set}$ includes all the links are physically inside 
the interference range $D_r$ when operating on each band $b \in B$.

Therefore, the problem we model is: to choose the connectivity graph $C'$ which maximizes
the metrics obey the constraints of multiband wireless network (defined below).
A key challenge is that selecting the optimal channels from
the set $B$ leads to a conflict graph $F$ which cannot be known {\it a priory}.
Previous works have proposed several coloring, cluster-independent set, 
mixed linear integer methodology for a single band $b$~\cite{peng2012efficient,tang2005interference,doraghinejad2014channel}. 
However, these works do not address a reduction in hop count or an increase in spatial reuse and channel occupancy for a set of 
diverse bands $B$. 

% Metrics
In network application, the bottleneck of mesh network capacity has been 
shown to be the gateway's wireless connections~\cite{robinson2008adding}.
The metric we use to evaluate the proposed algorithm is the traffic arrived at gateway nodes.
Networks are operated and maintained by vendors, such as AT\&T, T-Mobile, who charge the 
customers based on their data through Internet. Thus, we use random generated numbers 
to represent clients' traffic demands. The traffic arrived at gateway nodes correlated 
to the population of the area since people have almost traffic demand in long term average. 
The employed performance metric of traffic arrived gateway $X$, is represented the traffic 
arrived at the gateway nodes, where in Eq.~\ref{eq:goodput}:
\begin{equation}
\label{eq:goodput}
X=\sum_{w \in W, v \in V}T(w,v)
\end{equation}
The traffic arrived gateway node $w\in W$ considers all incoming and outgoing wireless traffic 
from access node $v\in V$ as $T$ onto the Internet.
Obviously, the traffic arrived gateway is also related to the routing and other factors, 
we use a simple routing method to keep the maximum the traffic arrived at the gateway nodes, 
the exact calculation of gateway traffic arrived gateway is described in 
Section~\ref{sec:experimentdesign} and consider where to put the gateway nodes are outside 
the scope of this work.

