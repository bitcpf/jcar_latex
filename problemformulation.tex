\section{Problem Formulation}
\label{sec:problemformulation}

In this section, we formulate the problem of how to optimally use WiFi and white space bands 
in concert when deploying wireless mesh networks. We first introduce the multiband mesh 
network system model and illustrate the challenges of such a WhiteMesh architecture in 
propagation and channel occupancy. As opposed to previous works such as
~\cite{tang2005interference,yuan2006cross,si2010overview}, we focus on both the access tier 
network deployment and backhual tier network design. 

% Background
\subsection{White Space Opportunity and Challenge}
\label{subsec:motivation}
% Propagation
Wireless propagation refers to the signal loss characteristics when wireless signals 
are transmitted through the wireless medium. The strength of the received signal 
depends on both the line-of-sight path (or lack thereof) and multiple other paths 
that result from reflection, diffraction, and scattering from obstacles~\cite{andersen1995propagation}. 
The widely-used Friis equation characterizes the received signal power $P_r$ in terms 
of transmit power $P_t$, transmitter gain $G_t$, receiver gain $G_r$, wavelength 
$\lambda$ of the carrier frequency, distance $R$ from transmitter to receiver, and 
path loss exponent $n$ according to~\cite{friis}:
\begin{equation}
\label{eq:friis}
P_r=P_t+G_t+G_r+10n \log_{10}\left( \frac{\lambda}{4\pi R}\right)
\end{equation}
Here, $n$ varies according to the aforementioned environmental factors with a value 
ranging from two to five in typical outdoor settings~\cite{rappaport}.

% Explain multiband vs multichannel Adding channel occupancy here, and also activity level
A frequency band is commonly defined as a group of channels which have similar propagation 
characteristics with small frequency separation. A common assumption in previous works that 
use many WiFi channels is that the propagation characteristics of each channel is similar 
to another, since the channel separation is relatively small (e.g., 5 MHz for the 2.4 GHz band).
% Previous work focus on multichannel
Many works which rely on such uniform propagation assumption have focused on the allocation 
of multiple WiFi channels with multiple radios in multihop wireless networks with channels in 
one band~\cite{doraghinejad2014channel}. However, as FCC licensed the white space bands for 
communication, the propagation variation has to be a consideration in wireless networks. 
Moreover, the FCC has flexible rules all over the states because of the existing TV stations 
and devices working in white space bands~\cite{googledatabase}. These existing channel occupancy 
has to be considered in the wireless communication within white space frequency~\cite{fallah2010congestion}. 

% Clearify the problem
% Network deployment objective
Wireless mesh networks are a particular type of multihop wireless network that provides access 
for users through wireless links with low cost. Naturally, in wireless network design, there is 
trade off between the budget and the quality of service. Typically, wireless mesh network is 
considered to have at least two tiers~\cite{CRSK06}: {\it (i)} an access tier, where client 
traffic is aggregated to and from mesh nodes, and {\it (ii)} a multihop backhaul tier for 
connecting all mesh nodes to the Internet through gateway nodes. To provide better service for 
more users, the network has to be designed with more access points in the access tier, and 
offers more capacity in the backhual tier. However, more access points and backhual capacity 
induce huge cost for network deployment. Fortunately, white space large propagation helps solving 
these issues. In this work, we explore the white space spectrum application for both access tier 
network deployment and further focus on optimally allocating white space and WiFi bands on a 
finite set of radios per mesh node along the backhaul tier.  

% Advantages
In the white mesh networks, the white space spectrum offers not only more allowed frequency resources
but also a larger service area for a single access point. With the flexible FCC restrictions and 
artificial activity diversity of white space spectrum, in sparsely-populated rural areas, the 
lower frequencies of the white space bands might be a more appropriate choice for access point 
deployment and connections for backhual networks. A white space access point has a larger service
area comparing with WiFi access point only counting propagation. Long propagation also reduce the 
hop counts in backhual networks with more capacity.  
% Drawbacks
However, as the population and demand scales up (e.g., for urban regions), the reduced spatial 
reuse and greater levels of interference of white space bands might detract from the overall wireless
network deployment strategy. In such urban areas, select links with more spatial reuse ability might
be a more appropriate choice, especially since the number of available channels in white space bands 
is often inversely proportional to the population (due to the existence of greater TV channels).




\begin{figure}
\vspace{-0.0in}
\centering
\includegraphics[width=84mm]{figures/interferencerange2}
\vspace{-0.1in}
\caption{Example WhiteMesh topology with different mesh-node shapes 
representing different frequency band choices per link.}
\label{fig:interferencerange}
\vspace{-0.2in}
\end{figure}


\subsection{Problem Formulation}
\label{subsec:problem}
% Talk about the challenges
Larger propagation range results in cost reducing of access points deployment in access 
tier deployment through larger service area. Figure \ref{fig:interferencerange} depicts 
an example depicts the propagation diversity in white mesh networks. The mesh node $A$ 
could connect to mesh node $C$ through $B$ at 2.4 GHz, or directly connect to $C$ at 450 
MHz. If 2.4 GHz were used, link $D,E$ might be able to reuse 2.4 GHz if they are out of 
the interference range. However, in backhual tier network if link $A,C$ used 450 MHz, a 
lower hop count would result for the path, but lower levels of spatial reuse also result 
(e.g., for link $D,E$). While the issues of propagation, interference, and spatial reuse are 
simple to understand, the joint use of white space and WiFi bands to form optimal WhiteMesh 
topologies is challenging since the optimization is based on the knowledge of prior channel 
assignment which is not available before the work has been finished.


% Objective of the work,high level problem
White mesh wireless network involves the propagation variation and channel occupancy additional to 
the previous multichannel works. For access tier deployment, the key issue is to reduce the cost
for access points with quality of service constraints. In the backhual network, the objective of 
the deployment is to offer more capacity for the mesh nodes which are the access points in the 
access tier. 

% Difference from previous work
% Multi band model/ assumptions
The prior multi-channel models~\cite{tang2005interference, doraghinejad2014channel} fail to 
distinguish the propagation range $D_c$, interference range $D_r$ and channel capacity variation 
among frequency bands. While these works would attempt to minimize the interference in a multihop 
network topology by an optimal channel assignment for a given set of radios, we hypothesize that
using radios with a greater diversity in propagation could yield overall network performance gains
in both access tier and backhual tier networks. Therefore, in the white mesh model, for a given set 
of radios, we allow the channel choices to come from multiple frequency bands (i.e., multiband channel 
assignment, which also includes multiple channels in the same band). We assume all channels have 
the same bandwidth and the same channel capacity of each band with the same transmit power, channel 
bandwidth, and antenna gain under a classic protocol model~\cite{gupta2000capacity}. 


% Change to 2 sub problems
In a typical wireless network deployment, there are 3 sub steps. First step is to learn the 
electromagnetic environment of the target area for the network design. Next step is to design the 
access points deployment for the area. The last step is to build the backhual network offering 
the access to the Internet. We introduce the activity level and then apply the measured parameter 
to the other steps with constraints in our frameworks. 

% Explain multiband and activity level
% Concept of activity level
%
% Spectrum utility vary across different areas


In sparsely-populated rural areas, the lower frequencies of the white space bands might be a better 
choice for wireless service in sparsely populated areas for its low utility. However, as the population 
and traffic demand scales up (e.g., for urban regions), the greater levels of residents traffic demands 
might detract the white space bands from the overall deployment strategy. That motivate the learning of 
target area electromagnetic environment in prior of the network design. For spectrum utility and resulting 
channel availability, we use a long-term measurement for each band.  We define the percentage of sensing 
samples ($S_\theta$) above an interference threshold ($\theta$) over the total samples ($S$) in a time 
unit as the activity level ($A$) of inter-network interference:
\begin{equation}
\label{eq:actdef}
A=\frac{S_\theta}{S_a}
\end{equation}

% Discuss the application of activity level
Based on the in-field measurements, we can further get the achieved channel capacity through the remaining
free time according to:
\begin{equation}
\label{eq:intercap}
\delta_r=\delta*(1-\bar{A})
\end{equation}
The capacity of a clean channel is denoted by $\delta$ with the protocol model. The achieved channel capacity is
the start point for network design. In a joint white space and WiFi scenario, the activity level varies 
according to various interfering sources and the propagation characteristics induced by the environmental 
characteristics of the service area.


% Previous white mesh
Despite sufficient levels of received signal, interference can cause channels to be unusable (e.g., due to 
high levels of packet loss) or unavailable (e.g., due to primary users in cognitive radios~\cite{haykin2005cognitive}).
Prior work has worked to reduce interference levels via gateway deployment, channel assignment, and routing
~\cite{he2008optimizing,tang2005interference}. The interference of a wireless network could be divided into two 
categories according to the interfering source: {\it (i)} intra-network interference, caused by nodes in the 
same network, and {\it (ii)} inter-network interference, caused by nodes or devices outside of the network. 
The definition of activity level provides a method to quantify the inter-network interference in the deployment design.
We apply the measurements based activity level in both access tier network design and backhual network design.


% Winmee
For the access tier network, a simple method with the least number of access points to cover an area is to 
use multiple orthogonal lower-frequency channels. However, the FCC limits white space band availability 
for data networks in most metropolitan areas in the United States~\cite{googledatabase}. Moreover, the 
number of channels in each band is limited. Too many lower-frequency channels will cause high levels of 
intra-network interference for the network, which will be discussed in the backhual network design. We 
assume that the cost of the network is proportional to the number of access points required for a given 
user demand (i.e., due to the cost of hardware and installation). Therefore, given a geographical region for 
a new network deployment, we build a measurement-driven framework called Multiband Access Point Estimation 
(MAPE)\ref{sec:winmee} to compute the required number of access points.


Most of the existing works try to reduce the intra-network interference without regard to the inter-network 
interference~\cite{si2010overview}. However, the existence of inter-network interference becomes an important 
problem when involving white space bands.  While theoretical models which describe inter-network interference 
exist, accurately characterizing a particular region must be done empirically. To investigate the performance 
of white space bands, we model the problem as a linear program in \ref{sec:whitemesh}. Further, to reduce the 
complexity for channel assignment, a BPS(Band-based Path Selection) algorithm is proposed in \ref{sec:whitemesh}, and
the measurements driven simulation shows the gain of joint WiFi and white space with BPS and other channel assignment
methods.


