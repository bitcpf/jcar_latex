\section{Problem Formulation}
\label{sec:problemformulation}

%\begin{table*}[t] 
\centering % centering table 
\begin{tabular}{|l|r|} % creating 12 columns 
\hline %\hline % inserting double-line 
% Entering 1st row 
$\alpha$   & Path Loss Exponent     \\
\hline % inserts single-line 
R & Communication Range \\
\hline % inserts single-line 
$I_r$ & Interference Range \\
\hline % inserts single-line 

\end{tabular} 
\label{tab:2channelcombination} 
\caption{Throughput achieved through Gateway nodes (Mbps) for various combinations of WiFi and White Space (WS) mesh topologies (Offered Load = 4 Mbps, Network Size = 30 mesh nodes).} % title name of the table 
\vspace{-0.1in}
\end{table*} 


% Organization of the Sec
In this section, we formulate the problem of how to optimally 
use WiFi and white space bands in concert when deploying wireless 
mesh networks.  We first describe our system model and illustrate 
the challenges of such a WhiteMesh architecture.  We then discuss
how to evaluate WhiteMesh networks and the corresponding goal of
both the optimization framework and the heuristic algorithm that 
we propose in the following section.  Finally, we present 
our integer linear programming model used to address the problem. 
 
\subsection{WhiteMesh Network Architecture}
\label{subsec:architecture}

%A node in ~\emph{Multiband Wireless Mesh Network} has limited multiple slots for installing radios working in different bands. 
%Two nodes share the same link should have a common channel. The sum of the loads on the links should 
% Explain propagation, factors of the environment and so on
Wireless propagation is the behavior of the signal loss characteristics 
when wireless signals are transmitted from a transmitter to receiver.
The strength of the receiving signal depends on both the line-of-sight
path (or lack thereof) and multiple other paths that are a result of 
reflection, diffraction, and scattering from obstacles in the 
environment~\cite{andersen1995propagation}. The widely-used Friis
equation characterizes the power of the received signal $P_r$ in terms 
of the power $P_t$ and gain $G_t$ of the transmitting signal, gain of 
the receiver $G_r$, wavelength $\lambda$ of the carrier frequency, 
distance $R$ from transmitter to receiver, and path loss exponent $n$ according 
to~\cite{friis}:
\begin{equation}
\label{eq:friis}
P_r=P_t+G_t+G_r+10n \log_{10}\left( \frac{\lambda}{4\pi R}\right)
\end{equation}
Here, the path loss exponent $n$ changes according to the
aforementioned environmental factors and ranges from 2 to 5 in typical
outdoor settings~\cite{rappaport}.

% Explain multiband vs multichannel Adding channel occupancy here, and also activity level
A common assumption in works that use many WiFi channels is that the
propagation characteristics of one channel is similar to another, 
since the channel separation is relatively small (e.g., 5 MHz for 
the 2.4 GHz band).
Many works which rely on such an assumption have focused on the 
allocation of multiple WiFi channels with multiple radios in 
multihop wireless networks~\cite{si2010overview}.  Here, a frequency 
band is defined as a group of channels which have
similar propagation characteristics.
%We use a similar assumption for %channels within a frequency band, but consider the propagation 
%differences of a channel in one band (e.g., 450 MHz) as compared to 
%another (e.g., 2.4 GHz). FIXME modified
In the application of wireless communication, channel occupancy has
to be considered~\cite{fallah2010congestion}. To quantify the 
channel occupancy, we inherit the activity level from~\cite{cui2013leveraging}
to quantify the channel occupancy as shown in Eq.~\ref{eq:actlevel}.
The activity level is defined based on a long-term measurement 
of each band.  It is defined as the percentage of sensing samples ($S_\theta$) above an 
interference threshold ($\theta$) over the total samples ($S$) in a time unit as the 
activity level ($A$) of inter-network interference:
\begin{equation}
\label{eq:actlevel}
A=\frac{S_\theta}{S_a}
\end{equation}

In this work, we consider the diverse propagation characteristics
for four frequency bands: 450 MHz, 800 MHz, 2.4 GHz, and 5.8 GHz.
We refer to the two former frequency bands as white space (WS) bands, whereas
we refer to the two latter frequency bands as WiFi bands.

Wireless mesh networks are a particular type of multihop wireless network
that are typically considered to have at least two
tiers~\cite{CRSK06}: {\it (i)} an access tier, where client traffic 
is aggregated to and from mesh nodes, and {\it (ii)} a multihop 
backhaul tier for connecting all mesh nodes to the Internet through 
gateway nodes. In this work, we focus on how to optimally allocate 
white space and WiFi bands on a finite set of radios per mesh node
along the backhaul tier, since we assume that client devices will use 
WiFi (due to the economies of scale).  In each of the WhiteMesh 
topologies studied in Section~\ref{sec:experimentdesign}, a sufficient 
number of orthogonal WiFi channels remain for the access tier 
clients connection through additional radios co-located on the mesh nodes.

\begin{figure}
\vspace{-0.0in}
\centering
\includegraphics[width=84mm]{figures/interferencerange2}
\vspace{-0.1in}
\caption{Example WhiteMesh topology with different mesh-node shapes 
representing different frequency band choices per link.}
\label{fig:interferencerange}
\vspace{-0.2in}
\end{figure}

% Make multiband challenges
Due to the broadcast nature of the wireless medium, greater levels of
propagation induce higher levels of interference.  Thus, in sparsely-populated
rural areas, the lower frequencies of the white space bands might be a
more appropriate choice for multihop paths to gateways having reduced hop
count. However, as the population and demand scales up (e.g., for 
urban regions), the reduced spatial reuse and greater levels of interference 
of white space bands might detract from the overall deployment strategy. In 
such urban areas, select links of greater distance might be the most 
appropriate choice for white space bands, especially since the number of 
available channels is often inversely proportional to the population (due 
to the existence of greater TV channels).
%The broadcast nature of the wireless medium makes it generate multiple access interference in wireless network.
%Employing White Space Band in lower frequency brings advantages for mesh network, 1) more orthogonal bandwidth reduce the contention and conflict in the network,
% 2) the propagation variation brings flexible topology by reducing connection hop counts in the network.
%However, at the same time, links in White Space Band also increase the interference range in the network making space reuse of the white space band channel difficult. 

Figure \ref{fig:interferencerange} depicts an example where mesh node $A$ 
could connect to mesh node $C$ through $B$ at 2.4 GHz, or directly connect 
to $C$ at 450 MHz. If 2.4 GHz were used, link $D,E$ might be able to reuse
2.4 GHz if they are out of the interference range. However, if link $A,C$
used 450 MHz, a lower hop count would result for the path, but lower levels
of spatial reuse also result (e.g., for link $D,E$). While the issues of 
propagation, interference, and spatial reuse are simple to understand,
the joint use of white space and WiFi bands to form optimal WhiteMesh 
topologies is challenge.

%To balance the larger communication range and interference range of white space band in mesh network is a key issue in ~\emph{Multiband Mesh Network} from ~\emph{Multiradio} scenario.

\subsection{Model and Problem Formulation}
\label{subsec:problem}

% Assumptions of the network
In this work, we have a model similar to prior multi-channel models~\cite{tang2005interference,
doraghinejad2014channel,si2010overview}. However, in these previous models, 
they fail to distinguish the communication range $D_c$ and interference range
$D_4$ variation across frequency bands.  
%FIXME HERE 18:10
While
these works would attempt to minimize the interference in a multihop network topology
by an optimal channel assignment for a given set of radios, we hypothesize that
using radios with a greater diversity in propagation could yield overall network 
performance gains.  
% NEWClaim we consider we could use either multiple channels in a band and also multiple channels in different bands
Therefore, for a given set of radios, we allow the channel
choices to come from multiple frequency bands (i.e., multiband channel 
assignment, which also includes multiple channels in the same band).
To simplify the analysis, we assume the channel capacity of each band is equal.
% FIXME Weds. Here
In practice, we could easily calculate the proportional ratio of channels in 
each band according to their bandwidth.
We assume that the locations of mesh nodes and gateway nodes are given and
all mesh nodes have the same transmit power, channel bandwidth, and antenna gain.
Each mesh node operates with a classic protocol model~\cite{gupta2000capacity}. 

A mesh network could be represented by a unidirectional graph $G=(V,E)$, where
$V$ is the set of mesh nodes, and $E$ is the set of all possible physical links 
in the network. If the received signal (according to Eq.~\ref{eq:friis}) between 
two mesh nodes $i,j$ for a given frequency band (from the set of all bands $B$) 
is greater than a communication-range threshold, then a data link exists and 
belongs to the set $L$ with a fixed, non-zero capacity $\delta$ according to the protocol 
model.  
Due to the frequency occupancy, the available channel capacity is calculated 
through Eq.~\ref{eq:intercap}.
\begin{equation}
\label{eq:intercap}
\delta_r=\delta*(1-\bar{A})
\end{equation}
The capacity of a clean channel is denoted by $\delta$. With the protocol model, the capacity 
of a channel with inter-network interference $\delta_r$ could be represented as 
the remaining free time of the channel capacity according to the measured
activity level $A$.
Correspondingly, a connectivity graph $C$ is formed for each 
band in $B$ such that $C=(V,L,B)$.  If the received signal for a given band is 
above an interference-range threshold, then contention occurs between
nodes.  We extend the conflict matrix in~\cite{tang2005interference} related to
different interference per band according to $F=(E_{i,j},I_{Set},B)$, where $E_{i,j}$
represents the link and $I_{Set}$ includes all the links are physically inside 
the interference range $D_r$ when operating on each band $b$.

Therefore, the problem we model is: to choose the connectivity graph $C'$ which maximizes
the metrics according to the constriants of multiband wirless network (defined below).
A key challenge is that selecting the optimal channels from
the set $B$ leads to a conflict graph $F$ which cannot be known {\it a priori}.
Previous works have proposed a coloring, cluster-independent set, mixed linear integer methodology
for a single band $b$~\cite{peng2012efficient,tang2005interference,doraghinejad2014channel}. 
However, these works do not address a reduction in hop count or an increase in spatial reuse for a set of 
diverse bands $B$. 

% Metrics
In network application, the bottleneck of mesh network capacity has been shown to be the gateway's wireless 
connections~\cite{robinson2008adding}.
The metric we use to evaluate the proposed algorithm is the traffic arrived at gateway nodes.
Networks are operated and maintenaced by vendors.
The vendors, such as AT\&T, T-Mobile, they chare the custumers based on the data amount. 
The traffic arrived at gateway nodes under an random traffic generated area 
correlated to the incomes of the vendors. We use the performance metric of 
traffic arrived gateway $X$, represented the traffic arrived at the gateway nodes, where in Eq.~\ref{eq:goodput}:
\begin{equation}
\label{eq:goodput}
X=\sum_{w \in W, v \in V}T(w,v)
\end{equation}
The traffic arrived gateway node $w\in W$ considers all incoming and outgoing wireless traffic 
from access node $v\in V$ as $T$ onto the Internet.
Obviously, the traffic arrived gateway is also related to the routing, we use a simple routing
method to keep the maximum the traffic arrived at the gateway nodes, the exact calculation 
of gateway traffic arrvied gateway is described in Section~\ref{sec:experimentdesign} 
and consider where to put the gateway nodes are outside the scope of this work.

We also use the general metric \emph{network throughput} $NT$
in~\cite{tang2005interference,doraghinejad2014channel} 
which directly associated with the interfering number of each link, as~\ref{eq:networktpt}:
\begin{equation}
\label{eq:networktpt}
NT = \sum_{i=1}^{n}\frac{1}{1+Interference\ number\ of\ (e \in E)}
\end{equation}
The network throughput is the sum of shared bandwidth of each link in the network, 
which represents the worst case performance of the wireless network.

