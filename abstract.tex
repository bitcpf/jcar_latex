\begin{abstract}
The in-field trials revealed the node spacing required for WiFi propagation induced a prohibitive cost 
model for network carriers to deploy. The digitization of TV channels and new FCC regulations have 
reapportioned spectrum for data networks with far greater range than WiFi due to lower carrier frequencies. 
Thus, jointly apply the low frequency white space spectrum become an emergency issue for the deployment of 
data networks. 
In this paper, we quantify the channel occupancy in both WiFi and white space frequencies through measurements 
in Dallas-Fort Worth metropolitan, propose a measurement-driven band selection framework, Multiband Access 
Point Estimation (MAPE), and design a measurement driven heuristic algorithm, Band-based Path Selection (BPS), 
to approach optimally channel assignment in WhiteMesh networks with both white space and WiFi spectrum with 
reduced complexity. 
The numerical result shows that the number of access points reduces by up to 1650\% in sparse rural areas over similar 
WiFi-only solutions. It achieves up to 160\% served traffic flow gain competing with existing multi-channel, 
multi-radio algorithms, which are agnostic to diverse propagation characteristics across bands. Moreover, we 
show that, with similar channel resources and bandwidth, the joint use of WiFi and white space bands can achieve 
a served user demand of 170\% over that of mesh networks with only WiFi bands or white space bands, respectively. 
\end{abstract}
