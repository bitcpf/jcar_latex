\begin{abstract}
While there were high hopes for multihop wireless networks (mesh) to provide ubiquitous WiFi 
in many cities, in-field trials revealed the node spacing required for WiFi propagation 
induced a prohibitive cost model for network carriers to deploy. However, the digitization of TV 
channels and new FCC regulations have reapportioned spectrum for data networks with far 
greater range than WiFi due to lower carrier frequencies. While dense areas might still benefit from the use of WiFi for spatial reuse, white 
space spectrum becomes increasingly important as the population density decreases, 
especially since the availability of these white spaces are often inversely proportional to population density. 
% Our work
In this paper, we perform measurements in Dallas-Fort Worth 
metroplex to analyze channel occupancy in both WiFi and white space frequencies. 
% winmee
Based on these measurements, we propose a measurement-driven band 
selection framework, Multiband Access Point Estimation (MAPE) to quantify the benefit of
jointly using WiFi and white space frequencies along the access tier. 
% whitemesh
Further, to deploy heterogeneous backhaul tier, we design a measurement-driven heuristic algorithm, Band-based Path Selection (BPS), to approach optimal channel 
assignment of both white space and WiFi spectrum with reduced computational complexity. 
Numerical results show that MAPE reduces the number of access points by up to 1650\% in sparse 
rural areas over similar WiFi-only solutions. Also, BPS achieves up to 180\% in the served traffic flow of 
existing multi-channel, multi-radio algorithms, which are agnostic to diverse 
propagation characteristics across bands. 
Most importantly, this paper lays a foundation for optimal use of white space and WiFi bands in the access 
and backhaul tiers of mesh networks across diverse population densities.
%
%Moreover, we show that, with similar channel resources 
%and bandwidth, the joint use of WiFi and white space bands can achieve a served traffic flow of 
%180\% of WiFi only and white space only mesh networks, respectively. 
\end{abstract}
