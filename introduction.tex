\section{Introduction}
\label{sec:introduction}

% Background multiband networks, hetergeneous networks, vehicular applications
% Introduction of white spcae.
% Process:
% Clearify the channel in different envrionment 
% Introduce 2 possible gain: 1. reduce # mesh nodes 2. given topology get average capacity 
\emph{White Space Bands} is popularly referred to unused portions of the UHF and VHF spectrum includes, but not limited to 54M-72MHz,76M-88MHz,174M-216MHz,470M-608MHz,and 614M-806MHz ~\cite{whitespacewiki}.
FCC adopted rules to allow unlicensed radio transmitters to operate in the white space freed from TV band since 2010~\cite{fccwhitespace}. 
 These bands provides superior propagation and building penetration compared to licensed ISM Wifi bands like the 2.4 and 5GHz bands, holding rich potential for expanding broadband capacity and improving access for wireless users.
% Differences among cities, tell something about the evaluation
% Wireless mesh networks
The advantages of white space equipment could be seen straight forward from using in ~\emph{Wireless mesh network}.
Wireless mesh network is able to provide broadband Internet access to large contiguous areas through less mesh nodes equipped with white space devices(WSDs) for the last-one-mile due to the propagation characteristic of ISM bands. 
White space mesh is an economic way to provide back-haul Internet service in neighborhood area employing the propagation characteristics.
The propagation advantages White space could also be used to improve the urban or high density area connectivity through lower interference channels. 

As a kind of multi-radio architecture, \emph{Multi-band Multi-radio Mesh Network} also has the same advantages of \emph{Multi-Channel Mesh Network} as increasing the network capacity. And also the challenges of reducing interference among active links is a main issue. 
The problem of assign channels in mesh network has been proved as a NP-Hard problem.~\cite{jain2005impact}. 
The performance challenges of \emph{Multi-channel Multi-radio Mesh Network} have long been recognized and have led to a lot of research on the traffic profiling, channel assignment, routing, and topology control. A vast array of channel assignment for \emph{Multichannel Mesh Network} have been proposed and reviewed in several articles.
Ashish ~\cite{raniwala2004centralized} use \emph{Load Aware Channel Assignment} to approach the channel assignment optimization; Jian ~\cite{tang2005interference} employ a channel partion methodology to improve the channel assignment; In ~\cite{jain2005impact}, Kamal describe the up-bound and lower-band based on ~\emph{Conflict Graph}.

This paper also deals with the problem of computing the optimal channel assignment of a mesh network, even more we deal with the influence of different propagation across multi-band. We give wireless network configuration specified as inputs. 
The inputs have the node locations, available bands, ranges etc. We consider the ~\emph{Communication Range} and \emph{Interference Range} variety of bands as a new factor in our \emph{Multiband Multiradio Wireless Mesh Network} architecture.  
With the new multiband factor, a single link could get a better RSSI with better throughput according to propagation models, such as Friis model; However, at the same time, the interference in the whole network increase too. 
To balance the good and bad things, as Christelle proposed a framework in ~\cite{yuan2006cross}, we propose a linear optimization model to describe constraints of such a network with more detailed constraints. However, this model is still a NP-Hard which could not be resolved in a polynomial time. 

%Describe algorithms
To get an approaching channel assignment, we propose a ~\emph{Path Efficiency over Network} to evaluate each link and path in the network. Based on this parameter, we develop 2 novel channel assignment heuristic algorithms for the ~\emph{Multiband Multiradio Wireless Mesh Network}. The first algorithm is a tree grnerated process to reduce hop count avoiding interference. The second algorithm starts from the worst case then interated improve the \emph{Path Efficiency over Network}.

% Paper focusing
This work focus on channel assignment of mesh network given multiband gateways and mesh nodes information.
We analysis the problem and algorithm complexity through a regular grid network and evaluate our approaching in two in-field network placement. 

% Contribution
% problem/environment propose
% algorithms
% 
The main contributions of our work are as follows:
\begin{itemize}
\item We formulate the heterogeneous multiband multiradio channel assignment problem with a linear optimization model.  

\item We propose a parameter ~\emph{Path Efficiency over Network} to evaluate multi-hop path.

\item We develop 2 heuristic algorithms to approach the optimized channel assignment.
% Also should we consider the cost of different kind of nodes (1 radio, 2 radios, 3 radios )

\item We perform extensive simulation on regular grid network and in-field mesh network topology to evaluate our algorithms.


\end{itemize}

The remainder of this paper is organized as follows. In Section ~\ref{sec:model}, we formulate the multiband wireless mesh network channel assignment model and analyze the factors of this architecture. 
In Section ~\ref{sec:algorithms} discusses the algorithms approaching the optimal placment of a multiband netowrk. 
%Then we present the experiment design in Section ~\ref{sec:experiment_design} and analyze the results in Section ~\ref{sec:results}.

