\section{Introduction}
\label{sec:introduction}

% Background multiband 
% Mesh Netrork
% Mesh Network traditonal utibility
% White space benefit challenges
% Contributions about new data
% Paper organization
% Wireless Network


% Many cities starts wifi deploy, failed
During the last decade, numerous cities solicited proposals from network carriers for 
exclusive rights to deploy city-wide WiFi, spanning hundreds of square miles. But many 
of these attempts failed, as the node spacing prohibited their cost effective implementation. 
As a result, many network carriers opted to pay millions 
of dollars in penalties rather than face the exponentially-increasing deployment costs 
(e.g., Houston~\cite{cnet_aug07} and Philadelphia~\cite{arstechnica_may08}). 
Thus, while a few mesh networks have been deployed in certain communities~\cite{CRSK06},
wireless mesh networks have largely been unsuccessful in achieving the scale of what was 
once anticipated~\cite{taps}. Specific to rural areas, the lack of user density and 
corresponding traffic demand per unit area as compared to dense urban areas allows greater 
levels of traffic aggregation across larger areas to reduce the total number of required access points, lowering
network deployment costs. Conversely, in densely-populated urban areas, the greater concentration of users 
and higher levels of traffic demand can be served by maximizing the spatial reuse. 

% New Technology for network design brought by new policyT
Meanwhile, the FCC has approved the use of broadband services in the unused portions of 
UHF TV bands, which were formerly exclusively licensed to television broadcasters. 
These white space bands are now available for unlicensed public use, enabling the deployment 
of wireless access networks across a broad range of scenarios from sparse rural areas 
(one of the key applications identified by the FCC) to dense urban areas~\cite{carlson}. 
The white space bands operate in available channels from 54-806 MHz, having a far greater 
propagation range than WiFi bands for similar transmission power~\cite{balanis2012antenna}.
These new resources and policies offer opportunities for both industry and academia. 

As shown in the spectrum white space databases (e.g.,~\cite{googledatabase}), the number of available 
channels in these spectrum varies from city to city in the US but generally is inversely proportional to population levels. 
Hence, the question arises for 
improving the performance and costs of wireless mesh networks: {\it (i) how can the 
white space bands improve large-scale mesh network deployments?}, {\it (ii) To what degree can 
white space bands reduce the network deployment cost of sparsely-populated, rural areas as opposed 
to comparable WiFi-only solutions?}, and {\it (iii) Where along the continuum of user population 
densities do the white space bands no longer offer cost savings for wireless network deployments?} 
While much work has been done on deploying multiple channel wireless networks, the differences in 
propagation among diverse carrier frequencies have not been exploited in their models~\cite{tang2005interference, 
doraghinejad2014channel} and the availability of white spaces is not considered, which could be fundamental issues for the success 
of mesh networks going forward. 


% Paper topic problem and possible solution
In this paper, we perform a measurement-driven study which jointly considers the propagation 
characteristics and in-field spectrum availability of white space and WiFi channels 
for optimally building the access and backhual tiers of a wireless mesh network.
As a representative metropolitan area, we measure the spectral activity data in the metropolitan 
and surrounding areas of Dallas-Fort Worth (DFW).  
We propose a measurement-driven framework, Multi-band Access Point Estimation (MAPE), to find the 
number of access points required for wireless network deployments in certain target areas with population 
densities from our in-situ measurements and census data. 
%Moreover, we leverage the diversity in propagation with channel occupancy of white space and 
%WiFi bands in the planning and deployment of the multihop wireless backhual tier. 
Further, we design a linear program and a heuristic measurement-driven algorithm, Band-based Path Selection (BPS), 
to address the channel assignment problem in wireless network deployment with both WiFi and white space bands. 
%The performance of our scheme is compared against two well-known multi-channel, 
%multi-radio channel assignment algorithms across a wide range of scenarios, including 
%network size, population distribution, deployment distance gap, etc. 


% Need some calculation data to show the improvement
The main contributions of our work are as follows:
\begin{itemize}
\item We perform in-field measurements of spectrum utilization in various representative
scenarios across the Dallas-Fort Worth metropolitan area, ranging from sparse rural to dense urban areas,  
considering the environmental setting (e.g., downtown, residential, or university campus).
\item We develop a measurement-driven Multi-band Access Point Estimation (MAPE) framework 
to jointly leverage propagation and spectrum availability of white space and WiFi bands 
for wireless access networks across environmental settings. We then analyze the framework under capacity and 
coverage constraints to show that, with white space bands, the number of access points 
can be greatly reduced from WiFi-only deployments by up to 1650\% in rural areas. 
\item We build a reduced complexity heuristic-based measurement-driven algorithm, Band-based 
Path Selection (BPS). BPS considers the diverse propagation and overall interference level 
of WiFi and white space bands. 
We further perform extensive analysis across offered loads,
network sizes and mesh nodes spacing across WiFi/white space band combinations to evaluate the 
performance versus prior multi-channel, multi-radio algorithms. 
\item We study the channel occupancy and node spacing impacts on mesh performance given similar 
channel resources (bandwidth and transmission power). The results show the total traffic served 
improvement of BPS in typical configurations, achieving up to 180\% of previous multi-channel algorithms.
\end{itemize}

The remainder of this paper is organized as follows. In Section~\ref{sec:problemformulation}, 
we introduce the challenge of deploying diverse frequency bands in wireless networks. 
In Section~\ref{sec:measurements}, we perform our DFW measurements to drive our algorithms and analysis
for diverse populations. 
In Section~\ref{sec:winmee}, we analyze the access tier network deployment, 
propose the MAPE framework, and discuss the white space bands impact. 
In Section~\ref{sec:whitemesh}, we develop a heuristic-based algorithm which consider which bands 
and multihop paths to select in a backhaul white space topology. We then evaluate the performance of 
the heuristic algorithm versus the upper bound of the optimal solution and 
compare their performance against two well-known multi-channel, multi-radio 
algorithms in several scenarios. Related work is discussed in Section~\ref{sec:related} and conclusions 
are drawn in Section~\ref{sec:conclusion}.

