
\section{Introduction}
\label{sec:introduction}

% Background multiband networks, hetergeneous networks, vehicular applications
% Introduction of white spcae.
FCC adopted rules to allow unlicensed radio transmitters to operate in the white space freed from TV band since 2010~\cite{fccwhitespace}. White space is popularly referred to unused portions of the UHF and VHF spactrum includes, but not limited to 54M-72MHz,76M-88MHz,174M-216MHz,470M-608MHz,and 614M-806MHz ~\cite{whitespacewiki}.
To use these band, FCC ruling requires white space devices(WSDs) to learn of spectrum availability at their respective locations from a database of incumbents. For example, in Dallas/Fort Worth, TX, 482M-488MHz band should be yielded for TV channels, in Chicago, IL 470M-482MHz band is reserved for TV channels. ~\cite{boradband}
 These bands provides superior propagation and building penetration compared to licensed ISM Wifi bands like the 2.4 and 5GHz bands, holding rich poetntial for expanding broadband capacity and improving access for wireless users.

% Wireless mesh networks
Wireless mesh network is able to provide broadband Internet access to large contiguous areas through less mesh nodes equipped with WSD due to the propagation characteristic of ISM bands. 
Mesh deployment requires selecting the number and locations to place mesh nodes to cover the whole area and the nodes are inter-connected in order to access to Internet gateway points.
Unfortunately, prior white space mesh placement research fails to address the compatibility of current existing devices, for instance, iPhone, Mac Laptop do not have white space radios to access white space gateway points. For a realistic mesh network, the end hop to the client need to be able to work in ISM Wifi band. 

% Paper focusing
This work focus on improving the wireless mesh network capacity combing with existing wifi nodes.
We present FIXME mesh node placement algorithms so as to minimize the deployed nodes, guarantee the coverage, and mesh node inter-connectivity.


% 

% Contribution
% problem/environment propose
% algorithms
% 
The main contributions of our work are as follows:
\begin{itemize}
\item We formulate the hetergeneous mesh node placement problem as a FIXME problem.  

\item We propose a methodology to leverage the infulence of white space in mesh networks.

\item We .

\item We perform extensive outdoor experiments from multiple environments and simulations to evaluate the proposed algorithms.


\end{itemize}



%The remainder of this paper is organized as follows. In Section II, we present the multiband adaptation problem and proposed algorithms. Section III discusses experimental evaluation of the multiband algorithms. We conclude in Section IV.

