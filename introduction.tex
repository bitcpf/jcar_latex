\section{Introduction}
\label{sec:introduction}

% Background multiband 
% Mesh Netrork
% Mesh Network traditonal utibility
% White space benefit challenges
% Contributions about new data
% Paper organization
% Wireless Network


% New Technology for network design brought by new policy
Since 2009, the FCC has approved the use of broadband services in the white spaces of 
UHF TV bands, which were formerly exclusively licensed to television broadcasters. These 
white space bands are now available for unlicensed public use, enabling the deployment 
of wireless access networks across a broad range of scenarios from sparse rural areas 
(one of the key applications identified by the FCC) to dense urban areas~\cite{carlson}. 
The white space bands operate in available channels from 54-806 MHz, having a far greater 
propagation range than WiFi bands for similar transmission power~\cite{balanis2012antenna}.
These new resources and policy offer opportunities for both the industry and academia. 

During the last decade, numerous cities solicited proposals from network carriers for exclusive 
rights to deploy city-wide WiFi, spanning hundreds of square miles. While the vast majority 
of the resulting used a wireless mesh topology, initial field tests revealed that the actual 
WiFi propagation could not achieve the proposed mesh node spacing. As a result, many network 
carriers opted to pay millions of dollars in penalties rather than face the exponentially-increasing
deployment costs (e.g., Houston~\cite{cnet_aug07} and Philadelphia~\cite{arstechnica_may08}). 
Thus, while a few mesh networks have been deployed in certain communities~\cite{CRSK06},
wireless mesh networks have largely been unsuccessful in achieving the scale of what was 
once anticipated~\cite{taps}. Specific to rural areas, the lack of user density and 
corresponding traffic demand per unit area as compared to dense urban areas allows greater 
levels of spatial aggregation to reduce the total number of required access points, lowering
network deployment costs. In densely populated urban areas, the greater concentration of users 
and higher levels of traffic demand can be served by maximizing the spatial reuse. 


The digital TV transition freed more spectrum for data networks with increased propagation 
range as compared to WiFi~\cite{balanis2012antenna} as white space sprctrum. These additional 
channel capacity vares in the line with FFC regulation and existing channel occupancy. As shown 
in Google database~\cite{googledatabase}, the number of available channels in these spectrum 
vary from city to city in US. Naturally, the question arises for improving the performance as 
well as the optimization of utilization: {\it how can the emerging white space bands improve 
large-scale mesh network deployments?}, {\it (ii) To what degree can white space bands reduce 
the network deployment cost of sparsely populated rural areas as opposed to comparable WiFi-only 
solutions?}, and {\it (iii) Where along the continuum of user population densities do the white 
space bands no longer offer cost savings for wireless network deployments?} While much work has 
been done on deploying multiple channels wireless networks, the differences in propagation among 
large scale frequencies have not been exploited in their models~\cite{tang2005interference, 
long2013fair,doraghinejad2014channel}, which could be {\it the} fundamental issue for the success 
of mesh networks going forward. 



% Paper topic problem and possible solution
In this paper, we perform a measurement based study which jointly considers the propagation 
characteristics and in-field spectrum availability of white space and WiFi channels to find 
a possible solution of wireless network deployment. Across varying population densities in 
representative rural and metropolitan areas, we compare the cost savings (defined in terms 
of number of access points reduced) when white space bands are not used. To do so, we first 
define the metric to quantify the spectrum utility in a given target location. With the in-field 
measured spectrum utility data in metropolitan and surrounding areas of Dallas-Fort Worth (DFW), 
we calculate the typical channel occupancy of WiFi and white space bands in multiple types of areas. 
Second, we propose a measurement-driven framework MAPE (Multiband Access Point Estimation) to find 
the number of access points required for access tier wireless network deployment in a target area 
with population densities from our measurements and census data. We then evaluate our measurement-driven 
framework, showing the band selection across downtown, residential and university settings in 
urban and rural areas and analyze the impacts of white space and WiFi channel combinations on a 
wireless deployment in these representative scenarios. We further leverage the diversity in propagation 
with channel occupancy of white space and WiFi bands in the planning and deployment of large-scale 
backhual tier wireless mesh network. To do so, we form an linear program to jointly exploit white 
space and WiFi bands for optimal WhiteMesh topologies in channel assignment. Then, since similar 
problem formulations have been shown to be NP-hard problem~\cite{jain2005impact,doraghinejad2014channel}, 
we build a heuristic measurement driven algorithm, Band-based Path Selection (BPS) based on mathematical 
analysis to solve the problem. We further apply the approaching method in multiple scenarios with 
in-field measurement data. Across a wide range of scenarios, including network size, population 
distribution, deployment distance gap, we exploit the general rules of emerging white space bands 
in mesh network. The performance of our scheme is compared against two well-known multi-channel, 
multi-radio channel assignment algorithms across these scenarios, including those typical for rural 
areas as well as urban settings. We further discuss the channel occupancy impacts on wireless networks 
and show the comparison of our algorithm and previous methods in typical scenarios. Finally, we quantify 
the degree to which the joint use of both band types can improve the performance of wireless mesh network.

% Need some calculation data to show the improvement
The main contributions of our work are as follows:
\begin{itemize}
\item We perform in-field measurements of spectrum utilization in various representative
scenarios across the DFW metroplex, ranging from sparse rural to dense urban areas and 
consider the environmental setting (e.g., downtown, residential, or university campus).
\item We develop a measurement-driven Multi-band Access Point Estimation (MAPE) framework 
to jointly leverage propagation and spectrum availability of white space and WiFi bands 
for wireless access network across settings and analyze the framework under capacity and 
coverage constraints to show that, with white space bands, the number of access points 
can be greatly reduced from WiFi-only deployments by up to 1650\% in rural areas. 
%Based 
%on the results, We quantify the impact of white space and WiFi channel combinations to 
%understand the trade-offs involved in choosing the optimal channel setting, given a certain 
%number of available channels from multiple bands.
\item We build a complexity reducing heuristic measurement driven algorithm, Band-based 
Path Selection (BPS), which considers the diverse propagation, overall interference level 
of WiFi and white space bands and further perform extensive analysis across offered loads,
network sizes, mesh nodes spacing and WiFi/white space band combinations, to compare against previous 
multichannel multiradio algorithms. 
%
%We develop a linear program model to jointly leverage white space and WiFi bands to 
%advantages and disadvantages in wireless mesh networks with measured channel occupancy. 
%Then, 
%
\item We study the channel occupancy and mesh spacing impacts on the performance given similar 
channel resources (bandwidth and transmission power), We show the improvement of our BPS in typical
configurations up to \%180 vs. previous multichannel algorithms.
\end{itemize}

The remainder of this paper is organized as follows. In Section
~\ref{sec:problemformulation}, we introduce the opportunites and challenge of diverse 
frequency band in wireless network deployment and then formulate the problem. 
In Section~\ref{sec:measurements}, we perform our measurements and analysis. 
In Section~\ref{sec:winmee}, we analyze the access tier network deployment, 
propose MAPE framework and show the results of white space bands gains. We 
develop a heuristic algorithms which consider which bands and multihop paths 
to select in a backhual white space topology and evaluate the performance of 
the heuristic algorithm versus the upper bound of the optimal solution and 
compare their performance against two well-known multi-channel, multi-radio 
algorithms in Section ~\ref{sec:whitemesh} in several scenarios and conclusions 
are drawn in Section~\ref{sec:conclusion}.


