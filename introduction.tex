\section{Introduction}
\label{sec:introduction}

% Background multiband 
% Mesh Netrork
% Mesh Network traditonal utibility
% White space benefit 
% Issues
% Paper organization

% Wireless Network

About a decade ago, numerous cities solicited proposals from
network carriers for exclusive rights to deploy city-wide WiFi,
spanning hundreds of square miles.  
While the vast majority of the resulting awarded contracts used
a wireless mesh topology, initial field tests revealed that the 
actual WiFi propagation could not achieve the proposed mesh node
spacing. As a result, many network carriers opted to pay millions of 
dollars in penalties rather than face the exponentially-increasing
deployment costs (e.g., Houston~\cite{cnet_aug07} and 
Philadelphia~\cite{arstechnica_may08}). Thus, while a few mesh 
networks have been deployed in certain communities~\cite{CRSK06,google_imc08},
wireless mesh networks have largely been unsuccessful in achieving 
the scale of what was once anticipated~\cite{taps}.

%http://news.cnet.com/8301-10784_3-9768759-7.html
%http://arstechnica.com/gadgets/2008/05/philadelphias-municipal-wifi-network-to-go-dark/

Around the same time, the digital TV transition created more
spectrum for use with data networks~\cite{fccwhitespace}. These white 
space bands operate in available channels from 54-806 MHz, having
increased propagation characteristics as compared to 
WiFi~\cite{balanis2012antenna}. Hence, the FCC has identified rural
areas as a key application for white space networks since the reduced
population from major metropolitan areas allows a greater service area
per backhaul device without saturating wireless capacity. 
As shown in Google database~\cite{googledatabase}, the number of available white
space frequency channels is deferent among cities all over US. 
And previous work~\cite{cuiwinmee} has shown the occupancy of 
frequency varies related to population distribution.
Naturally, 
the question arises for these rural communities as well as more dense 
urban settings: {\it how can the emerging white space bands improve 
large-scale mesh network deployments?}  While much work has been done 
on deploying multihop wireless networks with multiple channels and 
radios, the differences in propagation have not been exploited in their 
models~\cite{tang2005interference, si2010overview,long2013fair,doraghinejad2014channel}, 
which could be {\it the} fundamental issue for the success of mesh 
networks going forward.

% Paper topic
In this paper, we leverage the diversity in propagation of
white space and WiFi bands in the planning and deployment
of large-scale wireless mesh networks. To do so, we first form an
integer linear program to jointly exploit 
white space and WiFi bands for optimal WhiteMesh topologies.
Second, since similar problem formulations have been shown to be 
NP-hard~\cite{jain2005impact,doraghinejad2014channel}, we design a heuristic algorithm, 
Band-based Path Selection (BPS). We then
show the algorithm approaches the performance of the 
optimal solution but with a reduced complexity. 
We also involve in-field measured data to quantify the 
frequency occupancy in our algorithms.
To assess the 
% fixme try new compared methods
performance of our scheme, we compare the performance of 
BPS against two well-known multi-channel, multi-radio deployment 
algorithms across a wide range of scenarios, including those
typical for rural areas as well as urban settings. Finally, we
quantify the degree to which the joint use of both band types can improve the 
performance of wireless mesh networks.

The main contributions of our work are as follows:
\begin{itemize}
\item We analyze the white space bands application in wireless 
network deployment and develop an optimization framework based on integer
linear programming to jointly leverage white space and WiFi bands
to serve the greatest user demand in terms of gateway throughput 
in wireless mesh networks.  
\item We build an algorithm, Band-based Path 
Selection (BPS), which considers the diverse propagation 
and overall interference level of WiFi and white space bands using
a two-stage approach.  In the first stage, we prioritize the bands
with the greatest propagation to reduce the overall hop count. In the
second stage, we compare the interference level of path choices with
similar hop count. 
\item We perform extensive analysis across offered loads,
network sizes, and WiFi/white space band combinations, showing that BPS outperforms existing
multi-channel, multi-radio algorithms techniques by 3 and 6 times
in terms of the served user demand.  
\item Given similar channel resources
(bandwidth and transmission power), we additionally show that while
WiFi-only mesh topologies would largely outperform mesh networks with only
white space bands, the joint use of the two types of bands (i.e., WhiteMesh 
networks) can yield up to 170\% of the served user demand compared to mesh networks
with only one type of band.
\end{itemize}

The remainder of this paper is organized as follows. In Section ~\ref{sec:problemformulation}, 
we introduce WhiteMesh network topologies, describe the challenge of 
diverse frequency band allocation, and formulate the integer linear
programming model. In Section~\ref{sec:wmalgorithms}, we develop two heuristic algorithms which consider which bands 
and multihop paths to select in a WhiteMesh topology.  We then validate the performance of the heuristic algorithms 
versus the upper bound of the optimal solution and compare their performance against two well-known multi-channel, 
multi-radio algorithms in Section~\ref{sec:experimentdesign}. Finally, we discuss related work in Section~\ref{sec:related} 
and conclude in Section~\ref{sec:conclusion}.

% FIXME
%Then we design the experiment design in Section ~\ref{sec:experiment_design} and analyze the results in Section ~\ref{sec:results}.

%Wireless Mesh Network(WMN) is typically mentioned as a multihop wireless network consisting of a large number of wireless nodes, 
%some of which have wired connection as gateway nodes. 
%Many researchers paid attention to the potential applications of WMN, including last-mile broadband Internet access, last-mile smart grid terminals, neighborhood gaming, and so on ~\cite{bahl2004opportunities}.
%
%The bottleneck of the applications is that today's WMN still cannot offer enough capacity for customers.
%% Legal permission
%Since 2010, FCC adopted rules to allow unlicensed radio transmitters to operate in the white space bands freed from analog TV broadcast bands providing permission for operating wireless communication in White Space Bands~\cite{fccwhitespace}. 
%White Space Bands is popularly referred to unused portions of the UHF and VHF spectrum includes, but not limited to 54M-72MHz,76M-88MHz,174M-216MHz,470M-608MHz,and 614M-806MHz ~\cite{whitespacewiki}. 
%The FCC new policy gives opportunity to improve the performance of wireless mesh network.
%
%
%Generally, the maximum link-layer data rate falls quickly with increasing distance between the transmitter and receiver due to propagation ~\cite{balanis2012antenna}. 
%The bandwidth problem is further aggravated for multi-hop scenarios because of interference from its neighbor links. 
%Both of the issues could get benefit from the application of white space bands.
%White Space Bands in lower frequency have superior propagation and building penetration compared to licensed ISM bands which Wifi are working in, such as 2.4GHz and 5.8GHz. 
%These characters make White Space bands holding rich potential benefits for expanding wireless capacity and improving access ranges for wireless clients.
%% Wireless mesh networks
%The advantages of white space equipment Wireless mesh network could be straightly seen from their longer transmission range in reducing data rate falls and hop counts.
%White Space Bands is able to link further nodes through the lower frequency propagation characteristics.
%These links in white space bands bring data transmission for the nodes far from the gateway nodes in less hops reducing the relay nodes.
%The long distance link of white space is an economic way to provide back-haul Internet service in rare populated area.
%Also, as a kind of multi-radio architecture, Multi-band Multi-radio Mesh Network has the advantages having more bandwidth to increase the network capacity. 
%Benefits and challenges of ~\emph{White Space Bands} are from these two characteristics in channel assignment. 

%In this paper, the focus is on the heterogeneous multiband wireless network. 
%There are several issues in multiband wireless network, such as routing, gateway placement and routing.
%Our focus is on the channel assignment problem. 
%Channel assignment is to build virtual links among different nodes according to their radios. 
%The problem is has been proved as a NP-Hard problem.~\cite{jain2005impact}. 
%We put our efforts in the channel assignment problem in static centralized multiband wireless mesh network scenario. 
%It is the first step to resolve the problem in dynamic or distributed scenarios.
%In such networks, static nodes with multiple radio slots form a multihop backhual wireless access network that provides connectivity to end-users. The network nodes cooperate with each other to relay data traffic to gateway nodes who have wired connection. 
%Our hypothesis includes we have the node locations, available bands, path loss exponent, and virtual carrier sensing, etc. 
%Pearlman et. show that path load balancing provides negligible performance improvement, we limit each mesh node connect only one gateway nodes ~\cite{pearlman2000impact}.
%This work focus on the backhual layer, with the assumption the access layer is using different ISM channels from backhual layer.
%
%
%A channel assignment has to keep several constraints, such as connection, radio amount limitation, and so on.
%Previous work for Multi-Channel Multi-Radios scenarios is partial solution of Multi-band Multi-Radio problem.
%These works target on different objective with multiple methodology. Ashish ~\cite{raniwala2004centralized} use Load Aware Channel Assignment to approach the channel assignment optimization in network capacity; 
%Jian ~\cite{tang2005interference} employ a channel partition methodology to improve the interference of channel assignment; In ~\cite{jain2005impact}, Kamal describe the up-bound and lower-band based on Conflict Graph.
%Before FCC's new policy, these works fails to bring the the propagation difference of links in multiple bands to the topic.
%To solve the new problem, we analyze the architecture and propose our solutions for the problem.
%%Describe algorithms
%
%To better understand channel assignment problem, we propose a linear optimization model to describe constraints of such a network with more detailed constraints. 
%Though, this model is NP-Hard which could not be resolved in a polynomial time, it helps to understand the performance of a multi-band multi-radio wireless network. 
%To get an approaching channel assignment, we propose a Path Interference over Network (PIN) to evaluate each path in the network. Based on this parameter, we develop 2 novel channel assignment heuristic algorithms for the Multiband Multi-radio Wireless Mesh Network. 
%% Fix the name
%The first algorithm is a tree generated process to reduce hop count avoiding interference rooted at the gateway nodes. 
%The second algorithm starts from the furthest node and find path with lowest Path Interference over Network (PIN) to improve the performance.
%
%% Contribution
%% problem/environment propose
%% algorithms
%% 

