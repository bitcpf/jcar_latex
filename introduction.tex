\section{Introduction}
\label{sec:introduction}

% Background multiband 
% Mesh Netrork
% Mesh Network traditonal utibility
% White space benefit 
% Issues
% Paper organization

% Wireless Network
~\emph{Wireless Mesh Network}(WMN) is typically mentioned as a multihop wireless network onsisting of a large number of wireless nodes, some of which have wired connection as gateway nodes. Many researchers paid attention to the potential applications of WMN, including last-mile broadband Internet access, last-mile smart grid terminals, neighborhood gaming, and so on ~\cite{bahl2004opportunities}.
The bottleneck of the applications is that today's WMN still cannot offer enough capacity for customers. The maximum link-layer data rate falls quickly with increasing distance between the transmitter and receiver. The bandwidth problem is further aggravated for multi-hop scenarios because of interference from its neighbor links. Researchers have proposed several ~\emph{Multi-Radio} solutions for the bandwidth problem with the permission of using non-overlapping frequency channels simultaneously~\cite{raniwala2004centralized,tang2005interference,si2010overview}.
However, these works only focus on increasing the bandwidth by reducing the interference in WMN. ~\emph{White Space Bands} provide opportunities to improve the performance through both reducing interference and building long distance links. 

% White Space advantages
\emph{White Space Bands} is popularly referred to unused portions of the UHF and VHF spectrum includes, but not limited to 54M-72MHz,76M-88MHz,174M-216MHz,470M-608MHz,and 614M-806MHz ~\cite{whitespacewiki}. 
~\emph{White Space Bands} have superior propagation and building penetration compared to licensed ISM bands Wifi are working in such as 2.4GHz and 5.8GHz. These characters make ~\emph{White Space bands} holding rich potential benefits for expanding wireless capacity and improving access ranges for wireless clients.
% Legal permission
And since 2010, FCC adopted rules to allow unlicensed radio transmitters to operate in the white space bands freed from analog TV broadcast bands providing permission for operating wireless communication in ~\emph{White Space Bands}~\cite{fccwhitespace}. 

% Wireless mesh networks
The advantages of white space equipment could be seen straightly in ~\emph{Wireless mesh network}.
~\emph{White Space Bands} is able to link further nodes due to the lower frequency propagation characteristics.
These links in white space bands bring data rate for the nodes far from the gateway nodes in less hops reducing the relay nodes.
Obviously, these white space band long distance link is an economic way to provide back-haul Internet service in rare populated area.
Also, as a kind of multi-radio architecture, \emph{Multi-band Multi-radio Mesh Network} has the advantages of \emph{Multi-Channel Mesh Network} who has more bandwidth to increase the network capacity. 
Benefits and challenges of ~\emph{White Space Bands} are from these two characteristics in channel assignment. 

~\emph{Channel assignment} is to build virtual links among different nodes according to their radios. A channel assignment has to keep several constraints, such as connection, radio amount limition, and so on.
The problem is has been proved as a NP-Hard problem.~\cite{jain2005impact}. 
Previous work in ~\emph{Multi-Channel Multi-Radios} scenarios could be seen as partial solution of ~\emph{Multi-band Multi-Radio} problem without considering the propagation difference of links in multiple bands.
These works target on different objective with multiple methodology. Ashish ~\cite{raniwala2004centralized} use \emph{Load Aware Channel Assignment} to approach the channel assignment optimization in network capacity; Jian ~\cite{tang2005interference} employ a channel partition methodology to improve the interference of channel assignment; In ~\cite{jain2005impact}, Kamal describe the up-bound and lower-band based on ~\emph{Conflict Graph}.

% More benefit for mesh networks in different cities as chrismas story



% Paper topic
In this paper, the focus is on the problem of channel assignment in static centralized multiband wireless mesh network scenario. 
In these networks, static nodes with multiple radio slots form a multihop backhual wireless access network that provides connectivity to end-users. The network nodes cooperate with each other to relay data traffic to gateway nodes who have wired connection. 
Our hypothesis includes we have the node locations, available bands, path loss exponent, and virtual carrier sensing, etc. 
Pearlman et. show that path load balancing provides negligible performance improvement, we limit each mesh node connect only one gateway nodes ~\cite{pearlman2000impact}.
This work focus on the backhual layer, with the assumption the access layer is using different ISM channels from backhual layer.
%Describe algorithms
To get an approaching channel assignment, we propose a ~\emph{Path Efficiency over Network} (PEN) to evaluate each link and path in the network. Based on this parameter, we develop 2 novel channel assignment heuristic algorithms for the ~\emph{Multiband Multi-radio Wireless Mesh Network}. 
% Fix the name
The first algorithm is a tree generated process to reduce hop count avoiding interference. The second algorithm starts from the furthest node then iteration improve the \emph{Path Efficiency over Network} (PEN) to improve the channel assignment.
To better understand channel assignment problem, we propose a linear optimization model to describe constraints of such a network with more detailed constraints. Though, this model is NP-Hard which could not be resolved in a polynomial time, it helps to understandstand the performance of a multi-band multi-radio wireless network. 

% Contribution
% problem/environment propose
% algorithms
% 
The main contributions of our work are as follows:
\begin{itemize}
\item We formulate the heterogeneous ~\emph{Multiband Multi-radio} architecture with channel assignment problem and propose a ~\emph{Integer Linear Program} model for analyzing.  

\item We propose a  parameter ~\emph{Path Efficiency over Network} (PEN) to evaluate mixed ~\emph{Multiband Multi-hop Path}.

\item We develop 2 heuristic algorithms,1,2, to approach the channel assignment solution.
% Also should we consider the cost of different kind of nodes (1 radio, 2 radios, 3 radios )

\item We perform extensive simulation with maxinum throughput calculation to evaluate our algorithms.


\end{itemize}

The remainder of this paper is organized as follows. In Section ~\ref{sec:problemformulation}, we formulate the ~\emph{Multiband Wireless Mesh Network} architecture, describe the ~\emph{Channel Assignment} problem. We analyze multi-band multi-radio wireless mesh networks performance with variaty factors. 
In ~\ref{sec:linearopt} we represent the ~\emph{ILP} model and discuss its complexity.
In Section ~\ref{sec:wmalgorithms} discusses the mixed ~\emph{Multiband Multi-hop Path}, based on the analysis, we propose two heuristic algorithms approaching the optimal channel assignment of a multiband network. 
% FIXME
%Then we design the experiment design in Section ~\ref{sec:experiment_design} and analyze the results in Section ~\ref{sec:results}.

