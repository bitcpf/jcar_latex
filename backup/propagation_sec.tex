\section{Propagation in Multiband Mesh Network}
\label{sec:propagation}

In this section, we introduce the influence of propagation across bands in mesh network.

% Concept of propagation, factors of the environment and so on
Wireless propagation is the behavior of the signals loss characteritics when they are transmitted from one point to another.
The factors rule radio propagation are complex and diverse, and in most propagation models there are three basic propagation mechanisms: reflection, diffraction, and scattering ~\cite{andersen1995propagation}.
Wireless propagation could be affected by the daily changes of environment, weather, and atmosphere changes due to cosmos activities. 
Understanding the effects of varying conditions on wireless propagation has many practical applications for wireless network, from choosing frequencies, designing multihop routing, avoiding interference, to frequency reusing.

% propagation fomular, explain the band influence
Usually in wireless networks, the received signal power of a node is represented as 
$P_{dBm}(d)=P_{dBm}-10\alpha log_{10}(\frac{d}{d_0})+\epsilon$. 
Pathloss exponent $\alpha$ in outdoor environments range from 2 to 5, higher frequency has a heavier pathloss. \cite{camp2006measurement}. 
Based on the model propagation difference of frequency becomes an important characteristic in multiband network since the pathloss exponent varies with channel frequency.
The propagation difference makes the performance of radios vary from band to band in the same location with the same transmitting power configuration.

% Make multiband networks interesting 
Employing multiband gateway nodes bring 2 advantages for mesh network, 1) more bandwidth make the contention in the network lower,
 2) the propagation difference brings flexiable topology by increasing the communication range without more contention.

Specifying each link individually enables us to encode propagation characteristics of different bands $k$. In other words, idealy without considersing cost, each grid can be covered by an arbitrary node through specific band.  
The propagation alternation brings the advantages of providing more possible path for multihop network without increase interference on their neighbors. 

% Different environment
The path loss exponent varies from plane to mountains, from populated area to rural area make the question which bands should be chosen and how to assign the bands interesting.
A lower band node may have more coverage with a small path loss to decrease the cost of wireless network deployment. However a higher band may have more network capacity through frequency reuse.
These characteristics make the white spcae mesh node have a role to get the gain either in reduce the cost for network deployment or increase the network capacity.
Furthermore, FCC has different rule to balance white band utility in different areas and environment, the demand of evaluating the white band mesh network in different scenario for industry design request and FCC regulation is urgent.

To answer the question how to balance white space bandwidth and propagation range in a network.  
We choose the neighborhood area with different propagation characteristics and evaluate the FIXME algorithms to leverage the propagation influence for network deployment and network capacity.
