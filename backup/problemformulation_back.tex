\section{Problem Formulation}
\label{sec:problemformulation}

In this section, we formulate the problem of how to optimally use WiFi and white space bands 
in concert when deploying wireless mesh networks. We first introduce the multiband mesh 
network, system model and illustrate the challenges of such a WhiteMesh architecture in 
propagation and channel occupancy. As opposed to previous works such as
~\cite{tang2005interference,yuan2006cross,si2010overview}, we focus on both the access tier 
network deployment and backhual tier network design. 

% Background
\subsection{White Space Opportunity and Challenge}
\label{subsec:motivation}
% Propagation
Wireless propagation refers to the signal loss characteristics when wireless signals 
are transmitted through the wireless medium. The strength of the received signal 
depends on both the line-of-sight path (or lack thereof) and multiple other paths 
that result from reflection, diffraction, and scattering from obstacles~\cite{andersen1995propagation}. 
The widely-used Friis equation characterizes the received signal power $P_r$ in terms 
of transmit power $P_t$, transmitter gain $G_t$, receiver gain $G_r$, wavelength 
$\lambda$ of the carrier frequency, distance $R$ from transmitter to receiver, and 
path loss exponent $n$ according to~\cite{friis}:
\begin{equation}
\label{eq:friis}
P_r=P_t+G_t+G_r+10n \log_{10}\left( \frac{\lambda}{4\pi R}\right)
\end{equation}
Here, $n$ varies according to the aforementioned environmental factors with a value 
ranging from two to five in typical outdoor settings~\cite{rappaport}.

% Explain multiband vs multichannel Adding channel occupancy here, and also activity level
A frequency band is commonly defined as a group of channels which have similar propagation 
characteristics with small frequency separation. A common assumption in previous works that 
use many WiFi channels is that the propagation characteristics of each channel is similar 
to another, since the channel separation is relatively small (e.g., 5 MHz for the 2.4 GHz band).
% Previous work focus on multichannel
Many works which rely on such uniform propagation assumption have focused on the allocation 
of multiple WiFi channels with multiple radios in multihop wireless networks with channels in 
one band~\cite{doraghinejad2014channel}. However, as FCC licensed the white space bands for 
communication, the propagation variation has to be a consideration in wireless communication. 
Moreover, the FCC has flexible rules all over the states for the existing TV stations 
and devices working in white space bands~\cite{googledatabase}. 
The existing channel occupancy in the air has to be counted in the wireless communication~\cite{fallah2010congestion}. 

% Clearify the problem
% Network deployment objective
Wireless mesh networks are a particular type of multihop wireless network that provides access 
for users through wireless links with low cost. Naturally, in wireless network design, there is 
trade off between the budget and the quality of service. Typically, wireless mesh network is 
considered to have at least two tiers~\cite{CRSK06}: {\it (i)} an access tier, where client 
traffic is aggregated to and from mesh nodes, and {\it (ii)} a multihop backhaul tier for 
connecting all mesh nodes to the Internet through gateway nodes. To provide better service for 
more users, the network has to be designed with more access points in the access tier, and 
offers more capacity in the backhual tier. However, more access points and backhual capacity 
induce huge cost for network deployment. Fortunately, white space large propagation helps solving 
these issues. In this work, we explore the white space spectrum application for both access tier 
network deployment and further focus on optimally allocating white space and WiFi bands on a 
finite set of radios per mesh node along the backhaul tier.  

% Advantages
In the white mesh networks, the white space spectrum offers not only more allowed frequency resources
but also a larger service area for a single access point. With the flexible FCC restrictions and 
artificial activity diversity of white space spectrum, in sparsely-populated rural areas, the 
lower frequencies of the white space bands might be a more appropriate choice for access point 
deployment and connections for backhual networks. A white space access point has a larger service
area comparing with WiFi access point only counting propagation. Long propagation also reduce the 
hop counts in backhual networks with more capacity.  
% Drawbacks
However, as the population and demand scales up (e.g., for urban regions), the reduced spatial 
reuse and greater levels of interference of white space bands might detract from the overall wireless
network deployment strategy. In such urban areas, select links with more spatial reuse ability might
be a more appropriate choice, especially since the number of available channels in white space bands 
is often inversely proportional to the population (due to the existence of greater TV channels).




\begin{figure}
\vspace{-0.0in}
\centering
\includegraphics[width=84mm]{figures/interferencerange2}
\vspace{-0.1in}
\caption{Example WhiteMesh topology with different mesh-node shapes 
representing different frequency band choices per link.}
\label{fig:interferencerange}
\vspace{-0.2in}
\end{figure}




\subsection{System Model}
\label{subsec:problem}
% Talk about the challenges
Larger propagation range results in cost reducing of access points deployment in access 
tier deployment through larger service area. Figure \ref{fig:interferencerange} depicts 
an example of the propagation diversity in white mesh networks. The mesh node $A$ 
could connect to mesh node $C$ relayed by node $B$ at 2.4 GHz, or directly connect to $C$ at 450 
MHz. If 2.4 GHz were chosen, link $D,E$ is able to reuse 2.4 GHz when they are out of 
the interference range. However, in backhual tier network if link $A,C$ used 450 MHz, a 
lower hop count would result for the path, but lower levels of spatial reuse also result 
(e.g., for link $D,E$). While the issues of propagation, interference, and spatial reuse are 
simple to understand, the joint use of white space and WiFi bands to form optimal WhiteMesh 
topologies is challenging since the optimization is based on the knowledge of prior channel 
assignment which is not available before the work has been done.

% Need to tell the whole issue
The prior multi-channel models~\cite{tang2005interference, doraghinejad2014channel} fail to 
distinguish the propagation range $D_c$, interference range $D_r$ and channel capacity variation 
among frequency bands. While these works would attempt to minimize the interference in a multihop 
network topology by an optimal channel assignment for a given set of radios, we hypothesize that
using radios with a greater diversity in propagation could yield overall network performance gains
in both access tier and backhual tier networks. Therefore, in the white mesh model, for a given set 
of radios, we allow the channel choices to come from multiple frequency bands (i.e., multiband channel 
assignment, which also includes multiple channels in the same band). We assume all channels have 
the same bandwidth and the same channel capacity of each band with the same transmit power, channel 
bandwidth, and antenna gain under a classic protocol model~\cite{gupta2000capacity}. 

In sparsely-populated rural areas, the lower frequencies of the white space bands might be a better 
choice for wireless service in sparsely populated areas for its low utility. However, as the population 
and traffic demand scales up (e.g., for urban regions), the greater levels of residents traffic demands 
might detract the white space bands from the overall deployment strategy. That motivate the learning of 
target area electromagnetic environment in prior of the network design. For spectrum utility and 
resulting channel availability, we use a long-term measurement for each band.  We define the percentage 
of sensing samples ($S_\theta$) above an interference threshold ($\theta$) over the total samples ($S$) 
in a time unit as the activity level ($A$) of inter-network interference:
\begin{equation}
\label{eq:actdef}
A=\frac{S_\theta}{S_a}
\end{equation}

% Discuss the application of activity level
Based on the in-field measurements, we can further get the achieved channel capacity through the remaining
free time according to Eq.\ref{eq:intercap}:
\begin{equation}
\label{eq:intercap}
\delta_r=\delta*(1-\bar{A})
\end{equation}
The capacity of a clean channel is denoted by $\delta$ with the protocol model. In a joint white space and 
WiFi scenario, the activity level varies according to various interfering sources and the propagation 
characteristics induced by the environmental characteristics of the service area.

% Objective of the work,high level problem
A white mesh wireless network involves the propagation variation and channel occupancy additional to 
the previous multichannel works. Wireless networks plays a central role to connect clients to others.
Position in the carriers, the key issue of wireless network deployment is to provide affordable and 
quality service to the clients. In the two tiers of wireless network, there are variable objective in
the access tier and backhual network. For access tier deployment, the key issue is to reduce the cost 
for access points with quality of service constraints. In the backhual network, the objective of the 
deployment is to offer more capacity for the mesh nodes which are the access points in the access tier. 
% 2 sub problems
As induced, in wireless network deployment, we have to solve \emph{1. Access Tier Network and
2. Backhual Tier Network} deployment sub problems. 

% Previous works
\textbf{Access Tier Network} Despite sufficient levels of received signal, interference can cause channels 
to be unusable (e.g., due to high levels of packet loss) or unavailable (e.g., due to primary users in 
cognitive radios~\cite{haykin2005cognitive}). Prior work has worked to reduce interference levels via gateway 
deployment, channel assignment, and routing~\cite{he2008optimizing,tang2005interference}. The interference 
of a wireless network could be divided into two categories according to the interfering source: {\it (i)} 
intra-network interference, caused by nodes in the same network, and {\it (ii)} inter-network interference, 
caused by nodes or devices outside of the network. The definition of activity level provides a method to 
quantify the inter-network interference in the deployment design. We apply the measurements based activity 
level in both access tier network design and backhual network design.

% Network Constraint
Typically, the deployment of wireless access networks is subject to coverage and capacity constraints 
for a given region. Coverage is defined with respect to the ability of clients to connect to access 
points within their service area.  We use a coverage constraint ratio of $95\%$ in this work for a 
target area~\cite{robinson2010deploying}. Capacity is defined with respect to the ability of a network 
to serve the traffic demand of clients. Spatial reuse allows improved capacity but increases the cost
of deploying a network by increasing the total number of access points required. Hence, for densely-populated 
areas, the greatest level of spatial reuse possible is often desired. In contrast, sparsely-populated 
rural areas have lower traffic demand per unit area.  Thus, aggregating this demand with the use of lower 
frequencies via white space bands could be highly effective in reducing the total number of access points 
required to achieve similar coverage and capacity constraints.  Moreover, since less TV channels tend to 
be occupied in sparsely-populated areas~\cite{msdatabase}, a larger number of white space bands can be 
leveraged in these areas.

An access network deployment should ideally provide network capacity equal to the demand of the service 
area to maintain the capacity constraint. The demand of a service area could be calculated as the 
summation of individual demands all over the service area $D_a=\sum_{p\in P}D_p$. Since household demand 
for the Internet has been previously characterized~\cite{rosston2011household}, $D_a$ could represent the 
population distribution $f$ and service area $k$ as $D_a=\sum_{f \in F,k \in K}\bar{D_p}*f*k$. The capacity 
constraint could be represented with an access point set $M$ according to:
\begin{equation}
\label{eq:nlbound}
\sum_{m \in M}\delta_r^m \ge \sum_{f \in F,k \in K}\bar{D_p}*f*k
\end{equation}
At the same time, the wireless network must additionally satisfy the coverage constraint in the service 
area where the access points provide connectivity for client devices. Generally, a coverage of $95\%$ is 
acceptable for wireless access networks~\cite{robinson2010deploying}.


\textbf{Backhual Network} Most of the existing backhual works try to reduce the intra-network interference 
without regard to the inter-network interference~\cite{si2010overview,doraghinejad2014channel} and propagation
variation. To investigate the performance of white space bands in backhual wireless network, we consider the 
diverse propagation characteristics for four frequency bands: 450 MHz, 800 MHz, 2.4 GHz, and 5.8 GHz. The two 
former frequency bands are located in white space (WS) bands, whereas the two latter frequency bands are WiFi 
bands. Thus, in sparsely-populated rural areas, the lower frequencies of the white space bands might be a more 
appropriate choice for multihop paths to gateways having reduced hop count with more capacity. However, as the 
population and demand scales up (e.g., for urban regions), the reduced spatial reuse and greater levels of 
interference of white space bands might detract from the overall channel assignment strategy. In such urban 
areas, select links of greater distance might be the most appropriate choice for white space bands, especially 
since the number of available channels is often inversely proportional to the population (due to the existence 
of greater TV channels).

To distinguish the dynamics of white space bands, we employ the activity level based on multiple measurements 
in the target area to represent the achieved channel capacity. Correspondingly, a connectivity graph $C$ is 
formed for each band in $B$ such that $C=(V,L,B)$.  If the received signal for a given band is above an 
interference-range threshold, then contention occurs between nodes. We extend the conflict matrix in~\cite{tang2005interference} 
related to various interference per band according to $F=(E_{i,j},I_{Set},B)$, where $E_{i,j}$ represents the 
link and $I_{Set}$ includes all the links are physically inside the interference range $D_r$ when operating on 
each band $b \in B$.

Therefore, the problem we model is: to choose the connectivity graph $C'$ which maximizes the served traffic flow obey 
the constraints of multiband wireless network (defined below). A key challenge is that selecting the optimal channels from
the set $B$ leads to a conflict graph $F$ which cannot be known {\it a priory}. Previous works have proposed several 
coloring, cluster-independent set, mixed linear integer methodology for a single band $b$ 
~\cite{peng2012efficient,tang2005interference,doraghinejad2014channel}. 
However, these works do not address a reduction in hop count or an increase in spatial reuse and channel occupancy for 
a set of diverse bands $B$. 

% Metrics
In network application, the bottleneck of mesh network capacity has been shown to be the gateway's wireless 
connections~\cite{robinson2008adding}. The metric we use to evaluate the proposed algorithm is the served traffic flow.
Wireless networks are operated and maintained by vendors, such as AT\&T, T-Mobile, who charge the customers based on their data 
through Internet. In practice, the traffic demand of the user obey Poisson distribution\cite{saaty1961elements}, 
we then generated Poisson distributed numbers to represent clients' traffic demands in our simulation. The served traffic 
flow is correlated to the population of the area since each user has similar traffic demand in long term average. The 
employed performance metric of served traffic flow $X$, is represented the traffic arrived at the gateway nodes, where 
in Eq.~\ref{eq:goodput}:
\begin{equation}
\label{eq:goodput}
X=\sum_{w \in W, v \in V}T(w,v)
\end{equation}
The traffic arrived gateway node $w\in W$ considers all incoming and outgoing wireless traffic from access node $v\in V$ 
as $T$ onto the Internet. Obviously, the served traffic flow is also related to the routing and other factors, we use a 
simple routing method to keep the maximum the traffic arrived at the gateway nodes, the exact calculation of served traffic
flow is described in Section~\ref{subsec:wmanalysis}.


