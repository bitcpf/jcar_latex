\section{Experiment and Analysis}
\label{sec:experiment}

In this section, we introduce the measurements experiments setup and 
evaluate the process of probabilistic forecasting of channel state.
Based on the probabilistic forecasting result, we perform numeric
simulation of the Bayesian Beam Schedule. We further compare the 
performance of BBS to the equal time schedule and maximum channel 
capacity schedule and show the gain of the BBS with analysis.


% Subsec Experiment design
\subsection{Experiment Design}
\label{subsec:experimentdesign}

The BBS is a measurements driven method, espicelly for the probabilistic forecasting
of the channel states of each client. We perform experiments with an Android application, 
WiEye worldwide measurements. And also we perform small scale local measurements in typical downtown,
urban, neighborhood and campus area.

% Wieye
The application for the data collection is currently available for download and usage via the Google 
Android Marketunder the name WiEye. Users are able to view all Wi-Fi access points that are within range 
oftheir cellular device in both graphical and tabular form. All datacollection is done in the background, 
either continuously while theuser is running the application or periodically if the user has optedin to 
background data collection to assist our study. Our data collection has been approved by the Southern 
Methodist UniversityInstitutional Review Board, a human subjects research committee, ensuring that all 
ethical precautions have been taken in collectingdata from the users of our application.

Through this piece of data, we investigate the large scale beam schedule through BBS in a city.
FIXME

% Local Measurements 
We employ a Linux-based 802.11 testbed, which includes aGateworks 2358 board with Ubiquiti XR radios 
(XR9 at 900MHz, XR2 at 2.4 GHz, XR5 at 5.2 GHz) and a DoodleLabsDL475 radio at 450 MHz. We develop shell 
scripts whichutilize tcpdump to enable the testbed to work as a sniffer,recording all 802.11 packets. 
However, since the Gateworksplatform only updates its estimate of received signal strengthupon the 
reception of a new packet (and not all relevantchannel activity is 802.11 based), we employ a spectrumanalyzer 
to form a notion of inter-network interference withfiner granularity. Hence, we also use a Rohde \& Schwarz FSH 8 
portable spectrum works from 100 KHz to 8 GHz. Theportable spectrum analyzer is controlled by a Python 
script on a laptop to measure the received signal strength.

Through the local measurements, we explore the application in small scale which the measurements has more 
correlation with each other.
FIXME

\subsection{Numerical Simulation}
\label{subsec:result}

% Network setup, area, beams,traffic, payoff parameters
In the part, we set up a measurements driven simulation to evaluate the performance of the BBS framework.
And we further compare the performance with the non beam forming access, equal time schedule, and the post 
upper bound of beam schedule. In the non beam forming access, we omit the RTS/CTS time cost in the simulation.
Equal time schedule is to schedule the beam equally to all the clients in the served area. In the post upper
bound beam schedule, we already know the best channel state, in such situation, we can update the beam schedule
with the best options.
% Area Here
We choose 3 pieces of measurements data in the simulations. One from the measurements from a public park near
our campus, one from the WiEye data in a major city in US, the last one set is from the WiEye data in an 
urban area. Dense area with more traffic demands is more applicable for beam forming technology.

% Setup
In the first public scenario, we looked up the access points from Google(FIXME) and select the clients from the 
measurements we got from the area.


From the measurements, we analysis the activities across time and output the variation of time domain.
Large scale application

Small scale application

% Measurements and probabilistic forecasting




% Bayesian beam schedule




% Compare with other method









