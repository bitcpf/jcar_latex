\section{White Space Mesh Deployment}
\label{sec:model}

% Organization of the Sec
In this section, we first formulate the 
white mesh deployment problem
with propagation, connectivity and QOS constraint of multiband scenarios.
We then propose {FIXME} white mesh deployment algorithms for multiband multihop networks objectives.
%trageting to improve the network capacity with the same topology and reduce the mesh nodes for the same coverage of an area. 
For comparison, we also propose {FIXME} baseline assignment methods based on existing solutions.

\subsection{Problem Formulation}
% Objective
The objective of the mesh node placement problem in multiband scenario could either be 
to minimize the number of deployed mesh nodes and relay nodes with the constraint of fulfilling coverage of the target area and connectivity to the Internet existing users. 
or to maximize the capacity of existing mesh network topology with replacing ISM Wifi nodes to white space nodes.

% Input & Output
The set of potential gateway node accessing to Internet and relay nodes locations, $O$, is assumed know. 
Discrete locations for mesh nodes follows naturally from practical constraints on deployment, such as the availability of wired connection store or other infrastructure for gateway node installation.
The coverage area of the mesh network is discredited into small grid of coverage locations $N$. The center of each grid is the point output the received signal justifying the connection and channel capacity.
Mesh nodes $R$  are defined as nodes do not have direct Internet access but can connect to the mesh node and provide Wi-fi coverage of a number of grid.
Mesh nodes and gateway nodes could work in white space band and ISM Wifi band represented as bands set $B$.
The vertex set of the input is defined as $O=M\cup N$,the union of  coverage grid and the potential gateway nodes location. 
The solution of the problem could tell which gateway node will be built and where should a mesh node be added. The set of chosen gateway nodes is represented as $M_b$. 
The output of the problem is the nodes location and number of wireless nodes $R\cup M_b$, the union of gateway nodes chosen from potential locations and the mesh nodes. 


% Connectivity constraint
\subsection{Coverage Constraint}
\label{subsec:ccconstraint}
The connectivity graph represented as $G=(V,E)$ could indicate the existence of usable links. This formulation encodes the signal quality of each link independently can represent the link quality diversity across bands.
The links set, $E$, from the coverage grids $V$ to either relay nodes or mesh node $O$, corresponding to the estimated or measured signal strength is above a signal strength threshold. 
More formally, the connectivity graph where both target coverage locations and potential mesh node locations form a unified connectivity graph.
The mesh network have to satisfy the coverage of each grid through an edge $E$ from either a mesh node or a relay node $O$.
Each coverage grid $v_i\in {V}$ has $E_{B_j},{B_j}\in {B}$ valued ${0,1}$, which represents the signal strength is below threshold as 0 or above threshold as 1 in band $B_j$.
The sum of $E$ in band is also valued $(0,1)$, which means the grid has connection or not.
According to the definition of $E$, mapping to the connectivity constraint, it should satisfy $\sum{E}\geq V$.






\subsection{Mesh Network Capacity}
Wireless bandwidth is shared among all clients and mesh nodes, as a result, it is usually desirable to limit the number of potential shares of the scarce wireless spectrum. 
The link $E$ \ref{subsec:ccconstraint} could guarantee the minimum bandwidth sharing for the client, however, it could not represent the quality of the service. Our formulation enforces this by imposing a maximum degree $b_v=f(E,P_{rB})$ on the connectivity of a mesh node, represented the down-link/up-link capacity of the mesh node to clients. More complete capacity of formulations in other research take into account on-demands and fairness ~\cite{arkoulis2013optimal}, but we do not get into these scenarios in this paper.


% Note the capacity gain of putting white space nodes
Subject to the ~\emph{Coverage Constraint} \ref{subsec:ccconstraint}, the capacity of mesh network is the sum of bandwidth assigned for each grid $\sum{b_v}$. If the metric would be the capacity of mesh network, as the ISM Wifi mesh network has capacity $C_{ISM}$, white space mesh network has capacity $C_{ws}$, the capacity gain could be noted as $C_{ws}-C{ISM}$.

% Capacity-calculation
We carry the capacity calculation from ~\cite{robinson2008adding} of access networks where all traffic to and from clients must traverse a gateway. 
The calculation considers gateway multi-hop deferring to ongoing transmissions in contention range. 
The ~\emph{gateway limited fair capacity} ~\cite{robinson2008adding} is defined as a functioon of the airtime utilization of gateways, which depends on the routes used and amount of time the routes lead to a gateway deferring. From the definition, the capacity is affected by fairness. The calculation impose a fairness constraint of each mesh node to receive their fair share of the wireless airtime at the gateway nodes. 
We re-define the capacity calculation for single band in ~\cite{robinson2008adding} to adopt multiband scenario.

% Record the rough idea, need more care
Mesh node $i$ has a traffic demand $d[i]$ represents the aggregate demand of all the edn-clients associated with it. 
We present the routes used by each mesh node to reach one or more gateways as a 3 dimensional matrix $R$, where $R[i,j,k]$ indicates the amount of node $i$'s demand that traverses physical link $j$ on band $k$. 
$src(i)$ is designated as the access tier link for mesh node $i$ and assign $R[i,src(i)]=d[i]$. The calculation ensure fairness by requiring that $\lambda d[i]$ units of mesh nodes $ i$'s demand are served by gateways.
And $R$ matrix as solution to a transhipment problem optimizing capacity, potentially allowing multipath routing ~\cite{robinson2008adding}.
The contention caused by each physical link $j$ on band $k$ three-dimensional matrix $I$, where $I[i,j,k]$ indicates if physical link $j$  in contention range of node $i$ of band $k$.

A link induces contention equal to the number of mesh nodes that cannot be actively transmitting or receiving when the link is active in the band.
Contention is used as a simplification of interference only happens in the same band 

The total contention on a gateway node $g\in G$ caused by link $j$ in band $k$ is $\sum_{i=1}^nR[i,j,k] \times I[g,j,k]$. the total contention on gateway $g$, $v_g$ is then given by:
\begin{equation}
\label{eq:contention}
v_{gk}=\sum_{j=1}^m \sum_{i=1}^n R[i,j,k]\times I[g,j,k]
\end{equation}

Gateway $g$ services total demand $s_g$ which is the smu of demands on al links incident to gateway $g$, denoted as the link subset $link(gk)$ which has connection of the gateway node in band $k$:
\begin{equation}
s_{gk}=\sum_{j\in link(gk)} \sum_{i=1}^n R[i,j,k]\times I[g,j,k]
\end{equation}
 
The capacity of gateway $g$ as the amount of wireless time $v_g$ required to server $s_g$ units of time for internet service.
\begin{equation}
u_{g}=\sum_{k\in B} s_{gk}/v_{gk}
\end{equation}

% Add more introduce the characters of the calculation
The sum is a the worst case for mesh network due to double-counting of contention with the gateway.

% Just outlines need to add more
The capacity of a wireless is limited by its contention in worst case. 
Wireless network is able to increase the capacity through decreasing the contention or increase the contention to deploy less nodes for an arbitary network. 

Multiband network provide a solution of decreasing the contention and deploy less nodes simultaneously. 
The question comes out how could a mesh network approaching the objective has the lowest cost for construction subject to the capacity request or achieve the highest capacity under a certain budget.



% Note the gain of reducing mesh nodes
The gain of minimize deployment mesh node could be noted as the nodes amount of ISM Wifi mesh network $A_{ISM}$ minus the nodes amount of mesh network with white space nodes $A_{ws}$, $G_a=A_{ISM}-A_{ws}$.


To employ the propagation advantages brings a NP-hard problem to arrive the optimal solution~\cite{arkoulis2013optimal}. 
To approach the optimal solution we have FIXME frameworks to solve part of the problem subject to time fairness of each node.




