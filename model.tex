\section{Multiband Adaptation}
\label{sec:model}

% Organization of the Sec
In this section, we first formulate the 
white mesh deployment problem
with propagation, connectivity and QOS constrainr of multiband sceniario.
We then propose {FIXME} white mesh deployment algorithms for multiband multihop networks. For comparison, we also propose {FIXME} baseline assignment methods based on existing solutions.

\subsection{Problem Formulation}
% Objective
The objective of the mesh node placement problem in multiband scenario is to minimize the number of deployed mesh nodes and relay nodes with the constraint of full coverage of the target area and connectivity to the Internet existing users. 
% Input & Output
The set fo potential mesh node accessing to Internet locations, $M$, is assumed know. 
Discrete locations for mesh nodes follows naturally from pratical constraints on deployment, such as the availability of wired connection store or other infrastructure for mesh node installation.
The coverage area of the mesh network is discretized into small grid of coverage locations $N$. The center of each grid is the point output the received signal justifying the connection and channel capacity.
Relay nodes $R$  are defined as nodes do not have direcet Internet access but can connect to the mesh node and provide Wi-fi coverage of a number of grid.
The vertex set of the input is defined as $V=M\cup N$,the union of  coverage grid and the potential mesh nodes location. 
The solution of the problem could tell which mesh node will be built and where should a relay node be added. The set of chosen mesh nodes is represented as $M_b$. 
The output of the problem is the nodes location and number of wireless nodes $R\cup M_b$, the union of mesh nodes chosen from potential locations and the relay nodes. 

% Connectivity constraint
\subsection{Connectivity Constraint}
\label{subsec:ccconstraint}
The connctivity graph represented as $G=(V,E)$ could indicate the existence of usable links. This formulation encodes the signal quality of each link independently can represent the link quality diversity across bands.
The links set, $E$, from the coverage grids to either relay nodes or mesh node, corresponding to the estimated or measured signal strength is abouve a signal strength threshold. 
More formally, the connectivity graph where both target coverage locations and potential mesh node locations form a unified connectivity graph.
The mesh network have to satisfy the coverage of each grid through an edge $E$ from either a mesh node or a relay node.





\subsection{Propagation across Bands}
% FIXME Propagation of multiband
 For wireless networks, the received signal power of a node is represented as $P_{dBm}(d)=P_{dBm}-10\alpha log_{10}(\frac{d}{d_0})+\epsilon$. Pathloss exponent $\alpha$ in outdoor environments range from 2 to 5, higher frequency has a heavier pathloss. \cite{camp2006measurement}. 
The propagation of frequency becomes an important characteristic since the pathloss exponent varies with channel frequency. The propagation difference makes the performance of radios vary from band to band in the same location. 
Specifying each link individually enables us to encode non-uniform propagation. In other words, each grid can be covered by an arbitrary node.  
The propagation alternation brings the advantages of providing more possible path for multihop network without increase interference of their neighbors. 


\subsection{Capacity Constraint}
Wireless bandwidth is shared among all clients and mesh nodes, as a result, it is offen desirable to limit the number of potential sharers of the scarce wireless spectrum. 
The link $E$ \ref{subsec:ccconstraint} could gurantee the minimum bandwidth sharing for the client, however, it could not represent the quality of the service. Our formulation enforces this by imposing a maximum degree $b_v$ on the connectivity of a mesh node, represented the downlink/uplink capacity of the mesh node to clients. More complete capacity of formulations in other research take into account on-demands and fairness ~\cite{arkoulis2013optimal}, but we do not get into these scenarios in this paper.





To employ the propagation advantages brings a NP-hard problem to arrive the optimal solution~\cite{arkoulis2013optimal}. 
To approach the optimal solution we have FIXME frameworks to solve part of the problem subject to time fairness of each node.




