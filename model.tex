\section{White Space Mesh Deployment}
\label{sec:model}

% Organization of the Sec
In this section, we first formulate the 
white mesh deployment problem
with propagation, connectivity and QOS constraint of multiband scenarios.
Then introduce the \emph{Protocol Model} gateway-limited fair capacity of a white band mesh network modified from previous work ~\cite{robinson2008adding}. 
For ease of discussion, all capacity points are refered as "gateways" no matter how they get access to internet.

In this paper, we consider two-tier wireless mesh networks, consisting of an access tier for connection between mesh nodes and clients, and a backhaul tier for interconnection between mesh nodes. 
Since most of the clients have only ISM band wireless devices, we assume all the mesh node has the capacity in ISM band for clients. And for the backhaul tier mesh nodes have multiband capacity with different click range according to the propagation in each band.

Further, we focus on a multiradio, multiband backhual and access tier architecture. We let the user-specified cost of installing a physical wire or dedicated wireless connection be different for each location and allow non-uniform capacities at each location.
 

\subsection{Problem Formulation}
% Objective
The objective of the wireless mesh node placement problem in multiband scenario is refered to
minimize the number of deployed gateway nodes related to budget and mesh nodes adding capacity for existing network. 
or to maximize the capacity of existing mesh network topology by given specified budget for adding multiradio capacity points in existing network.

% Input & Output
It is defined as follows.
The set of potential gateway node locations, $W$, is assumed know. 
Let \emph{G} be a $k$ dimension binary $(0,1)$-vector of size $n$ that indicates whether a given mesh node $i$ in band $k$ is a capacity point or not. 
Discrete locations for mesh nodes follows naturally from practical constraints on deployment, such as the availability of wired connection or other infrastructure for gateway nodes installation.
On an operational multiband mesh network, $G[i,k]=1$, for all $i\in \emph{G}$, having active radio working in band $k$. Let the monetary cost of installing a capacity point be $f[i]+r(i)\times \sum_kG[i,k]$, $f(i)$ represent the cost for infrastructure, $r(i)$ is the cost for a single band, and \emph{$G_0$} represent the currently deployed capacity points.
We define the total cost, $C(G)$, of installing new capacity points in the mesh network as:

\begin{equation}
\label{eq:cost}
C(G)=\sum_{\forall i\notin G_0} (f(i)+r(i)\times \sum_kG[i,k]) \times sgn\{\sum_kG(i,k)\}
\end{equation}





\subsection{Mesh Network Capacity}
\label{subsec:capacity}
Wireless bandwidth is shared among all clients and mesh nodes, as a result, it is usually desirable to limit the number of potential shares of the scarce wireless spectrum. 
In this work the capacity calculation based on ~\emph{Protocol Model}, a node can use the channel if the distance between the node and next relay nodes is less than the threshold and other nodes in the click is not transmitting. The model is good for 802.11-style MAC ~\cite{jain2005impact}.

The capacity calculation in this paper considers access networks where all traffic to and from clients must traverse a gateway, making the gateways bottlenecks in the network. 
Therefore, the performance of gateway nodes represent the capacity of the network. 
More complete capacity of formulations in other research take into account on-demands and fairness ~\cite{arkoulis2013optimal}, but we do not get into these scenarios in this paper.
The advantages of this model are 1) exact computation in polynomial time and 2) extension to local search algorithms by enabling tractable approximations which optimize over one of two components of capacity definition:route lengths or contention ~\cite{robinson2008adding}.

% Capacity-calculation
We carry the capacity calculation idea from ~\cite{robinson2008adding} to model the wireless interface of gateway as alternating its time between transmitting to one-hop neighbors, receiving from one-hop neighbors, and deferring to other neighbors within contention range. 
The ~\emph{gateway limited fair capacity} is a functioon of the airtime utilization share of gateways, which depends on the routes used and amount of time the routes lead to a gateway deferring. From the definition, the capacity represent fairness for all nodes in the network. 
The calculation impose a fairness constraint of each mesh node to receive their fair share of the wireless airtime at the gateway nodes. 
For multiband scenario we re-define the expression of the capacity calculation.

% Record the rough idea, need more care
Mesh node $i$ has a traffic demand $d[i]$ represents the aggregate demand of all the edn-clients associated with it. 
We present the routes used by each mesh node to reach one or more gateways as a 3 dimensional matrix $R$, where $R[i,j,k]$ indicates the amount of node $i$'s demand that traverses physical link $j$ on band $k$. 
$src(i)$ is designated as the access tier link for mesh node $i$ and assign $R[i,src(i)]=d[i]$. The calculation ensure fairness by requiring that $\lambda d[i]$ units of mesh nodes $ i$'s demand are served by gateways.
And $R$ matrix as solution to a transhipment problem optimizing capacity, potentially allowing multipath routing ~\cite{robinson2008adding}.
The contention caused by each physical link $j$ on band $k$ three-dimensional matrix $I$, where $I[i,j,k]$ indicates if physical link $j$  in contention range of node $i$ of band $k$.

A link induces contention equal to the number of mesh nodes that cannot be actively transmitting or receiving when the link is active in the band.
Contention is used as a simplification of interference only happens in the same band 

The total contention on a gateway node $g\in G$ caused by link $j$ in band $k$ is $\sum_{i=1}^nR[i,j,k] \times I[g,j,k]$. the total contention on gateway $g$, $v_g$ is then given by:
\begin{equation}
\label{eq:contention}
v_{gk}=\sum_{j=1}^m \sum_{i=1}^n R[i,j,k]\times I[g,j,k]
\end{equation}
We assume the access tier and backhaul tier in the same ISM band choose different channels to avoid interference among the two tiers. For the backhaul tier the contension only comes from other mesh nodes.

Gateway $g$ services total demand $s_g$ which is the smu of demands on al links incident to gateway $g$, denoted as the link subset $link(gk)$ which has connection of the gateway node in band $k$:
\begin{equation}
s_{gk}=\sum_{j\in link(gk)} \sum_{i=1}^n R[i,j,k]\times I[g,j,k]
\end{equation}
 
The capacity of gateway $g$ as the amount of wireless time $v_g$ required to server $s_g$ units of time for internet service, also a gateway node will provide capacity to the area it located in.
\begin{equation}
u_{g}=\sum_{k\in B} s_{gk}/v_{gk}+R[g,k]
\end{equation}

% Add more introduce the characters of the calculation
The sum is a the worst case for mesh network due to double-counting of contention with the gateway.
% Just outlines need to add more
Wireless network is able to increase the capacity through decreasing the contention or increase the contention to deploy less nodes for an arbitary network. 
Multiband network provide potential solutions of decreasing the contention and deploy less nodes simultaneously. 
The question comes out how could a mesh network approaching the objective has the lowest cost for construction subject to the capacity request or achieve the highest capacity under a certain budget.


% Note the gain of reducing mesh nodes
The gain of minimize deployment mesh node could be noted as the nodes amount of ISM Wifi mesh network $A_{ISM}$ minus the nodes amount of mesh network with multiband nodes $A_{ws}$, $G_a=A_{ISM}-A_{ws}$.

To employ the propagation advantages brings a NP-hard problem to arrive the optimal solution~\cite{arkoulis2013optimal}. 
To approach the optimal solution we have FIXME frameworks to solve part of the problem subject to time fairness of each node.

