\section{Multiband Adaptation}
\label{sec:model}

% Organization of the Sec
In this section, we first formulate the multiband 
multihop problem 
in multiband networks and introduce the context information 
that we use. We then propose {FIXME} multiband channel assignment algorithms for multihop networks. For comparison, we also propose {FIXME} baseline assignment methods based on existing solutions.

\subsection{Problem Formulation}
% The problem we are going to resolve is to find a band has the best throughput in multiple available options as:
% Try to express the problem in math words
Consider a network make of $\{1,2,\ldots,m\}$ nodes equipped radios of $n$ frequency bands, represented by an index set~$\{1,2, \ldots, n\}$. 
% Time sharing
Each node will share a frequency band with neighborhood nodes in time $T_s{mn}$, means a node can use this part of the time slot for transmitting. Each node in the network has the right to share the wireless resource with its neighbors, if a number of nodes in the same click of a band, they should share the channel of the band equally in time occupancy. For instance, if in a click there are 3 nodes all fully backlogged, each of them should have one third time sharing of this band. The time fairness is what we hold for all nodes in channel assignment. 
% Fully backlogged source nodes
We assume the source nodes are fully backlogged. 
% Objective
The objective of the work is to achieve higher thoughput through assigning the bands to the nodes subject to fair time sharing,~$T_s{best}$,
% Factors may influence throughput 
In a network, the throughput~$r_m$ of a node~$m$ depends on several factors, such as in a single band $i$, the received signal power~$P_R^{mi}$, noise power~$P_N^mi$, the activity/occupancy level~$B^i$, the time sharing of a band, and other factors such as location and contextual information. 
This relationship is represented in general as~$r_m =\sum_n{f(P_R^i,P_N^i,B^i,context-information)}$. The objective can be stated as
\begin{equation}
T_s\{best\}= \arg \max_{T_s} r_m 
\end{equation}

% FIXME Propagation of multiband
For a multiband wireless network, 
The received signal power is represented as $P_{dBm}(d)=P_{dBm}-10\alpha log_{10}(\frac{d}{d_0})+\epsilon$. Pathloss exponent $\alpha$ in outdoor environments range from 2 to 5, higher frequency has a heavier pathloss. \cite{camp2006measurement}. 
The propagation of frequency becomes an important characteristic since the pathloss exponent varies with the wavelength. The propagation difference makes the performance of radios vary from band to band in the same location. The propagation alternation brings the advantages of providing more possible path for multihop network without increase interference of their neighbors. 
However, to employ the propagation advantages also bring a NP-hard problem to arrive the optimal solution~\cite{arkoulis2013optimal}. To approach the optimal solution we have FIXME frameworks to solve part of the problem subject to time fairness of each node.

% Factors in a single band
The framework allows us to 
separate the interference from other nodes using the same technology via the busy time and the interference from nodes using other technologies in the same and via the noise level~$P_N^i$. 
For instance, an 802.11 node can observe the packets of other 802.11 nodes but only the increase noise levels from other Zigbee/Bluetooth nodes.
The existing pattern embedded in the historical data connecting 
the performance of different bands and collected context information 
e.g.,  $B^i$, $P_N^i$, $P_R^i$ 
could be extracted and help make decisions
for multiband adaptation in a similar context~\cite{meikle2012global}.

% Defination of busy time FIXME should we keep the busy time?
To represent the utilization level of the channel, we define \emph{busy time}, $B$,
as the percentage of time when the channel is occupied by 
all competing sources $x_j ( j = 1, 2, 3, ...)$ other than the intended transmitter $y$. 
For 802.11-based transmissions, the busy time on band $i$ is defined as:
\begin{equation}
\label{eqn:80211activity}
B^i = \frac{\sum_j{\sum_k{\frac{L_k^{x_j}}{R_k^{x_j}}}}}{\sum_k{\frac{L_k^y}{R_k^y}}+\sum_j{\sum_k{\frac{L_k^{x_j}}{R_k^{x_j}}}}+S\sigma}
\end{equation}
where $L_k^{x_j}$ and $R_k^{x_j}$ represent, respectively, the packet length in bits and the data
rate at which that packet is transmitted, for external sources $x_j$;
$S$ and $\sigma$ are the idle slots and the slot duration. 
When considering the activity level of non-802.11 users 
(e.g., the bands currently licensed to TV and other non-802.11 devices), 
%whether the signal level from these
we use the received signal level from non-802.11 interference sources $P_N^i$ 
on band $i$ as an input to our algorithms. 

\subsection{Multiband Channel Assignment Algorithms}
% Framework introduction
In order to evaluate the proposed multiband adaptation algorithms, 
we construct two baseline multiband channel assignment methods: (1) We assign the
time sharing randomly to the nodes. (2) For each node, we try to transmit in all the time without considering the interferance of the networks: 

% FIXME should we keep this?
%% Throughput calculation
%The way we simulate the throughput of a single band through the 
%\begin{align}
%&\max_i T_{ideal}^i\times(1-B^i),
%\label{eq:baseline2}
%\end{align}
%The throughput $T_{ideal}$ is measured with Azimuth ACE-MX channel emulator~\cite{AzimuthACE}. 
%The details of system setup and data collection are discussed in Section~\ref{sec:experiment design}. 

% Framework1 start from assigning the highest band



% Framework2 start from assigning the loweset band



% Framework3 assign all the bands at one time


