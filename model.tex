\section{White Space Mesh Deployment}
\label{sec:model}

% Organization of the Sec
In this section, we first formulate the 
white mesh deployment problem
with propagation, connectivity and QOS constraint of multiband scenarios.
Then introduce the \emph{Protocol Model} gateway-limited fair capacity of a white band mesh network modified from Robinson's work ~\cite{robinson2008adding}. 
For ease of discussion, all capacity points are refered as "gateways" no matter how they get access to internet.

In this paper, we consider wireless mesh networks as two-tiers:
Consisting of an access tier for clients to mesh nodes, and a backhaul tier for interconnection from mesh nodes to gateway nodes. 
Most of the clients devices work in ISM band, such as iPhone, laptops, we assume all the mesh node has the capacity in ISM band for clients in access tier. And for the backhaul tier mesh nodes have multiband capacity with different click range according to the propagation in each band.

Further, we focus on a multiradio, multiband backhual tier architecture. We let the user-specified cost of installing a physical wire or dedicated wireless connection be different for each location and allow non-uniform capacities at each location.
 
\subsection{Problem Formulation}
% Objective
The objective of the wireless mesh node placement problem in multiband scenario is refered to
minimize the number of deployed gateway nodes to provide internet service according to the requirement for the clients. 

% Input & Output
The problem could be formulated as follows.
The set of node locations, $N$, are known as mesh nodes due to the construction limitation, such as power supply limitation or policy. Gateway nodes are selected from these mesh nodes.
In the target areas, $B$ represents the bands across ISM and white space band according to FCC policy.
Given the demand of mesh nodes $N$ for the service area as up link demand $UD$ of and down link demand $DD$. And also the the link capacity between two mesh nodes $R$ on band $B$ according to the \emph{Protocol Model} and propagation character of each band.
The cost of connect internet to mesh node making it a gateway node is given as $PT$, and the cost of installing radio of a band is given as $PC$.
The target of the problem is to minimize the total cost of the network construction, including the gateway connection cost and new radio installing cost.

Generally, the problem is a NP-hard problem only we traverse all the possible combination. To approach the optimization solution, we have a linear model to resolve the location and routing assignment of band and traffic at the same time.

 
Let \emph{G} be a binary $(0,1)$-vector of size $n$ that indicates whether a given mesh node $i$ is built as a capacity point or not. 
Discrete locations for mesh nodes follows naturally from practical constraints on deployment, such as the availability of wired connection or other infrastructure for gateway nodes installation.
On an operational multiband mesh network, $G[i,k]=1$, for all $i\in \emph{G}$, having active radio working in band $k$. Let the monetary cost of installing a capacity point be $f[i]+r(i)\times \sum_kG[i,k]$, $f(i)$ represent the cost for infrastructure, $r(i)$ is the cost for a single band, and \emph{$G_0$} represent the currently deployed capacity points.
We define the total cost, $C(G)$, of installing new capacity points in the mesh network as:

\begin{equation}
\label{eq:cost}
C(G)=\sum_{\forall i\notin G_0} (f(i)+r(i)\times \sum_kG[i,k]) \times sgn\{\sum_kG(i,k)\}
\end{equation}





\subsection{Mesh Network Capacity}
\label{subsec:capacity}
Wireless bandwidth is shared among all clients and mesh nodes, as a result, it is usually desirable to limit the number of potential shares of the scarce wireless spectrum. 
In this work the capacity calculation based on ~\emph{Protocol Model}, a node can use the channel if the distance between the node and next relay nodes is less than the threshold and other nodes in the click is not transmitting. The model is good for 802.11-style MAC ~\cite{jain2005impact}.

The capacity calculation in this paper considers access networks where all traffic to and from clients must traverse a gateway, making the gateways bottlenecks in the network. 
Therefore, the performance of gateway nodes represent the capacity of the network. 
More complete capacity of formulations in other research take into account on-demands and fairness ~\cite{arkoulis2013optimal}, but we do not get into these scenarios in this paper.
The advantages of this model are 1) exact computation in polynomial time and 2) extension to local search algorithms by enabling tractable approximations which optimize over one of two components of capacity definition:route lengths or contention ~\cite{robinson2008adding}.

% Capacity-calculation
We carry the capacity calculation idea from ~\cite{robinson2008adding} to model the wireless interface of gateway as alternating its time between transmitting to one-hop neighbors, receiving from one-hop neighbors, and deferring to other neighbors within contention range. 
The ~\emph{gateway limited fair capacity} is a functioon of the airtime utilization share of gateways, which depends on the routes used and amount of time the routes lead to a gateway deferring. From the definition, the capacity represent fairness for all nodes in the network. 
The calculation impose a fairness constraint of each mesh node to receive their fair share of the wireless airtime at the gateway nodes. 
For multiband scenario we re-define the expression of the capacity calculation.

% Record the rough idea, need more care
Mesh node $i$ has a traffic demand $d[i]$ represents the aggregate demand of all the edn-clients associated with it. 
We present the routes used by each mesh node to reach one or more gateways as a 3 dimensional matrix $R$, where $R[i,j,k]$ indicates the amount of node $i$'s demand that traverses physical link $j$ on band $k$. 
$src(i)$ is designated as the access tier link for mesh node $i$ and assign $R[i,src(i)]=d[i]$. The calculation ensure fairness by requiring that $\lambda d[i]$ units of mesh nodes $ i$'s demand are served by gateways.
And $R$ matrix as solution to a transhipment problem optimizing capacity, potentially allowing multipath routing ~\cite{robinson2008adding}.
The contention caused by each physical link $j$ on band $k$ three-dimensional matrix $I$, where $I[i,j,k]$ indicates if physical link $j$  in contention range of node $i$ of band $k$.

A link induces contention equal to the number of mesh nodes that cannot be actively transmitting or receiving when the link is active in the band.
Contention is used as a simplification of interference only happens in the same band 

The total contention on a gateway node $g\in G$ caused by link $j$ in band $k$ is $\sum_{i=1}^nR[i,j,k] \times I[g,j,k]$. the total contention on gateway $g$, $v_g$ is then given by:
\begin{equation}
\label{eq:contention}
v_{gk}=\sum_{j=1}^m \sum_{i=1}^n R[i,j,k]\times I[g,j,k]
\end{equation}
We assume the access tier and backhaul tier in the same ISM band choose different channels to avoid interference among the two tiers. For the backhaul tier the contension only comes from other mesh nodes.

Gateway $g$ services total demand $s_g$ which is the smu of demands on al links incident to gateway $g$, denoted as the link subset $link(gk)$ which has connection of the gateway node in band $k$:
\begin{equation}
s_{gk}=\sum_{j\in link(gk)} \sum_{i=1}^n R[i,j,k]\times I[g,j,k]
\end{equation}
 
Previous work treat every link equally with ~\emph{Capacity Gain} with concurrent activite links ~\cite{amiri2010utility}, however, in mesh networks, the links close to the gateway nodes are bottle neck of the network ~\cite{robinson2008adding}.
The capacity of gateway $g$ as the amount of wireless time $v_g$ required to server $s_g$ units of time for internet service, also a gateway node will provide capacity to the area it located in.
\begin{equation}
u_{g}=\sum_{k\in B} s_{gk}/v_{gk}+R[g,k]
\end{equation}

% Add more introduce the characters of the calculation
The sum is a the worst case for mesh network due to double-counting of contention with the gateway.
% Just outlines need to add more
Wireless network is able to increase the capacity through decreasing the contention or increase the contention to deploy less nodes for an arbitary network. 
Multiband network provide potential solutions of decreasing the contention and deploy less nodes simultaneously. 
The question comes out how could a mesh network approaching the objective has the lowest cost for construction subject to the capacity request or achieve the highest capacity under a certain budget.


% Note the gain of reducing mesh nodes
The gain of minimize deployment mesh node could be noted as the nodes amount of ISM Wifi mesh network $A_{ISM}$ minus the nodes amount of mesh network with multiband nodes $A_{ws}$, $G_a=A_{ISM}-A_{ws}$.

To employ the propagation advantages brings a NP-hard problem to arrive the optimal solution~\cite{arkoulis2013optimal}. 
To approach the optimal solution we have FIXME frameworks to solve part of the problem subject to time fairness of each node.



\subsection{Mixed Integer Linear Programming Formulation}
\label{subsec:linearopt}

We now present a mixed integer, linear programming formulation for 
optimizing gateway goodput when selecting channels
for WhiteMesh topologies across diverse bands. We assume that the
set of available mesh nodes $V$ , gateways $W$ and available 
bands $B$ are given.  The communication links and conflict graph 
are given as parameters.
%the ~\emph{Multiband Multi-Radio} wireless mesh network described in section ~\ref{sec:problemformulation} to model the problem and provide a way to approach the upbound a network throughput achieve gateways.
% Fixme if with more input it is still NP-hard
%Assume that we are given the nodes and available bands as the variable set. The communication links and conflict graph are given as parameters.

\noindent
{\bf Sets:}
\begin{tabular}{ll}
$V$ & set of nodes \\
$B$ & set of bands \\
\end{tabular}

\noindent
{\bf Parameters:}\\
\\
%\vspace{0.1in}
%\begin{tabular}{lll}
\begin{tabular}{llp{3.4cm}}
%\hline
$\gamma_{i,j}^b$ & $(i,j)\in V, b \in B$ & capacity of link $i,j$ on band $b$\\
%\hline
%\end{tabular}\\
%\begin{tabular}{llp{2.8cm}}
$I_{ij,lm}^b$ & $(i,j,l,m) \in V, b\in B $ & Interference of link $(i,j)$ on band $b$\\
%\hline
%\end{tabular}\\
%\begin{tabular}{llp{2.8cm}}
$W_i$ & $i \in V\ binary$ & Gateways in network\\
%\hline
%\end{tabular}\\
%\begin{tabular}{llp{2.8cm}}
$D_{di}$ & $i \in V\ $ & Downlink demand of node i\\
%\hline
%\end{tabular}\\
%\begin{tabular}{llp{2.8cm}}
$D_{ui}$ & $i \in V\ $ & Uplink demand of node i\\
%\hline
\end{tabular}

% We could vary the objective
% treat each mesh node with the same demand even generally the demand of the mesh node is random. So the goodput of a integer linear program is the summation of all the demand served by the gateway nodes. We assign a uplink demand variable $\lambda u$ and downlink demand $\lambda d$ to each node. The goodput of the network could be represented as $\sum_{n \in V}(\lambda u_n+ \lambda d_n)$, the linear program is givin to $Maximize\ Goodput$.

We define time share to represent the percentage of time a 
single link transmits according to~$\alpha_{i,j}^b$
for link $i,j$ in band $b$. Two flow 
variables are defined as uplink and downlink flows as below:
%on 
%a link $i,j$ for node $k$ in band $b$, $uy_{i,j,k}^b,dy_{i,j,k}^b$.

\noindent
%\vspace{2pt}
{\bf Variables:}\\
\\
%\vspace{1pt}
\begin{tabular}{llp{3cm}}
$0\le \alpha_{ij}^b \le 1$  & $b\in B, (i,j) \in N$ & 
Time share of link $(i,j)$ on band $b$\\ 
$0\le uy_{i,j,k}^b$ & $(i,j,k) \in V, b \in B$ & 
Uplink flow of node $k$ on link $(i,j)$ at band $b$ \\ 
$0\le dy_{i,j,k}^b$ & $(i,j,k) \in N, b \in B$ & 
Downlink flow of node $k$ on link $(i,j)$ at band $b$ \\ 
\end{tabular}

%\vspace{3pt}
Our objective is to maximize the gateway goodput.

\noindent
{\bf Objective:}
\begin{align}
& Max \sum_i\sum_j\sum_k\sum_b(uy_{i,j,k}^b+dy_{j,i,k}^b) \; When \; w_j=1
\end{align}

The constraints for the variables are represented as:  
%\setcounter{equation}{0}\\
%\vspace{1pt}
%Objective:
%\begin{align}
%\max \quad
%& \sum_{i \in N}(\lambda u_i+ \lambda d_i)
%\end{align}\\

\noindent
%{\bf Constraints:}
{\bf Connectivity Constraints:}
\begin{align}
\label{opt:1}
& \alpha_{i,j}^b + \alpha_{j,i}^b + \sum_l\sum_m(\alpha_{l,m}^b \cdot I_{ij,lm}^b) \leq 1, i\neq j \\
\label{opt:2}
& \sum_i uy_{i,j,k}^b + \sum_i dy_{i,j,k}^b \leq r_{j,k}^b \cdot \alpha_{j,k}^b 
\end{align}
\noindent
{\bf Uplink Constraints:} 
\begin{align}
\label{opt:3}
& \sum_k \sum_b uy_{i,i,k}^b \leq D_{ui} \; When \; w_k=0, i \neq k \\
\label{opt:4}
& uy_{i,j,k}^b = 0 w_k=1 \\
%\label{opt:5}
%& \sum_i\sum_b uy_{i,j,k}^b - \sum_m\sum_b uy_{j,m,k}^b = 0 \; When \; w_k=0, i\neq k\\
\label{opt:6}
& \sum_i\sum_b uy_{i,j,k}^b = \sum_m \sum_b uy_{j,m,k}^b \; When \; w_k=0, i \neq k\\
\label{opt:7}
& uy_{i,j,i}^b=0 
\end{align}
\noindent
{\bf Downlink Constraints:} 
\begin{align}
{\bf}
\label{opt:8}
& \sum_j \sum_b dy_{i,j,i}^b \leq D_{di} \; When \; w_i=0 \\
\label{opt:9}
& dy_{i,j,k}^l =0 \; When \; w_k=1 \\
%\label{opt:10}
%& \sum_j\sum_b dy_{i,j,k}^b - \sum_m\sum_b dy_{i,k,m}^b \geq , i \neq k \\
\label{opt:11}
& \sum_j\sum_b dy_{i,j,k}^b = \sum_m \sum_b dy_{j,m,k}^b,\; When \; w_k=0,  i \neq k \\
\label{opt:12}
& dy_{i,i,j}^b=0
\end{align}

In the ILP, (\ref{opt:1}) represents the summation of the incoming and outgoing 
time share and the interfering links' time share, which should all be less than 1.
Constraint (\ref{opt:2}) represents the incoming and outgoing traffic flow, which 
should be less than the link capacity for link $i,j$. Uplink constraints (\ref{opt:3})
and (\ref{opt:4}) represent that the summation of any flow $i,j$ should be less than
the demand of node $k$.  Contraints (\ref{opt:6}) and (\ref{opt:7}) are used to restrict
the sum of all incoming data flows for a given mesh node $k$ to be equal to the 
sum of all outgoing flows. Downlink constraints (\ref{opt:8}) and (\ref{opt:9}) are
similar to (\ref{opt:3}) and (\ref{opt:4}) but in the downlink direction.  Similarly,
constraits (\ref{opt:11}) and (\ref{opt:12}) are downlink versions of (\ref{opt:6}) and (\ref{opt:7}).

%link capacity distributed by its time share $\alpha$; 
%% Uplink GA/MN constraints
%Constraints (\ref{opt:6}) are to describe relay behavior of the nodes in network. If node $i$ is a mesh, then $Gateway_i=0$, the total in-coming traffic should equal to the total out-coming traffic; 
%otherwise node $i$ is a gateway, when $Gateway_i=1$, traffic get into gateway node, in-coming traffic should be greater than out-coming traffic;
%(~\ref{opt:7}) make sure no loop in the assignment, there is no traffic generated by node $i$ will go back to node $i$;
%(~\ref{opt:10}),(~\ref{opt:11}),(~\ref{opt:12})
% make gateway node provide all the down-link traffic from itself. The in-coming traffic equals to the out-coming traffic for relay traffic on mesh nodes.

%%%%Other constraints could be modified according to different objectives. 
% In gateway placement
%%%%For example, if an objective were to minimize a gateway deployment with a QoS 
%%%%constraint, the constraints that work for this objective would be $Min \sum{W_i}$ by
%%%%moving the gateway from parameter list to the variable list and modifying relative uplink and downlink constraints.  % In network traffic upbound approaching
%\noindent
%%{\bf  Constraints:} 
%\begin{align}
%& \sum_k \sum_b uy_{i,i,k}^b \geq D_{ui} - J\cdot w_k , i \neq k \\
%& \sum_i\sum_b uy_{i,j,k}^b - \sum_m\sum_b uy_{j,m,k}^b \geq 0 , i\neq k\\
%& \sum_i\sum_b uy_{i,j,k}^b - \sum_m \sum_b uy_{j,m,k}^b\leqW_k\cdot w_k , i \neq k\\
%\end{align}
%%%%Alternatively, if the objective were to maximize throughput with fairness, 
%%%%all mesh nodes would have the same demand, $Max\sum((uy_i+dy_i),i \in W)$ 
%%%%with variable $D_{ui}=a,D_{di}=b$, $a,b$ are constant number.
% and adding relative constraints.

Linear programs which attempt to solve channel assignment and routing in multihop
wireless networks have been proved to be NP hard~\cite{tang2005interference,yuan2006cross}. 
The model jointly considers factors to be considered in channel 
assignment and provides the methodology to achieve the upper bound for a 
channel assignment.  When we have a particular channel assignment $A_{i,j}^k$, we could 
modify the objective function, parameters, and constraints to find the maximum 
satisfied demand in the network.  
%More details will be discussed in section ~\ref{sec:experimentdesign} 


% How to use the model talked in experiment design
% FIXME

%Previous work has shown even in a simplified ~\emph{MultiChannel Model} a mixed integer linear program is NP-hard ~\cite{marina2010topology}. In this subsection we would like to formulate our channel assignment problem as an integer linear program and derive a upbound via its relaxation 
%in running time, iteration improvement, or even omit the integrality requirement.


