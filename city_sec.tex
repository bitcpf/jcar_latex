\section{Whte Mesh of Typical Cities}
\label{sec:cities}
% Need to check the WiEye database to see if the propagation character is correct

% Intro different scenarios, should we keep it or merge it to the previous SEC...
In this section, we introduce the multiband mesh network placement problem from policy regulation and our in-field measurement. 

FCC has assigned white space with different bandwidth according to the population density and geographical condition across US.
There are cities which have strict rule for white space usability, such as in New York, there is no white space band available ~\cite{googlespectrumdatabase}. 
In Dallas, the available spectrum is 2 channels, 12MHz which is much smaller than Tuscon, AZ as 16 channels, 96MHz ~\cite{googlespectrumdatabase}. 

The gain of a mesh network deployment varies of these cities due to the FCC ruled bandwidth.
Also, these cities have different environment, for instance, Arizona cities may have less influence from the trees, but in Dallas area, there are tons of plant will change the large scale propagation characteristics.
Propagation is another factor for deploying mesh network in these cities as mentioned in ~\ref{sec:propagation}.
An intuition is that with small propagation, the hop count for a mesh network would increase make the capacity decrease.
With the data collection in ~\ref{sec:experiment_design}, we leverge the propagation characteristics in different environments: Downtown, neighborhoods, urban area.
The propagation difference exist among these areas, this difference change the topology of the network directly with different communication and contention click in the same frequency as introduced in ~\ref{sec:propagation}.

Combined with the propagation characteristics of different environment in these cities, generally there are four scenarios in large scale we are trying to leverage for multiband mesh network placement fit for FCC regulation.


\subsection{More bandwidth Low propagation}
The first kind of cities are with lower path loss exponent and wide FCC approved white space band, such as Tuscon, AZ. We measured the propogation as FIXME through WiEye will discuss in ~\ref{sec:experiment_design}.
In such cities, the advantage white space band has larger coverage and less interference bringing substantial gain of white space.


\subsection{More bandwidth High propagation}
The second kind of cities we are considering are high path loss exponent due to geographical conditions or buildings, but have wide FCC approved white space band, such as Austin with 15 channels, 90MHz white bandwidth. FIXME{Check the WiEye database}

More bandwidth inject to a networks would bring gains for capacity, but the reduction of the cost for deployment does not benifet a lot from the white band.

\subsection{Less bandwidth Low propagation}
A third kind of cities which may benefit for the larger coverage of white space mesh nodes but the capacity is sitll limited by the bandwidth such as Houston FIXME



\subsection{Less bandwidth High propagation}
Cities like Dallas who has few available channels, 2 channels, 12MHz, but high propagation in its downtown area are the fourth scenario we are trying to deal with.

In such cities, the advantages of white space band is the least across all these cities, but we still could see some gain from white mesh node deployment.




