\subsection{Mixed Integer Linear Formulation}
\label{subsec:linearopt}

We now present a ~\emph{Mixed Integer Linear Program} formulation for the ~\emph{Multiband Multi-Radio} fixed wireless mesh network described in ~\ref{sec:problemformulation}.
% Fixme if with more input it is still NP-hard



Assume that we are given the nodes and available bands as the variable set. The communication links and conflict graph are given as parameters.
\\
\begin{tabular}{ll}
$N$ & set of nodes  \\
$B$ & set of bands  \\
\end{tabular}\\
\vspace{2pt}
Parameters:
\vspace{1pt}

%\begin{tabular}{lll}
\begin{tabular}{llp{3cm}}
$r_{i,j}^k$ & $(i,j)\in N, k \in B$ & capacity of link $i,j$ on band $k$\\
$I_{ij,lm}^k$ & $(i,j,l,m) \in N, k\in B $ & Interference of link $(i,j)$ on band $k$\\
$g_i$ & $i \in N\ binary$ &
Gateway placement\\
$uy_{i,j,k}^l$ & $(i,j,k) \in N, l \in B$ & 
 UL of node $i$ on $(j,k)$ at band $l$ \\ 
$dy_{i,j,k}^l$ & $(i,j,k) \in N, l \in B$ & 
DL of node $i$ on $(j,k)$ at band $l$ \\ 
\end{tabular}\\

% We could vary the objective
% treat each mesh node with the same demand even generally the demand of the mesh node is random. So the goodput of a integer linear program is the summation of all the demand served by the gateway nodes. We assign a uplink demand variable $\lambda u$ and downlink demand $\lambda d$ to each node. The goodput of the network could be represented as $\sum_{n \in V}(\lambda u_n+ \lambda d_n)$, the linear program is givin to $Maximize\ Goodput$.

In order to the link constraints, we define a ~\emph{Time Share} variable to represent the time division of a single link as ~\emph{$\alpha_{i,j}^k$} which is the time share for link $i->j$ in band $k$. 
Two flow variables are defined as up-link and down-link flow on a link $i->j$ for node $k$ in band $l$, $uy_{i,j,k}^l,dy_{i,j,k}^l$.

\vspace{2pt}
Variables:
\vspace{1pt}
\begin{tabular}{llp{3cm}}
$\alpha_{ij}^k$ & $k\in B, (i,j) \in N$ & 
Time share of $(i,j)$ on band $k$\\ 
$\lambda_{i}$ & $i \in N$ & 
Satisfied demand of node i\\ 
\end{tabular}

\vspace{3pt}
The constraints is given as:

%\setcounter{equation}{0}\\
\vspace{1pt}

%Objective:
%\begin{align}
%\max \quad
%& \sum_{i \in N}(\lambda u_i+ \lambda d_i)
%\end{align}\\

Constraints:
\begin{align}
& \text{Variable-Type Constraints:}\notag \\
& \alpha _{i,j}^k \leq 1 \\
& uy_{i,j,k}^l \geq 0 \\
& dy_{i,j,k}^l \geq 0 \\
& \text{Connectivity Constraints:} \notag \\
\label{opt:1}
& \sum_i \alpha_{i,j}^k + \sum_i \alpha_{j,i}^k + \sum_l\sum_m(\alpha_{l,m}^k \cdot I_{ij,lm}^k) \leq 1, i\neq j \\
\label{opt:2}
& \sum_i uy_{i,j,k}^l + \sum_i dy{i,j,k}^l \leq r_{j,k}^l \cdot \alpha_{j,k}^l \\
& \text{Uplink Constraints:}\notag \\
\label{opt:3}
& \sum_k \sum_l uy_{i,i,k}^l \geq \lambda u_i - J\cdot g_i \\
\label{opt:4}
& uy_{i,j,k}^l \leq J(1-g_i) \\
\label{opt:5}
& \sum_j\sum_l uy_{i,j,k}^l - \sum_m\sum_l uy_{i,k,m}^l \leq g_i \cdot J , i \neq k\\
\label{opt:6}
& \sum_j\sum_l uy_{i,j,k}^l - \sum_m \sum_l uy_{i,k,m}^l \geq 0, i \neq k\\
\label{opt:7}
& uy_{i,j,i}^l=0\\
& \text{Downlink Constraints:}\notag \\
\label{opt:8}
& \sum_j \sum_l dy_{i,j,i}^l \geq \lambda d_i - J\cdot g_i \\
\label{opt:9}
& dy_{i,j,k}^l \leq J(1-g_k) \\
\label{opt:10}
& \sum_j\sum_l dy_{i,j,k}^l - \sum_m\sum_l dy_{i,k,m}^l \geq -g_i \cdot J, i \neq k\\
\label{opt:11}
& \sum_j\sum_l dy_{i,j,k}^l - \sum_m \sum_l dy_{i,k,m}^l \leq 0, i \neq k\\
\label{opt:12}
& dy_{i,i,j}^l=0
\end{align}

In these constraints, $J$ is a large value to represent different behavior of mesh node and gateway node in linear. In implementation we will use the total link capacity of ~\emph{Gateway} as $J$ to reduce the computation complexity.
In the ILP, \ref{opt:1} is to restrict the link conflict constraint; \ref{opt:2} represent the link capacity distributed by time share $\alpha$; 
% Uplink GA/MN constraints
~\ref{opt:5}~\ref{opt:6} is to describe relay behavior of the network. If node $i$ is a mesh, then $g_i=0$, the total in-coming traffic should equal to the total out-coming traffic; otherwise node $i$ is a gateway, when $g_i=1$, traffic get into gateway node, in-coming traffic should be greater than out-coming traffic;
~\ref{opt:7} make sure no loop in the assignment, there is no traffic generated by node $i$ will go back to node $i$;
~\ref{opt:10},~\ref{opt:11}, ~\ref{opt:12}
 make gateway node provide all the down-link traffic from itself.The in-coming traffic equals to the out-coming traffic for relay traffic on mesh nodes.
The constraints above are need to be satisfied, otherwise the channel assignment could not be scheduled.

Other constraints could be modified according to different objectives. When the objective is to find the maximum throughput with fairness, $Max\sum(\lambda u_i+\lambda d_i), \lambda u_i=\lambda u_j, \lambda d_i=\lambda d_j, i\neq j$, constraints ~\ref{opt:3},\ref{opt:4} could be represented as above. 
If the node is a mesh, the summation out-coming flows should be greater or equal to the demand of node $i$, otherwise as a gateway, it transfers data to wired Internet directly without out-going traffic flow for up-link traffic;
Constraints ~\ref{opt:8},~\ref{opt:9}, 
restrict the down-link behavior of the network nodes similar to the up-link constraints. For mesh nodes, the in-coming flows should be greater or equal than the demand of node $i$, a gateway node has no out-coming traffic for itself.


Similar linear program has been proved as NP-hard problem ~\cite{tang2005interference,yuan2006cross}. 
The model itself provide a way to understand the factors need to be considered in channel assignment and provide methodology to achieve the throughput upper bound of a channel assignment.
When we have the channel assignment $A_{i,j}^k$, we could modify the objective function, the parameters and the constraints to find the maximum satisfied demand in the network. More details will be discussed in ~\ref{sec:experimentdesign} 




% How to use the model talked in experiment design
% FIXME

%Previous work has shown even in a simplified ~\emph{MultiChannel Model} a mixed integer linear program is NP-hard ~\cite{marina2010topology}. In this subsection we would like to formulate our channel assignment problem as an integer linear program and derive a upbound via its relaxation 
%in running time, iteration improvement, or even omit the integrality requirement.

