\subsection{Mixed Integer Linear Programming Formulation}
\label{subsec:linearopt}

We now present a mixed integer, linear programming formulation for 
finding the upper bound on gateway goodput when selecting channels
for WhiteMesh topologies across diverse bands. We assume that the
set of available mesh nodes $V$ , gateways $W$ and set of available bands $B$ are given. 
The communication links and conflict graph are given as parameters.
%the ~\emph{Multiband Multi-Radio} wireless mesh network described in section ~\ref{sec:problemformulation} to model the problem and provide a way to approach the upbound a network throughput achieve gateways.
% Fixme if with more input it is still NP-hard
%Assume that we are given the nodes and available bands as the variable set. The communication links and conflict graph are given as parameters.

\noindent
{\bf Sets:}
\begin{tabular}{ll}
$V$ & set of nodes \\
$B$ & set of bands \\
\end{tabular}

\noindent
{\bf Parameters:}\\
\\
%\vspace{2pt}
%\begin{tabular}{lll}
\begin{tabular}{llp{2.8cm}}
%\hline
$\gamma_{i,j}^b$ & $(i,j)\in V, b \in B$ & capacity of link $i,j$ on band $b$\\
%\hline
$I_{ij,lm}^b$ & $(i,j,l,m) \in V, b\in B $ & Interference of link $(i,j)$ on band $b$\\
%\hline
$W_i$ & $i \in V\ binary$ & Gateways in network\\
%\hline
$D_{di}$ & $i \in V\ $ & Downlink demand of node i\\
%\hline
$D_{ui}$ & $i \in V\ $ & Uplink demand of node i\\
%\hline
\end{tabular}\\

% We could vary the objective
% treat each mesh node with the same demand even generally the demand of the mesh node is random. So the goodput of a integer linear program is the summation of all the demand served by the gateway nodes. We assign a uplink demand variable $\lambda u$ and downlink demand $\lambda d$ to each node. The goodput of the network could be represented as $\sum_{n \in V}(\lambda u_n+ \lambda d_n)$, the linear program is givin to $Maximize\ Goodput$.

We define a variable time share to represent the percentage of time for a single link as~\emph{$\alpha_{i,j}^b$},
which is the time share for link $i->j$ in band $b$.Two flow variables are defined as all uplink and downlink flows 
on a link $i->j$ for node $k$ in band $b$, $uy_{i,j,k}^b,dy_{i,j,k}^b$.
\\
%\vspace{2pt}
{\bf Variables:}\\
\\
%\vspace{1pt}
\begin{tabular}{llp{3cm}}
$0\le \alpha_{ij}^b \le 1$  & $b\in B, (i,j) \in N$ & 
Time share of $(i,j)$ on band $b$\\ 
$0\le uy_{i,j,k}^b$ & $(i,j,k) \in V, b \in B$ & 
Up link flow of node $i$ on $(j,k)$ at band $b$ \\ 
$0\le dy_{i,j,k}^b$ & $(i,j,k) \in N, b \in B$ & 
Down link flow of node $i$ on $(j,k)$ at band $b$ \\ 
\end{tabular}

%\vspace{3pt}
To maximum the Goodput the objective in the linear program, the objectivecould be represented as:

\noindent
{\bf Objective:}\\
\\
\begin{align}
& Max \sum_i\sum_j\sum_k\sum_b(uy_{i,j,k}^b+dy_{j,i,k}^b) \; When \; w_j=1
\end{align}

If we put the linear program with QoS constraint, all the demands of mesh nodes shoule be satisfied, then the constraints are given as:
%\setcounter{equation}{0}\\
%\vspace{1pt}
%Objective:
%\begin{align}
%\max \quad
%& \sum_{i \in N}(\lambda u_i+ \lambda d_i)
%\end{align}\\

\noindent
%{\bf Constraints:}
{\bf Connectivity Constraints:}
\begin{align}
\label{opt:1}
& \alpha_{i,j}^b + \alpha_{j,i}^b + \sum_l\sum_m(\alpha_{l,m}^b \cdot I_{ij,lm}^b) \leq 1, i\neq j \\
\label{opt:2}
& \sum_i uy_{i,j,k}^b + \sum_i dy{i,j,k}^b \leq r_{j,k}^b \cdot \alpha_{j,k}^b \\
\end{align}

\noindent
{\bf Uplink Constraints:} 
\begin{align}
\label{opt:3}
& \sum_k \sum_b uy_{i,i,k}^b \leq D_{ui} \; When \; w_k=0, i \neq k \\
\label{opt:4}
& uy_{i,j,k}^b = 0 w_k=1 \\
%\label{opt:5}
%& \sum_i\sum_b uy_{i,j,k}^b - \sum_m\sum_b uy_{j,m,k}^b = 0 \; When \; w_k=0, i\neq k\\
\label{opt:6}
& \sum_i\sum_b uy_{i,j,k}^b = \sum_m \sum_b uy_{j,m,k}^b \; When \; w_k=0, i \neq k\\
\label{opt:7}
& uy_{i,j,i}^b=0 \\
\end{align}

\noindent
{\bf Downlink Constraints:} 
\begin{align}
{\bf}
\label{opt:8}
& \sum_j \sum_b dy_{i,j,i}^b \leq D_{di} \; When \; w_i=0 \\
\label{opt:9}
& dy_{i,j,k}^l =0 \; When \; w_k=1 \\
%\label{opt:10}
%& \sum_j\sum_b dy_{i,j,k}^b - \sum_m\sum_b dy_{i,k,m}^b \geq , i \neq k \\
\label{opt:11}
& \sum_j\sum_b dy_{i,j,k}^b = \sum_m \sum_b dy_{j,m,k}^b,\;When\; w_k=0,  i \neq k \\
\label{opt:12}
& dy_{i,i,j}^b=0
\end{align}

In the ILP, (\ref{opt:1}) is to restrict the link conflict constraint; (\ref{opt:2}) represent the link capacity distributed by time share $\alpha$; 
% Uplink GA/MN constraints
Constraints (~\ref{opt:5})(~\ref{opt:6}) are to describe relay behavior of the nodes in network. If node $i$ is a mesh, then $Gateway_i=0$, the total in-coming traffic should equal to the total out-coming traffic; 
otherwise node $i$ is a gateway, when $Gateway_i=1$, traffic get into gateway node, in-coming traffic should be greater than out-coming traffic;
(~\ref{opt:7}) make sure no loop in the assignment, there is no traffic generated by node $i$ will go back to node $i$;
(~\ref{opt:10}),(~\ref{opt:11}),(~\ref{opt:12})
 make gateway node provide all the down-link traffic from itself. The in-coming traffic equals to the out-coming traffic for relay traffic on mesh nodes.



Other constraints could be modified according to different objectives. 
% In gateway placement
When the objective is to minimum a gateway deployment with QoS constraint, the constraints work for this objective $Min \sum{Gateways_i}$ with moving the $Gateway$ from parameter to an variable.
% In network traffic upbound approaching
If the objective is to maximum throughput with fairness, all mesh nodes has the same demand, $Max\sum((uy_i+dy_i),i \in Gateway)$ 
with parameter $D_{ui}=a,D_{di}=b$, $a,b$ are constant number.

Linear program of network channel assignment and routing has been proved a NP-hard problem ~\cite{tang2005interference,yuan2006cross}. 
The model itself provide a way to understand the factors need to be considered in channel assignment and provide methodology to achieve the throughput upper bound of a channel assignment.
When we have the channel assignment $A_{i,j}^k$, we could modify the objective function, the parameters and the constraints to find the maximum satisfied demand in the network. 

%More details will be discussed in section ~\ref{sec:experimentdesign} 


% How to use the model talked in experiment design
% FIXME

%Previous work has shown even in a simplified ~\emph{MultiChannel Model} a mixed integer linear program is NP-hard ~\cite{marina2010topology}. In this subsection we would like to formulate our channel assignment problem as an integer linear program and derive a upbound via its relaxation 
%in running time, iteration improvement, or even omit the integrality requirement.

