\subsection{Mixed Integer Linear Formulation}
\label{subsec:linearopt}

We now present a ~\emph{Mixed Integer Linear Program} formulation for the ~\emph{Multiband Multi-Radio} wireless mesh network described in section ~\ref{sec:problemformulation} to model the problem and provide a way to approach the upbound a network throughput achieve gateways.
% Fixme if with more input it is still NP-hard


Assume that we are given the nodes and available bands as the variable set. The communication links and conflict graph are given as parameters.

Set:\\
\\
\begin{tabular}{ll}
$V$ & set of nodes  \\
$B$ & set of bands  \\
\end{tabular}\\
%\vspace{2pt}

Parameters:
%\vspace{2pt}
\\
%\begin{tabular}{lll}
\begin{tabular}{|l|l|p{2.8cm}|}
\hline
$LC_{i,j}^k$ & $(i,j)\in V, k \in B$ & capacity of link $i,j$ on band $k$\\
\hline
$I_{ij,lm}^k$ & $(i,j,l,m) \in V, k\in B $ & Interference of link $(i,j)$ on band $k$\\
\hline
$Gateway_i$ & $i \in V\ binary$ & Gateways in network\\
\hline
$D_{di}$ & $i \in V\ $ & Down link demand of node i\\
\hline
$D_{ui}$ & $i \in V\ $ & Up link demand of node i\\
\hline
\end{tabular}\\

% We could vary the objective
% treat each mesh node with the same demand even generally the demand of the mesh node is random. So the goodput of a integer linear program is the summation of all the demand served by the gateway nodes. We assign a uplink demand variable $\lambda u$ and downlink demand $\lambda d$ to each node. The goodput of the network could be represented as $\sum_{n \in V}(\lambda u_n+ \lambda d_n)$, the linear program is givin to $Maximize\ Goodput$.

We define ~\emph{Time Share} variable to represent the time division of a single link as ~\emph{$\alpha_{i,j}^k$} which is the time share for link $i->j$ in band $k$. 
Two flow variables could be defined as up-link and down-link flow on a link $i->j$ for node $k$ in band $l$, $uy_{i,j,k}^l,dy_{i,j,k}^l$.
\\
%\vspace{2pt}
Variables:\\
\\
%\vspace{1pt}
\begin{tabular}{llp{3cm}}
$0\le \alpha_{ij}^k \le 1$  & $k\in B, (i,j) \in N$ & 
Time share of $(i,j)$ on band $k$\\ 
$0\le uy_{i,j,k}^l$ & $(i,j,k) \in V, l \in B$ & 
Up link flow of node $i$ on $(j,k)$ at band $l$ \\ 
$0\le dy_{i,j,k}^l$ & $(i,j,k) \in N, l \in B$ & 
Down link flow of node $i$ on $(j,k)$ at band $l$ \\ 
\end{tabular}

%\vspace{3pt}
If we put the linear program with QoS constraint, all the demands of mesh nodes shoule be satisfied, then the constraints are given as:

%\setcounter{equation}{0}\\
\vspace{1pt}

%Objective:
%\begin{align}
%\max \quad
%& \sum_{i \in N}(\lambda u_i+ \lambda d_i)
%\end{align}\\

Constraints:
\begin{align}
& \text{Connectivity Constraints:} \notag \\
\label{opt:1}
& \sum_i \alpha_{i,j}^k + \sum_i \alpha_{j,i}^k + \sum_l\sum_m(\alpha_{l,m}^k \cdot I_{ij,lm}^k) \leq 1, i\neq j \\
\label{opt:2}
& \sum_i uy_{i,j,k}^l + \sum_i dy{i,j,k}^l \leq r_{j,k}^l \cdot \alpha_{j,k}^l \\
& \text{Uplink Constraints:}\notag \\
\label{opt:3}
& \sum_k \sum_l uy_{i,i,k}^l \geq D_{ui} - J\cdot Gateway_i \\
\label{opt:4}
& uy_{i,j,k}^l \leq J(1-Gateway_i) \\
\label{opt:5}
& \sum_j\sum_l uy_{i,j,k}^l - \sum_m\sum_l uy_{i,k,m}^l \leq Gateway_i \cdot J , i \neq k\\
\label{opt:6}
& \sum_j\sum_l uy_{i,j,k}^l - \sum_m \sum_l uy_{i,k,m}^l \geq 0, i \neq k\\
\label{opt:7}
& uy_{i,j,i}^l=0\\
& \text{Downlink Constraints:}\notag \\
\label{opt:8}
& \sum_j \sum_l dy_{i,j,i}^l \geq D_{di} - J\cdot Gateway_i \\
\label{opt:9}
& dy_{i,j,k}^l \leq J(1-g_k) \\
\label{opt:10}
& \sum_j\sum_l dy_{i,j,k}^l - \sum_m\sum_l dy_{i,k,m}^l \geq -Gateway_i \cdot J, i \neq k\\
\label{opt:11}
& \sum_j\sum_l dy_{i,j,k}^l - \sum_m \sum_l dy_{i,k,m}^l \leq 0, i \neq k\\
\label{opt:12}
& dy_{i,i,j}^l=0
\end{align}

In these constraints, $J$ is a large value to represent the different behavior of gateway nodes and mesh nodes in linear. In implementation we will use the sum of link capacity of ~\emph{Gateway} as $J$..
In the ILP, (\ref{opt:1}) is to restrict the link conflict constraint; (\ref{opt:2}) represent the link capacity distributed by time share $\alpha$; 
% Uplink GA/MN constraints
Constraints (~\ref{opt:5})(~\ref{opt:6}) are to describe relay behavior of the nodes in network. If node $i$ is a mesh, then $Gateway_i=0$, the total in-coming traffic should equal to the total out-coming traffic; 
otherwise node $i$ is a gateway, when $Gateway_i=1$, traffic get into gateway node, in-coming traffic should be greater than out-coming traffic;
(~\ref{opt:7}) make sure no loop in the assignment, there is no traffic generated by node $i$ will go back to node $i$;
(~\ref{opt:10}),(~\ref{opt:11}),(~\ref{opt:12})
 make gateway node provide all the down-link traffic from itself. The in-coming traffic equals to the out-coming traffic for relay traffic on mesh nodes.



Other constraints could be modified according to different objectives. 
% In gateway placement
When the objective is to minimum a gateway deployment with QoS constraint, the constraints work for this objective $Min \sum{Gateways_i}$ with moving the $Gateway$ from parameter to an variable.
% In network traffic upbound approaching
If the objective is to maximum throughput with fairness, all mesh nodes has the same demand, $Max\sum((uy_i+dy_i),i \in Gateway)$ 
with parameter $D_{ui}=a,D_{di}=b$, $a,b$ are constant number.

Linear program of network channel assignment and routing has been proved a NP-hard problem ~\cite{tang2005interference,yuan2006cross}. 
The model itself provide a way to understand the factors need to be considered in channel assignment and provide methodology to achieve the throughput upper bound of a channel assignment.
When we have the channel assignment $A_{i,j}^k$, we could modify the objective function, the parameters and the constraints to find the maximum satisfied demand in the network. More details will be discussed in section ~\ref{sec:experimentdesign} 


% How to use the model talked in experiment design
% FIXME

%Previous work has shown even in a simplified ~\emph{MultiChannel Model} a mixed integer linear program is NP-hard ~\cite{marina2010topology}. In this subsection we would like to formulate our channel assignment problem as an integer linear program and derive a upbound via its relaxation 
%in running time, iteration improvement, or even omit the integrality requirement.

