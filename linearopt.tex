\subsection{Mixed Integer Linear Programming Formulation}
\label{subsec:linearopt}

To clarify the problem and approach the optimization solution of the 
problem, we present a mixed integer, linear programming 
formulation for optimizing channel assignment in multiband
scenario. We assume that the set of available mesh nodes 
($V$), gateways ($W$), and available bands ($B$) are pre-known. 
The communication links and conflict graph are given as parameters.
The capacity $\delta_b$ is given as the available channel capacity
according to activity level measurement noted in~\ref{eq:intercap}. 

\noindent
{\bf Sets:}
\begin{tabular}{ll}
$V$ & set of nodes \\
$B$ & set of bands \\
\end{tabular}

\noindent
{\bf Parameters:}\\
\\
%\vspace{0.1in}
%\begin{tabular}{lll}
\begin{tabular}{llp{3.4cm}}
%\hline
$\delta^b$ & $b \in B$ & Available capacity of band $b$ in target area\\
%\begin{tabular}{llp{2.8cm}}
$I_{ij,lm}^b$ & $(i,j,l,m) \in V, b\in B $ & Protocol Interference of link $(i,j)$ on band $b$ brought by link $(l,m)$\\
%\hline
%\end{tabular}\\
%\begin{tabular}{llp{2.8cm}}
$W_i$ & $i \in V\ binary$ & Gateway maker in mesh nodes\\
%\hline
%\end{tabular}\\
%\begin{tabular}{llp{2.8cm}}
$D_{d}$ & $i \in V\ $ & Downlink demand of node i\\
%\hline
%\end{tabular}\\
%\begin{tabular}{llp{2.8cm}}
$D_{u}$ & $i \in V\ $ & Uplink demand of node i\\
%\hline
\end{tabular}

In the variable set, we define a time share represents 
the percentage of time a single link transmits according 
to~$\alpha_{i,j}^b$ for link $i,j$ between node $i$ and 
node $j$ in band $b$. There are two terms $uy_{i,j,k}^{b}$
and $dy_{i,j,k}^{b}$ defined as uplink and downlink flows:

\noindent
%\vspace{2pt}
{\bf Variables:}\\
\\
%\vspace{1pt}
\begin{tabular}{llp{3cm}}
$0\le \alpha_{ij}^b \le 1$  & $b\in B, (i,j) \in N$ & 
Time share of link $(i,j)$ on band $b$\\ 
$0\le uy_{i,j,k}^b$ & $(i,j,k) \in V, b \in B$ & 
Uplink flow of node $k$ on link $(i,j)$ at band $b$ \\ 
$0\le dy_{i,j,k}^b$ & $(i,j,k) \in N, b \in B$ & 
Downlink flow of node $k$ on link $(i,j)$ at band $b$ \\ 
\end{tabular}
\vspace{1pt}

% FIXME talk about NT calculation
Our objective is represented for  
maximizing the traffic arrived at gateway ($X$)
described in Eq.~\ref{eq:goodput}.

\noindent
{\bf Objective:}
\begin{align}
& Max \sum_i\sum_j\sum_k\sum_b(uy_{i,j,k}^b+dy_{j,i,k}^b) \;\ When \; w_j=1 
%& Max \sum_i\sum_j\sum_b(\frac{1}{\sum_{l,m}\alpha_{i,j}^b\times I_{ij,lm}})
\end{align}

In the ILP, the connectivity, uplink, and downlink constraints are represented
as:  

\noindent
%{\bf Constraints:}
{\bf Connectivity Constraints:}
\begin{align}
\label{opt:1}
& \alpha_{i,j}^b + \alpha_{j,i}^b + \sum_l\sum_m(\alpha_{l,m}^b \cdot I_{ij,lm}^b) \leq \delta^b, i\neq j \\
\label{opt:2}
& \sum_i uy_{i,j,k}^b + \sum_i dy_{i,j,k}^b \leq \delta^b \cdot \alpha_{j,k}^b 
\end{align}
\noindent
{\bf Uplink Constraints:} 
\begin{align}
\label{opt:3}
& \sum_k \sum_b uy_{i,k,i}^b \leq D_{ui}  when \; w_k=0, i \neq k \\
\label{opt:4}
& uy_{i,j,k}^b \cdot w_k = 0 \\
%\label{opt:5}
%& \sum_i\sum_b uy_{i,j,k}^b - \sum_m\sum_b uy_{j,m,k}^b = 0 \; when \; w_k=0, i\neq k\\
\label{opt:6}
& \sum_{i\leq j}\sum_b uy_{i,j,k}^b = \sum_{m\leq j} \sum_b uy_{j,m,k}^b \; w_j = 0\\
\label{opt:7}
& uy_{i,j,i}^b=0 
\end{align}
\noindent
{\bf Downlink Constraints:} 
\begin{align}
{\bf}
\label{opt:8}
& \sum_k \sum_b dy_{i,k,i}^b \leq D_{di} \; when \; w_k=0 \\
\label{opt:9}
& dy_{i,j,k}^b =0 \; when \; w_k=1 \\
%\label{opt:10}
%& \sum_j\sum_b dy_{i,j,k}^b - \sum_m\sum_b dy_{i,k,m}^b \geq , i \neq k \\
\label{opt:11}
& \sum_{i\neq j} \sum_b dy_{i,j,k}^b = \sum_{m \neq j} \sum_b dy_{j,m,k}^b  \; when \; w_j=0, i \neq k 
%%\label{opt:12}
%%& dy_{i,i,j}^b=0
\end{align}

In the constraints, (\ref{opt:1}) represents the summation of the incoming and outgoing 
wireless time share and the interfering links' wireless time share, which should all be less than 1.
Constraint (\ref{opt:2}) represents the incoming and outgoing wireless traffic, which 
should be less than the link capacity for link $i,j$. Uplink constraints (\ref{opt:3})
and (\ref{opt:4}) represent that the summation of any wireless flow $i,j$ should be less than
the demand of node $k$.  Constraints (\ref{opt:6}) and (\ref{opt:7}) are used to restrict
the sum of all incoming data flows for a given mesh node $k$ to be equal to the 
sum of all outgoing flows. Downlink constraints (\ref{opt:8}) and (\ref{opt:9}) are
similar to (\ref{opt:3}) and (\ref{opt:4}) but in the downlink direction.  Similarly,
constraints (\ref{opt:11}) and (\ref{opt:12}) are downlink versions of 
(\ref{opt:6}) and (\ref{opt:7}).

Similar linear programs model is to solve channel assignment 
wireless networks have been proved to be NP-hard~\cite{yuan2006cross}. 
Our model jointly considers channel assignment factors and provides the 
methodology to achieve the optimization solution. 
When the particular configuration is given, we further 
choose the objectives, parameters, and relax constraints to 
find the approaching solution for the network.  
 
