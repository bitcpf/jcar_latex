\subsection{Mixed Integer Linear Programming Formulation}
\label{subsec:linearopt}

We now present a mixed integer, linear programming formulation for 
optimizing gateway goodput when selecting channels
for WhiteMesh topologies across diverse bands. We assume that the
set of available mesh nodes $V$ , gateways $W$ and available 
bands $B$ are given.  The communication links and conflict graph 
are given as parameters.
%the ~\emph{Multiband Multi-Radio} wireless mesh network described in section ~\ref{sec:problemformulation} to model the problem and provide a way to approach the upbound a network throughput achieve gateways.
% Fixme if with more input it is still NP-hard
%Assume that we are given the nodes and available bands as the variable set. The communication links and conflict graph are given as parameters.

\noindent
{\bf Sets:}
\begin{tabular}{ll}
$V$ & set of nodes \\
$B$ & set of bands \\
\end{tabular}

\noindent
{\bf Parameters:}\\
\\
%\vspace{0.1in}
%\begin{tabular}{lll}
\begin{tabular}{llp{3.4cm}}
%\hline
$\gamma_{i,j}^b$ & $(i,j)\in V, b \in B$ & capacity of link $i,j$ on band $b$\\
%\hline
%\end{tabular}\\
%\begin{tabular}{llp{2.8cm}}
$I_{ij,lm}^b$ & $(i,j,l,m) \in V, b\in B $ & Interference of link $(i,j)$ on band $b$\\
%\hline
%\end{tabular}\\
%\begin{tabular}{llp{2.8cm}}
$W_i$ & $i \in V\ binary$ & Gateways in network\\
%\hline
%\end{tabular}\\
%\begin{tabular}{llp{2.8cm}}
$D_{di}$ & $i \in V\ $ & Downlink demand of node i\\
%\hline
%\end{tabular}\\
%\begin{tabular}{llp{2.8cm}}
$D_{ui}$ & $i \in V\ $ & Uplink demand of node i\\
%\hline
\end{tabular}

% We could vary the objective
% treat each mesh node with the same demand even generally the demand of the mesh node is random. So the goodput of a integer linear program is the summation of all the demand served by the gateway nodes. We assign a uplink demand variable $\lambda u$ and downlink demand $\lambda d$ to each node. The goodput of the network could be represented as $\sum_{n \in V}(\lambda u_n+ \lambda d_n)$, the linear program is givin to $Maximize\ Goodput$.

We define time share to represent the percentage of time a 
single link transmits according to~$\alpha_{i,j}^b$
for link $i,j$ in band $b$. Two flow 
variables are defined as uplink and downlink flows as below:
%on 
%a link $i,j$ for node $k$ in band $b$, $uy_{i,j,k}^b,dy_{i,j,k}^b$.

\noindent
%\vspace{2pt}
{\bf Variables:}\\
\\
%\vspace{1pt}
\begin{tabular}{llp{3cm}}
$0\le \alpha_{ij}^b \le 1$  & $b\in B, (i,j) \in N$ & 
Time share of link $(i,j)$ on band $b$\\ 
$0\le uy_{i,j,k}^b$ & $(i,j,k) \in V, b \in B$ & 
Uplink flow of node $k$ on link $(i,j)$ at band $b$ \\ 
$0\le dy_{i,j,k}^b$ & $(i,j,k) \in N, b \in B$ & 
Downlink flow of node $k$ on link $(i,j)$ at band $b$ \\ 
\end{tabular}

%\vspace{3pt}
Our objective is to maximize the gateway goodput.

\noindent
{\bf Objective:}
\begin{align}
& Max \sum_i\sum_j\sum_k\sum_b(uy_{i,j,k}^b+dy_{j,i,k}^b) \; When \; w_j=1
\end{align}

The constraints for the variables are represented as:  
%\setcounter{equation}{0}\\
%\vspace{1pt}
%Objective:
%\begin{align}
%\max \quad
%& \sum_{i \in N}(\lambda u_i+ \lambda d_i)
%\end{align}\\

\noindent
%{\bf Constraints:}
{\bf Connectivity Constraints:}
\begin{align}
\label{opt:1}
& \alpha_{i,j}^b + \alpha_{j,i}^b + \sum_l\sum_m(\alpha_{l,m}^b \cdot I_{ij,lm}^b) \leq 1, i\neq j \\
\label{opt:2}
& \sum_i uy_{i,j,k}^b + \sum_i dy_{i,j,k}^b \leq r_{j,k}^b \cdot \alpha_{j,k}^b 
\end{align}
\noindent
{\bf Uplink Constraints:} 
\begin{align}
\label{opt:3}
& \sum_k \sum_b uy_{i,i,k}^b \leq D_{ui} \; When \; w_k=0, i \neq k \\
\label{opt:4}
& uy_{i,j,k}^b = 0 w_k=1 \\
%\label{opt:5}
%& \sum_i\sum_b uy_{i,j,k}^b - \sum_m\sum_b uy_{j,m,k}^b = 0 \; When \; w_k=0, i\neq k\\
\label{opt:6}
& \sum_i\sum_b uy_{i,j,k}^b = \sum_m \sum_b uy_{j,m,k}^b \; When \; w_k=0, i \neq k\\
\label{opt:7}
& uy_{i,j,i}^b=0 
\end{align}
\noindent
{\bf Downlink Constraints:} 
\begin{align}
{\bf}
\label{opt:8}
& \sum_j \sum_b dy_{i,j,i}^b \leq D_{di} \; When \; w_i=0 \\
\label{opt:9}
& dy_{i,j,k}^l =0 \; When \; w_k=1 \\
%\label{opt:10}
%& \sum_j\sum_b dy_{i,j,k}^b - \sum_m\sum_b dy_{i,k,m}^b \geq , i \neq k \\
\label{opt:11}
& \sum_j\sum_b dy_{i,j,k}^b = \sum_m \sum_b dy_{j,m,k}^b,\; When \; w_k=0,  i \neq k \\
\label{opt:12}
& dy_{i,i,j}^b=0
\end{align}

In the ILP, (\ref{opt:1}) represents the summation of the incoming and outgoing 
time share and the interfering links' time share, which should all be less than 1.
Constraint (\ref{opt:2}) represents the incoming and outgoing traffic flow, which 
should be less than the link capacity for link $i,j$. Uplink constraints (\ref{opt:3})
and (\ref{opt:4}) represent that the summation of any flow $i,j$ should be less than
the demand of node $k$.  Contraints (\ref{opt:6}) and (\ref{opt:7}) are used to restrict
the sum of all incoming data flows for a given mesh node $k$ to be equal to the 
sum of all outgoing flows. Downlink constraints (\ref{opt:8}) and (\ref{opt:9}) are
similar to (\ref{opt:3}) and (\ref{opt:4}) but in the downlink direction.  Similarly,
constraits (\ref{opt:11}) and (\ref{opt:12}) are downlink versions of (\ref{opt:6}) and (\ref{opt:7}).

%link capacity distributed by its time share $\alpha$; 
%% Uplink GA/MN constraints
%Constraints (\ref{opt:6}) are to describe relay behavior of the nodes in network. If node $i$ is a mesh, then $Gateway_i=0$, the total in-coming traffic should equal to the total out-coming traffic; 
%otherwise node $i$ is a gateway, when $Gateway_i=1$, traffic get into gateway node, in-coming traffic should be greater than out-coming traffic;
%(~\ref{opt:7}) make sure no loop in the assignment, there is no traffic generated by node $i$ will go back to node $i$;
%(~\ref{opt:10}),(~\ref{opt:11}),(~\ref{opt:12})
% make gateway node provide all the down-link traffic from itself. The in-coming traffic equals to the out-coming traffic for relay traffic on mesh nodes.

%%%%Other constraints could be modified according to different objectives. 
% In gateway placement
%%%%For example, if an objective were to minimize a gateway deployment with a QoS 
%%%%constraint, the constraints that work for this objective would be $Min \sum{W_i}$ by
%%%%moving the gateway from parameter list to the variable list and modifying relative uplink and downlink constraints.  % In network traffic upbound approaching
%\noindent
%%{\bf  Constraints:} 
%\begin{align}
%& \sum_k \sum_b uy_{i,i,k}^b \geq D_{ui} - J\cdot w_k , i \neq k \\
%& \sum_i\sum_b uy_{i,j,k}^b - \sum_m\sum_b uy_{j,m,k}^b \geq 0 , i\neq k\\
%& \sum_i\sum_b uy_{i,j,k}^b - \sum_m \sum_b uy_{j,m,k}^b\leqW_k\cdot w_k , i \neq k\\
%\end{align}
%%%%Alternatively, if the objective were to maximize throughput with fairness, 
%%%%all mesh nodes would have the same demand, $Max\sum((uy_i+dy_i),i \in W)$ 
%%%%with variable $D_{ui}=a,D_{di}=b$, $a,b$ are constant number.
% and adding relative constraints.

Linear programs which attempt to solve channel assignment and routing in multihop
wireless networks have been proved to be NP hard~\cite{tang2005interference,yuan2006cross}. 
The model jointly considers factors to be considered in channel 
assignment and provides the methodology to achieve the upper bound for a 
channel assignment.  When we have a particular channel assignment $A_{i,j}^k$, we could 
modify the objective function, parameters, and constraints to find the maximum 
satisfied demand in the network.  
%More details will be discussed in section ~\ref{sec:experimentdesign} 


% How to use the model talked in experiment design
% FIXME

%Previous work has shown even in a simplified ~\emph{MultiChannel Model} a mixed integer linear program is NP-hard ~\cite{marina2010topology}. In this subsection we would like to formulate our channel assignment problem as an integer linear program and derive a upbound via its relaxation 
%in running time, iteration improvement, or even omit the integrality requirement.

