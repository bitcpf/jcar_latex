\section{Mixed integer Linear Solution}
\label{sec:linearopt}

First we present a ~\emph{Mixed integer linear Program} formulation. A linear program combine both channel assignment and routing solution together. 
% Fixme if with more input it is still NP-hard
Previous work has shown even in a simplified ~\emph{MultiChannel Model} a mixed integer linear program is NP-hard ~\cite{marina2010topology}. In this subsection we would like to formulate our channel assignment problem as an integer linear program and derive a upbound via its relaxation 
in running time, iteration improvement, or even omit the integrality requirement.

To keep the fairness constraint, we treat each mesh node with the same demand even generally the demand of the mesh node is random. So the goodput of a integer linear program is the summation of all the demand served by the gateway nodes. We assign a uplink demand variable $\lambda u$ and downlink demand $\lambda d$ to each node. The goodput of the network could be represented as $\sum_{n \in V}(\lambda u_n+ \lambda d_n)$, the linear program is givin to $Maximize\ Goodput$.

We define a ~\emph{Time Share} variable to represent the time division of a single link as ~\emph{$\alpha_{i,j}^k$} which is the time share for link $i->j$ in band $k$. 
Two flow variables are defined as uplink and downlink flow on a link $i->j$ for node $k$ in band $l$, $uy_{i,j,k}^l,dy_{i,j,k}^l$.

\vspace{3pt}
The ILP is given as:

\vspace{2pt}
 Sets:
\vspace{2pt}

\begin{tabular}{ll}
$N$ & set of nodes \\
$B$ & set of bands \\
\end{tabular} 

\vspace{2pt}
Parameters: 
\vspace{1pt}

\begin{tabular}{lll}
capacity of link $(i,j)$ on band $k$ \\
$r_{i,j}^k$ & $(i,j) \in N, k \in B$ & \\
Interference of link $(i,j)$ on band $k$\\
$I_{ij,lm}^k$ & $(i,j,l,m) \in N, k \in B $ & \\
Gateway placement\\
$g_i$ & $i \in N\ binary$ &
\end{tabular}\\

\vspace{2pt}
Variables:
\vspace{1pt}

\begin{tabular}{lll}
Time share of a link $(i,j)$ on band $k$\\ 
$\alpha_{ij}^k$ & $k\in B, (i,j) \in N$ & \\
 Uplink traffic of node $i$ on $(j,k)$ at band $l$ \\ 
$uy_{i,j,k}^l$ & $(i,j,k) \in N, l \in B$ & \\
Dwonlink traffic of node $i$ on $(j,k)$ at band $l$ \\ 
$dy_{i,j,k}^l$ & $(i,j,k) \in N, l \in B$ & 
\end{tabular}
%\setcounter{equation}{0}\\
\vspace{1pt}

Objective:
\begin{align}
\max \quad
& \sum_{i \in N}(\lambda u_i+ \lambda d_i)
\end{align}\\

Constraints:
\begin{align}
& \text{Variable-Type Constraints:}\notag \\
& \alpha _{i,j}^k \leq 1 \\
& uy_{i,j,k}^l \geq 0 \\
& dy_{i,j,k}^l \geq 0 \\
& \text{Connectivity Constraints:} \notag \\
\label{opt:1}
& \sum_i \alpha_{i,j}^k + \sum_i \alpha_{j,i}^k + \sum_l\sum_m(\alpha_{l,m}^k \cdot I_{ij,lm}^k) \leq 1, i\neq j \\
\label{opt:2}
& \sum_i uy_{i,j,k}^l + \sum_i dy{i,j,k}^l \leq r_{j,k}^l \cdot \alpha_{j,k}^l \\
& \text{Uplink Constraints:}\notag \\
\label{opt:3}
& \sum_k \sum_l uy_{i,i,k}^l \geq \lambda u_i - J\cdot g_i \\
\label{opt:4}
& uy_{i,j,k}^l \leq J(1-g_i) \\
\label{opt:5}
& \sum_j\sum_l uy_{i,j,k}^l - \sum_m\sum_l uy_{i,k,m}^l \leq g_i \cdot J , i \neq k\\
\label{opt:6}
& \sum_j\sum_l uy_{i,j,k}^l - \sum_m \sum_l uy_{i,k,m}^l \geq 0, i \neq k\\
\label{opt:7}
& uy_{i,j,i}^l=0\\
& \text{Downlink Constraints:}\notag \\
\label{opt:8}
& \sum_j \sum_l dy_{i,j,i}^l \geq \lambda d_i - J\cdot g_i \\
\label{opt:9}
& dy_{i,j,k}^l \leq J(1-g_k) \\
\label{opt:10}
& \sum_j\sum_l dy_{i,j,k}^l - \sum_m\sum_l dy_{i,k,m}^l \geq -g_i \cdot J, i \neq k\\
\label{opt:11}
& \sum_j\sum_l dy_{i,j,k}^l - \sum_m \sum_l dy_{i,k,m}^l \leq 0, i \neq k\\
\label{opt:12}
& dy_{i,i,j}^l=0
\end{align}

$J$ is a large value to distinguish different behavior of mesh node and gateway node, it could be any large value, such as $10^6$ or even more. In implementation we will use the total link capacity of ~\emph{Gateway} as $J$ to reduce the computaion complexity.$J$ is used to keep the constraints linear.
We use two variables $uy,dy$ represents uplink and downlink traffic flow in the model. 
In the ILP, \ref{opt:1} is to restrict all the links interference each other share time in a unit; \ref{opt:2} deal with the link capacity distributed by time share $\alpha$; 
% Uplink GA/MN constraints
In ~\ref{opt:3},\ref{opt:4}, $J$ is a large value helping distinguish different behavior of mesh node and gateway node, if the node is a mesh node, the sumation out-coming flows should be greater or equal the demand of node $i$, otherwise as a gateway node, it transfer data to wired Internet directly without out-going traffic flow for up-link traffic;
~\ref{opt:5}~\ref{opt:6} is to describe relay behavior of mesh nodes. If $i$ is a mesh node, $g_i=0$, the total in-coming traffic should equal to the total out-coming traffic, otherwise, when $g_i=1$, traffic get into gateway node, in-coming traffic should be greater than out-coming traffic;
~\ref{opt:7} make sure no loop in the assignment, there is no traffic generated by node $i$ will go back to node $i$;
Constraints ~\ref{opt:8},~\ref{opt:9}, 
~\ref{opt:10},~\ref{opt:11}, ~\ref{opt:12}
restrict the downlink behavior of the network nodes similar to the uplink constraints, for mesh nodes, the in-coming flows should be greated or equal the demand of node $i$, a gateway node has no out-coming traffic for itself; As relay nodes, out-coming traffic for others equals in-comming traffic from others, gateway node will provide all the downlink traffic from it self, we use the same trick $J$ to represent such constarints.

The model resolve ~\emph{Channel Assignment} and ~\emph{Routing} problem simultaneously. However the model itself is NP-hard could not get an optimization resuslt in polynomial time. But most of the solver has configuration to set the running time or iteration difference. It provides us a methodology to approach a reasonable results for channel assignment.
% How to use the model talked in experiment design
% FIXME

