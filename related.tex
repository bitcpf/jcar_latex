\section{Related Work}
\label{sec:related}
% Deployment problem 
There are significant challenges in wireless mesh network deployment,
such as user priorities, user behaviors, long term throughput estimation, 
interference and energy efficiency.~\cite{tragos2013spectrum}
Previous works have identified the impact of interference in wireless mesh network 
deployments as the key issue~\cite{tang2005interference,irwin2013resource,chieochan2013channel}.
To overcome the challenges, previous works have optimized the 
deployment in forms ofincreasing throughput, minimizing resources, reducing interference,
~\cite{irwin2013resource,subramanian2008minimum,doraghinejad2014channel}.
Many works have studied the network deployment problem in multihop wireless 
networks~\cite{jain2005impact,akyildiz2005wireless,raniwala2004centralized,tragos2013spectrum}.
Both static and dynamic network deployments have been discussed in previous works under
an 802.11 WiFi scenario~\cite{wu2006analysis,ramachandran2006interference,subramanian2008minimum}. 
However, all of the aforementioned works have not considered propagation variation of the 
diverse frequency bands among white space and WiFi bands, which we show critical in reducing the cost of
mesh networks. Frequency agility in multiband scenario brings more 
complexity of 
resolving the interference issues.

% White Space
To be used effectively, white space band must ensure that available TV bands
exist but no interference exists from microphones and other devices~\cite{bahl2009white}. 
White space bands availability has to be known in advance of network deployment.
TV channels allowed by FCC are fairly static in their channel assignment. 
Databases have been used to account for white space channel availability 
(e.g., Microsoft's White Space Database~\cite{msdatabase}).
In fact, Google has even visualized the licensed white space channels 
in US cities with an API for research and commercial use~\cite{googledatabase}.
In contrast, we study the performance of mesh networks with a varying number 
of available white space channels at varying population densities, assuming 
such white space databases and mechanisms are in place.  
Many methods have been proposed to employ these white space bands. For example,
~\cite{bahl2009white} introduced WiFi-like white space link implementation on USRP 
links. The point to point communication
in a multiband scenario was discussed in~\cite{cui2013leveraging} . In~\cite{filippini2013new}, white space bands
were applied to a cognitive radio network for reducing maintenance cost. 
In this work, the objective is maximizing the served traffic flow of clients in the WhiteMesh networks.




