\section{Related Work}
\label{sec:related}
% Deployment problem 
There are significant challenges in wireless mesh network deployment,
such as user priorities, user behaviors, long term throughput estimation, 
interference and energy efficiency, etc.~\cite{tragos2013spectrum}
Previous works have recognize the impact of interference in wireless mesh network 
deployment is the key issue~\cite{tang2005interference,irwin2013resource,chieochan2013channel}.
To overcome the challenges, previous works have been done to optimize the 
deployment in increasing throughput, minimize resource, reducing interference,
etc~\cite{irwin2013resource,subramanian2008minimum,doraghinejad2014channel}.
Many works have studied the network deployment problem in multihop wireless networks
~\cite{jain2005impact,akyildiz2005wireless,raniwala2004centralized,tragos2013spectrum}.
Both static and dynamic network deployments have been discussed in previous works under
the 802.11 WiFi scenario~\cite{wu2006analysis,ramachandran2006interference,subramanian2008minimum}. 
However, all of the aforementioned works have not considered propagation variation of the 
diverse frequency bands among white space and WiFi, which we show are critical improving 
the performance of mesh networks. Frequency agility in multiband scenario brings more 
achieved channel capacity to wireless network deployment as well as more complexity of 
resolving the interference issues.

% White Space
To be used effectively, white space bands must ensure that available TV bands
exist but no interference exists between microphones and other devices~\cite{bahl2009white}. 
White space bands availability has to be known in prior of network deployment.
TV channels freed by FCC are fairly static in their channel assignment, 
databases have been used to account for white space channel availability 
(e.g., Microsoft's White Space Database~\cite{msdatabase}).
In fact, Google has even visualized the licensed white space channels 
in US cities with an API for research and commercial use~\cite{googledatabase}.
In contrast, we study the performance of mesh networks with a varying number 
of available white space channels at varying population densities, assuming 
such white space databases and mechanisms are in place. As FCC release these 
bands for research, many methods have been proposed to employ these frequency bands.
~\cite{bahl2009white} introduce WiFi like white space link implementation on USRP and 
link protocols. ~\cite{cui2013leveraging} discuss the point to point communication
in multiband scenario. In~\cite{filippini2013new}, white space band application is 
discussed in cognitive radio network for reducing maintenance cost. 
In this work, the objective is maximizing the served traffic flow of clients in the wireless network.





