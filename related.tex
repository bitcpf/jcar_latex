\section{Related Work}
\label{sec:related}

{\bf Optimization and Game Theory.} Optimization frameworks based on linear programming
have been used to determine the channel assignment in multihop wireless networks for given 
link bandwidths and route requests~\cite{tang2005interference}.  Since these formulations 
attempt to solve NP-hard problems, LP relaxation is commonly used to output network flows 
that potentially are not feasible channel assignments~\cite{FIXME}. 
Game theory has also been used to guide distributed channel assignment and routing 
algorithms which attempt to balance the load of the network~\cite{raniwala2005architecture,
wang2010game}.  In such a framework, nodes advertise their dynamically-changing costs 
to reach their associated gateway according to the residual bandwidth. If a node learns of
a less expensive path to the current gateway or another gateway, the appropriate action is
taken to construct such a path, but hysteresis must be carefully considered to avoid route flapping.

{\bf Multi-Channel, Multi-Radio Algorithms.} Many works have studied the channel assignment problem in multihop wireless
networks~\cite{jain2005impact,akyildiz2005wireless,raniwala2004centralized}.
In addition, multiple radios have been used to improve the routing in multi-channel
scenarios~\cite{draves2004routing}. Still others have used static channel assignments
where everything is known about the network parameters~\cite{subramanian2008minimum},
which is in contrast to dynamic channel assignment where demand and interference are 
not known {\it a priori}~\cite{wu2006analysis,ramachandran2006interference}. 
%FIXME: where does this fit? Further investigations include ~\cite{marina2010topology} and i~\cite{raniwala2004centralized} show that it is possible to improve the performance of multi-channel network by applying smart channel assignment algorithms. 
We have implemented such channel assignment algorithms (CCA from~\cite{draves2004routing} and 
BFSCA from~\cite{ramachandran2006interference}) and shown significant gains.  
Most importantly, all of the aforementioned works have not considered propagation 
differences of the diverse frequency bands of white space and WiFi, which we show are 
critical improving the performance of mesh networks.

%FIXME
{\bf White Space.} We need to cite some work by Microsoft on white spaces (SIGCOMM 2009 and other
Ranveer Chandra papers).  Talk about databases for available white space channels and say that 
in this work, we study the performance of mesh networks with varying number of available white 
space channels at varying population densities (by adjusting the offered load), assuming such 
white space databases and mechanisms are in place.

%However, none of them is adaptive to the propagation difference in the wireless mesh network.
%Prior research could be classified as ~\emph{Static Channel Assignment} and ~\emph{Dynamic Channel Assignment}.
%The ~\emph{Static Channel Assignment} focus on assigning a network once with everything known of the network~\cite{subramanian2008minimum}.
%The ~\emph{Dynamic Channel Assignment} is to resolve the channel assignment according to dynamic parameters, such as demand, dynamic interference ~\cite{wu2006analysis,ramachandran2006interference}.

%A simple approach ~\emph{Common Channel Assignment} (CCA) for utilizing multi-radios network is
%presented in ~\cite{draves2004routing}, though the main purpose of the author is to have multi-raidos working for routing
%to benefit from multi-channel structure. 

%Further investigations include ~\cite{marina2010topology} and i~\cite{raniwala2004centralized} show that it is possible to improve the performance of multi-channel network by applying smart channel assignment algorithms. 


%Assuming the knowledge of the set of connection request to be routed, both an optimal algorithm based on solving a Linear Programming an a simple heuristic are proposed to route such requests given the link bandwidth availability determined by the computed channel assignment in ~\cite{tang2005interference}. 
%These formulation are NP-hard problems, the authors solve the LP relaxation of the problem. The output is a network flow along with a possibly unfeasible channel assignment. 
 

% Distributed
%Distributed channel assignment and routing algorithms are developed based on game theory ~\cite{raniwala2005architecture,wang2010game}.Nodes advertise their cost to reach the gateway they are currently associated with.
%Cost dynamically changes as it depends on residual bandwidth to achieve load balancing. If a node is notified of a less cost path towards another gateway, it starts a procedure to associate with that gateway. If the cost of a path is less than current occupacied path, the node starts evaluate wheather to associate the new path. 
%Since cost is dynamic, the strategy may leadd to route flaps and to a non convergent network behavior. Thus the equilibrium need to be carefully designed.



% Game theory? 
