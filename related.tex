\section{Related Work}
\label{sec:related}

{\bf Optimization and Game Theory.} Optimization frameworks based on linear programming
have been used to determine the channel assignment in multihop wireless networks for given 
link bandwidths and route requests~\cite{tang2005interference}.  Since these formulations 
attempt to solve NP-hard problems, LP relaxation is commonly used to output network flows 
that potentially are not feasible channel assignments~\cite{si2010overview}. 
Game theory has also been used to guide distributed channel assignment and routing 
algorithms which attempt to balance the load of the network~\cite{raniwala2005architecture,
wang2010game}.  In such a framework, nodes advertise their dynamically-changing costs 
to reach their associated gateway according to the residual bandwidth. If a node learns of
a less expensive path to the current gateway or another gateway, the appropriate action is
taken to construct such a path, but hysteresis must be carefully considered to avoid route flapping.

{\bf Multi-Channel, Multi-Radio Algorithms.} Many works have studied the channel assignment problem in multihop wireless
networks~\cite{jain2005impact,akyildiz2005wireless,raniwala2004centralized}.
In addition, multiple radios have been used to improve the routing in multi-channel
scenarios~\cite{draves2004routing}. Still others have used static channel assignments
where everything is known about the network parameters~\cite{subramanian2008minimum},
which is in contrast to dynamic channel assignment where demand and interference are 
not known {\it a priori}~\cite{wu2006analysis,ramachandran2006interference}. 
%FIXME: where does this fit? Further investigations include ~\cite{marina2010topology} and i~\cite{raniwala2004centralized} show that it is possible to improve the performance of multi-channel network by applying smart channel assignment algorithms. 
We have implemented such channel assignment algorithms (CCA from~\cite{draves2004routing} and 
BFS-CA from~\cite{ramachandran2006interference}) and shown significant gains.  
Most importantly, all of the aforementioned works have not considered propagation 
differences of the diverse frequency bands of white space and WiFi, which we show are 
critical improving the performance of mesh networks.
% 
%Coloring methodology assume all the color equal failing to process multiple topology generating by bands combination ~\cite{marina2010topology}. 
% 
%
%We need to cite some work by Microsoft on white spaces (SIGCOMM 2009 and other
%Ranveer Chandra papers).  Talk about databases for available white space channels and say that 
%in this work, we study the performance of mesh networks with varying number of available white 
%space channels at varying population densities (by adjusting the offered load), assuming such 
%white space databases and mechanisms are in place.


{\bf White Space.} To be used effectively, white space bands must ensure that available TV bands
exist but no interference exists between microphones and other devices~\cite{bahl2009white}. 
%Spatial variation gives rise to new challenges for implementing a wireless network in this band. 
%Government department FCC has licensed white space bands in US ~\cite{fccwhitespace}.
Since TV channels are fairly static in their channel assignment, databases have been used to 
account for white space channel availability (e.g., Microsoft's White Space Database~\cite{msdatabase}).
%~\emph{Microsoft Research White Space databas} for white space application reference ~\cite{msdatabase}. 
In fact, Google has even visualized the licensed white space channels in US cities with an API for 
research and commercial use~\cite{googledatabase}.  In contrast, we study the performance of mesh 
networks with a varying number of available white space channels at varying population densities, assuming such white space databases and mechanisms are in place.

%The white space potential and the open free spectrum database provide chance to improve the mesh network performance in combining
% these UHF channels and Wi-Fi channels.
% In this work, we study the performance of mesh network with varying number of available white space channels at varying population densities as offerload to offer solution for white space channel application in mesh networks. 

%However, none of them is adaptive to the propagation difference in the wireless mesh network.
%Prior research could be classified as ~\emph{Static Channel Assignment} and ~\emph{Dynamic Channel Assignment}.
%The ~\emph{Static Channel Assignment} focus on assigning a network once with everything known of the network~\cite{subramanian2008minimum}.
%The ~\emph{Dynamic Channel Assignment} is to resolve the channel assignment according to dynamic parameters, such as demand, dynamic interference ~\cite{wu2006analysis,ramachandran2006interference}.

%A simple approach ~\emph{Common Channel Assignment} (CCA) for utilizing multi-radios network is
%presented in ~\cite{draves2004routing}, though the main purpose of the author is to have multi-raidos working for routing
%to benefit from multi-channel structure. 

%Further investigations include ~\cite{marina2010topology} and i~\cite{raniwala2004centralized} show that it is possible to improve the performance of multi-channel network by applying smart channel assignment algorithms. 


%Assuming the knowledge of the set of connection request to be routed, both an optimal algorithm based on solving a Linear Programming an a simple heuristic are proposed to route such requests given the link bandwidth availability determined by the computed channel assignment in ~\cite{tang2005interference}. 
%These formulation are NP-hard problems, the authors solve the LP relaxation of the problem. The output is a network flow along with a possibly unfeasible channel assignment. 
 

\section{Related Work}
\label{sec:related}

% Max Performance 
There are significant challenges in wireless mesh network deployment,
such as user priorities, user behaviors, long term throughput estimation, selfish clients,
interference and energy efficiency, etc.~\cite{tragos2013spectrum}
These challenges are distributed under the topics of channel assignment,
cognitive radio, protocol design, etc.~\cite{tragos2013spectrum,akyildiz2006next}
Previous works have recognize the impact of interference in 
wireless mesh network deployment is the key issue~\cite{tang2005interference,irwin2013resource,chieochan2013channel}.
To overcome the challenges, a lot of works have been done to optimize the 
deployment in increasing throughput, minimize resource, reducing interference,
etc.~\cite{irwin2013resource,subramanian2008minimum,doraghinejad2014channel}
Many works have studied the network deployment problem in multihop wireless
networks~\cite{jain2005impact,akyildiz2005wireless,raniwala2004centralized,tragos2013spectrum}.
In addition, multiple radios have been used to improve the routing in multi-channel
scenarios~\cite{draves2004routing,irwin2013resource}. 
Both static and dynamic network deployments have been discussed in previous works under
the 802.11 WiFi scenario~\cite{wu2006analysis,ramachandran2006interference,subramanian2008minimum}. 
However, all of the aforementioned works have not considered propagation 
differences of the diverse frequency bands of white space and WiFi, which we show are 
critical improving the performance of mesh networks.
Frequency agility in 
multiband scenario brings more traffic capacity to wireless network deployment
as well as more complexity of resolving the interference issues.

In wireless network deployment, reduce the interference is the key issue.
Previous work~\cite{pcuiwinmee} involve the inter-network interference in
multiband scenario, but did not offer the solution of intra-network interference.
As a new designed wireless network, intra-network interference is 
more important for performance estimation. Previous work focus on
WiFi wireless networks proposed several methods to reduce the 
interference targeting on multiple metrics.
~\cite{tang2005interference,he2008optimizing,robinson2010deploying}
focus on reducing the gateway mesh nodes. ~\cite{irwin2013resource,subramanian2008minimum} try to reduce the
overall interference in the worst case of traffic independent scenario.
~\cite{chieochan2013channel,li2012channel} improve the performance
in throughput. However, these works fails to involve the traffic demands
of clients in their solutions.~\cite{robinson2010deploying,long2013fair} consider
the QoS requirements in the WiFi network design. Our work also
consider the traffic demands from the client as part of our 
network design to satisfy both customers and vendors.



% Solutions
The wireless network deployment problem has been 
proved as a NP-hard problem~\cite{si2010overview}. 
Several works introduce relaxed linear program formulation 
to find the optimization of multihop wireless networks 
\cite{tang2005interference,irwin2013resource,filippini2013new}.  
Also, game theory methods is another option to solve 
the problem~\cite{raniwala2005architecture,
wang2010game}.  
Social network analysis is also popular in wireless 
network design~\cite{zhu2013survey}.
In contrast, we formulate the multiband scenario problem
as a graph model similar to~\cite{robinson2010deploying}
for approaching.

% White Space
To be used effectively, white space bands must ensure that available TV bands
exist but no interference exists between microphones and other devices~\cite{bahl2009white}. 
White space bands availability has to be known in prior of network deployment.
TV channels freed by FCC are fairly static in their channel assignment, 
databases have been used to account for white space channel availability 
(e.g., Microsoft's White Space Database~\cite{msdatabase}).
In fact, Google has even visualized the licensed white space channels 
in US cities with an API for research and commercial use~\cite{googledatabase}.
In contrast, we study the performance of mesh networks with a varying number 
of available white space channels at varying population densities, assuming 
such white space databases and mechanisms are in place. As FCC release these 
bands for research, many methods have been proposed to employ these frequency bands.
~\cite{bahl2009white} introduce WiFi like white space link implementation on USRP and 
link protocols. ~\cite{cui2013leveraging} discuss the point to point communication
in multiband scenario. In~\cite{filippini2013new}, white space band application is 
discussed in cognitive radio network for reducing maintenance cost. 
In~\cite{deb2009dynamic}, the white space is proposed to increase the 
data rates through spectrum allocation. 
In contrast, we focus on reducing the deployment cost with customer 
constraints in mesh nodes number.






