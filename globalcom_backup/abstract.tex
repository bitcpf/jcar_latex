\begin{abstract}

Wireless mesh networks were previously thought to be an ideal solution for
large-scale Internet connectivity in metropolitan areas.  However, in-field
trials revealed that the node spacing required for WiFi propagation 
induced a prohibitive cost model for network carriers to deploy. The digitization 
of TV channels and new FCC regulations have reapportioned spectrum for data 
networks with far greater range than WiFi due to lower carrier frequencies. 
Also, channel occupancy of change the performance of wireless network across 
both ISM bands and white space bands. In this work, we consider how these white 
space bands can be leveraged in large-scale wireless mesh network deployments 
with in-field measured channel capacity. In particular, we present an integer 
linear programming model to leverage diverse propagation characteristics and 
the  channel occupancy of white space and WiFi bands to deploy optimal WhiteMesh 
networks. Since such problem is known to be NP-hard, we design a measurement driven 
heuristic algorithm, Band-based Path Selection (BPS), which we show approaches 
the performance of the optimal solution with reduced complexity.  We additionally 
compare the performance of BPS against two well-known multi-channel, multi-radio 
deployment algorithms across a range of scenarios spanning those typical for 
rural areas to urban areas. In doing so, we achieve up to 160\% traffic achieved 
gateways gain of these existing multi-channel, multi-radio algorithms, which are 
agnostic to diverse propagation characteristics across bands.  Moreover, we show 
that, with similar channel resources and bandwidth, the joint use of WiFi and 
white space bands can achieve a served user demand of 170\% that of mesh networks 
with only WiFi bands or white space bands, respectively. Further, through the 
result, we leverage the channel occupancy and spacing impacts on white mesh
network and study the general rules of band selection.



\end{abstract}
