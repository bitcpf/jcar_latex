\section{Introduction}
\label{sec:introduction}

% Background multiband 
% Mesh Netrork
% Mesh Network traditonal utibility
% White space benefit challenges
% Contributions about new data
% Paper organization
% Wireless Network

About a decade ago, numerous cities solicited proposals from
network carriers for exclusive rights to deploy city-wide WiFi,
spanning hundreds of square miles.  
While the vast majority of the resulting used
a wireless mesh topology, initial field tests revealed that the 
actual WiFi propagation could not achieve the proposed mesh node
spacing. As a result, many network carriers opted to pay millions of 
dollars in penalties rather than face the exponentially-increasing
deployment costs (e.g., Houston~\cite{cnet_aug07} and 
Philadelphia~\cite{arstechnica_may08}). Thus, while a few mesh 
networks have been deployed in certain communities~\cite{CRSK06},
wireless mesh networks have largely been unsuccessful in achieving 
the scale of what was once anticipated~\cite{taps}.

Around the same time, the digital TV transition created more
spectrum for use with data networks~\cite{fccwhitespace}. These white 
space bands operate in available channels from 54-806 MHz, having
increased propagation range as compared to WiFi~\cite{balanis2012antenna}. 
Hence, the FCC has identified rural areas as a key application for 
white space networks since the reduced
population from major metropolitan areas allows a greater service area
per backhaul device without saturating wireless capacity.
At the same time, the new allowed white space channels
offers more capacity all over the US. The propagation diversity
and additional channel capacity are the two key impacts on
wireless network performance of white space bands.
% Previous work on channel occupancy and channel assignment
The additional channel capacity vary according to FFC regulation and
existing channel occupancy. As shown in Google database~\cite{googledatabase}, 
the additional the number of available white
space frequency channels vary from city to city in US. 
Existing channel occupancy discussed in previous work~\cite{pcuiwinmee} 
has shown that the occupancy of frequency impacts on wireless network deployment.
Naturally, the question arises for improving the performance
as well as the optimization of utilization: {\it how can the emerging white space bands improve 
large-scale mesh network deployments?}  While much work has been done 
on deploying multihop wireless networks with multiple channels and 
radios, the differences in propagation have not been exploited in their 
models~\cite{tang2005interference, long2013fair,doraghinejad2014channel}, 
which could be {\it the} fundamental issue for the success of mesh 
networks going forward.

% Paper topic
In this paper, we leverage the diversity in propagation and channel occupancy of
white space and WiFi bands in the planning and deployment
of large-scale wireless mesh networks. To do so, we first form an
integer linear program to jointly exploit white space and WiFi 
bands for optimal WhiteMesh topologies in channel assignment. 
Second, since similar problem formulations have been shown to 
be NP-hard problem~\cite{jain2005impact,doraghinejad2014channel}, 
we design a heuristic measurement driven algorithm, Band-based 
Path Selection (BPS) based on mathematical analysis to solve the problem. 
We then apply the approaching method in multiple scenarios with in-field
measurement data. Across a wide range of scenarios, including 
network size, population distribution, deployment distance gap, we 
exploit the general rules of emerging white space bands in mesh networks. 
The performance of our scheme is compared against two well-known 
multi-channel, multi-radio channel assignment algorithms across 
these scenarios, including those typical for rural areas as well 
as urban settings. We further discuss the channel occupancy 
impacts on wireless networks and show the comparison of our
algorithm and previous methods in typical scenarios. 
Finally, we quantify the degree to which the joint use of 
both band types can improve the performance of wireless mesh networks.


% Need some calculation data to show the improvement
The main contributions of our work are as follows:
\begin{itemize}
\item We analyze the white space bands application in wireless 
network deployment and develop an optimization framework based on integer
linear programming to jointly leverage white space and WiFi bands
to advantages and disadvantages in wireless mesh networks with measured 
channel occupancy.  
\item We build a heuristic measurement driven algorithm, Band-based Path 
Selection (BPS), which considers the diverse propagation, overall interference 
level of WiFi and white space bands with measurement adjust.  
\item We perform extensive analysis across offered loads,
network sizes, mesh nodes spacing and WiFi/white space band combinations, to 
compare against previous multichannel multiradio algorithms. And we
further exploit the general rules of white space bands application in wireless network.  
\item We discuss the channel occupancy and mesh spacing impacts 
on the performance given similar channel resources (bandwidth and
transmission power), We show the improvement of our BPS in typical
configurations up to \%180 vs. previous multichannel algorithms.
\end{itemize}

The remainder of this paper is organized as follows. In Section
~\ref{sec:problemformulation}, we introduce WhiteMesh network 
topologies, describe the challenge of diverse frequency band 
allocation, and formulate the integer linear programming model. 
In Section~\ref{sec:wmalgorithms}, we analyze the WhiteMesh network
and develop a heuristic algorithms which consider which bands 
and multihop paths to select in a WhiteMesh topology.  We then 
evaluate the performance of the heuristic algorithm versus the 
upper bound of the optimal solution and compare their performance 
against two well-known multi-channel, multi-radio algorithms in Section
~\ref{sec:experimentdesign} in several scenarios and analyze the 
result for answering where WhiteMesh is better. Finally, we discuss 
related work in Section~\ref{sec:related} and conclude in Section~\ref{sec:conclusion}.
