\section{Conclusion}
\label{sec:conclusion}
In this paper, we exploited the joint use of WiFi and white space bands for 
improving the served user demand of wireless mesh networks.  To do so, we
used an integer programming model to find optimal WhiteMesh topologies.  We
then constructed two heuristic algorithms, Growing Spanning Tree and Band-based
Path Selection, to achieve similar performance with reduced complexity. Through 
extensive analysis across varying offered loads, network sizes, and white space 
channel availability, we show that our algorithms can achieve up to 320\% gains 
from previous multi-channel, multi-radio solutions since we leverage diverse 
propagation characteristics offered by WiFi and white space bands.  Moreover,
we quantify the degree to which the joint use of these bands can improve the served
user demand. Our BPS algorithm shows that WhiteMesh topologies can achieve up to 
140\% more gateway goodput than similar WiFi- or white-space-only configurations.
%In future work, we will adapt our algorithms to be used with dynamically-changing
%network conditions, in the field on large-scale WhiteMesh networks.

%investigated the channel assignment in multi-band scenario to leverage the propagation incluence for mesh network applications. 
%We have presented the multi-band mesh network architecture, a new defination of path interference over network, and analyze the advantages and disadvantages of white space bands.
%According to the analysis, we formally propose Best Path Selection and Growing Spanning Tree algorithms for channel assignment in multi-band network. Simulation results show that our scheme outperforms the existing scheme substantially.
%Dynamic and distributed algorithms for multi-band channel assignment problems will be of our future work.

