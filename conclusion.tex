\section{Conclusion}
\label{sec:conclusion}
In this paper, we jointly considered the use of WiFi and white space bands 
in WhiteMesh network deployments. 
We first performed our spectral analysis in the DFW metropolitan area and surrounding 
areas. We then proposed a measurement-driven Multiband Access Point Estimation 
(MAPE) framework to estimate the number of access points required in a given region for 
wireless access network. Further, we proposed measurement-driven Band-based Path Selection (BPS) algorithm 
for the backhual tier channel assignment. 
%
%Further, we investigate different 
%band combinations in two population densities to show that greater access to white space 
%channels have greater total savings of mesh nodes when the total number of channels used 
%in the network is fixed (i.e., given a total number of allowable WiFi and white space channels). 
Through extensive analysis across varying population density and channel combinations across bands, 
we showed that white space bands can reduce the number of access points by 1650\%
and 660\% in rural and sparse urban areas. However, the same cost savings are not achieved in dense urban 
and downtown type area. As the population and spectrum utilization increase, the cost savings of 
white space bands diminished to the point that WiFi-only channel combinations can be optimal.
The simulation showed that our BPS algorithm can achieve 180\% of the served traffic flow versus previous 
multi-channel, multi-radio solutions in multiband scenarios, since we leveraged diverse propagation 
characteristics offered by WiFi and white space bands. Moreover, we quantified the degree to which the joint 
use of these bands can improve the served user demand. Our BPS algorithm showed that WhiteMesh topologies 
can achieve up to 160\% of the served traffic flow of similar WiFi or white-space-only configurations.

% Future work
%In the future, we will consider the heterogeneous access points and traffic demand scenarios
%in wireless network deployments.


