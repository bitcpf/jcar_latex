\subsection{Path Interference Induced on the Network}
\label{subsec:PEN}

In WhiteMesh networks, multihop paths can be intermixed with WiFi 
for more spacial reuse and white space bands with less hops.  
To deal with the tradeoff, we consider
analyze the band choices reduce the number of hops along a path and the 
aggregate level of interference that hop-by-hop path choices have
on the network (i.e., Path Interference induced on the Network).

Mesh nodes closer to the gateway generally achieve
greater levels of throughput at sufficiently high offered loads. 
To combat such starvation effects, we treat each flow with equal priority in the network when
assigning channels. In the worst case, all nodes along a particular path have equal 
time shares for contending links (i.e., intra-path interference).
We start the channel assignment assuming that $h$ mesh nodes are demanding
traffic from each hop of an $h$-hop path to the gateway. If each link along the 
path uses orthogonal channels, then each link could be active simultaneously,
otherwise they will complete with each other. 
We note each node along the path had traffic demand $T_d$, obviously the bottleneck 
link along the path would be the one closest to the gateway, and then next. 
Thus, the total traffic along the path $h \cdot T_d$ must be less than the 
bottleneck link's capacity $\delta$ estimated from the measurements. In such a scenario, the $h$-hop mesh node 
would achieve the minimum served demand, which we define as the network efficiency. 
In general, the active time per link for an $h$-hop mesh node can be represented 
by $1,\frac{h-1}{h},\frac{h-2}{h}\cdots \frac{1}{h}$. The summation of all active 
times for each mesh node along the path is considered the intra-path network cost.

Considering only intra-path interference, using lower carrier frequencies allows a
reduction in hop count and increase in the network efficiency of each mesh node along
the $h$-hop path. However, a lower carrier frequency will induce greater interference
to other paths to the gateway (i.e., inter-path interference). 
Fig.~\ref{fig:interferencerange} depicts such an example where
links in different bands are represented by circles for 450 MHz, rectangles for
2.4 GHz, and triangles for the nodes which can choose between the two.
Nodes $A$ and $C$ could be connected through two 2.4-GHz links or a single 450-MHz link.
With 2.4 GHz, the interfering distance will be less than using 450 MHz. For example, only 
link $D,E$ will suffer from interference, whereas $H,I$ would not. However, with 450 MHz,
link $A,C$ would interfere with links $F,G$, $M,L$, and $K,J$. At each time unit, the number of
links interfering with the active links along a path would be the inter-path network cost.

When an $h$-hop flow is transmitted to a destination node, it prevents 
activity on a number of links in the same frequency via the protocol model. 
The active time on a single link can be noted as 
$\frac{T}{\gamma_h}$. 
An interfering link from the conflict matrix $F$ counts as $I_h$ per unit time
and contributes to the network cost in terms of:
$\frac{hT}{\gamma_1}\cdot I_1 + \frac{(h-1)T}{\gamma_2}\cdot I_2 \cdots \frac{T}{\gamma_h}\cdot I_h$.
Then, the traffic transmitted in a unit of network cost for the $h$-hop node is:
\begin{equation}
\label{eq:originpen}
E_{\eta}=\frac{T}{\sum_{i \in h}\frac{(h-i+1)\cdot T}{\gamma_i}\cdot I_i }
\end{equation}
Using network efficiency, the equation simplifies to:
\begin{equation}
\label{eq:pen}
E_{\eta}=\frac{\gamma}{\sum_{i \in h} (h-i+1)\cdot I_i}
\end{equation}

The network efficiency is the amount of traffic that could be 
offered on a path per unit time. With multiple channels from the same band,
$I_i$ will not change due to the common communication range. With multiple
bands, $I_i$ depends on the band choice due to the communication range diversity.  
This network efficiency jointly considers hop count and interference. We define
the Path Interference induced on the Network (PIN) as the denominator of Eq.~\ref{eq:pen},
which represents the sum of all interfering links in the network by a given path. 
PIN is used to quantify the current state of channel for channel assignment
across WiFi and white space bands.
To determine when the lower carrier frequency will be better than two or more hops at a
higher carrier frequency, we consider the average interference $\bar{I}$ of a given path
at the higher frequency.  The problem could be formulated as:
\begin{equation}
\label{eq:benefit}
\frac{\gamma}{\frac{h(h-1)}{2}\cdot \bar{I}+I_x} \geq \frac{\gamma}{\frac{h(h+1)}{2}\cdot \bar{I}}
\end{equation}

Here, from Eq.~ref{eq:benefit} when $I_x \leq 2\cdot h\bar{I}$, the performacne of a lower-frequency link  
is better than two higher-frequency hops for the same destination node. $I_x$ is also a parameter of hop count 
in Eq.~\ref{eq:pen}. When the hop count is lower which closer to the gateway node, the threshold 
would be more strict since the interference would have a greater effect on the performance.


